\begin{table}[H]
  \begin{center}
    \begin{tabularx}{0.75\textwidth}{ X }
      \toprule
      \textbf{Codigo:} CP001
      \makebox[3cm][r]{}
      \makebox[6cm][r]{\textbf{Historia de Usuario:} US01} \\

      \addlinespace
      \textbf{Nombre:} Verificar la lista de lugares. \\

      \addlinespace
      \textbf{Descripción:} Validar que un usuario visitante puede ver la lista de lugares cuando ingresa al menú \emph{lugares}. \\

      \addlinespace
      \textbf{Condiciones de Ejecución:} \\
      \tab \textbf{a.} El usuario no debe estar registrado. \\
      \tab \textbf{b.} Deben existir lugares registrados en el sistema.\\

      \addlinespace
      \textbf{Entradas / Pasos de Ejecución:}  \\
      \tab \textbf{1.} Hacer tap sobre el botón \emph{menú} en la esquina superior-izquierda. \\
      \tab \textbf{2.} Seleccionar el menú \emph{lugares}.\\

      \addlinespace
      \textbf{Resultado Esperado:} El Usuario debe ver una lista con los lugares registrados en la Condición de Ejecución \emph{b}.  \\

      \addlinespace
      \textbf{Evaluación de la Prueba:} Prueba exitosa. \\

      \bottomrule
    \end{tabularx}
    \caption{Prueba de Aceptación - CP001}
    \label{tab:CP001}
  \end{center}
\end{table}

\begin{table}[H]
  \begin{center}
    \begin{tabularx}{0.75\textwidth}{ X }
      \toprule
      \textbf{Codigo:} CP002
      \makebox[3cm][r]{}
      \makebox[6cm][r]{\textbf{Historia de Usuario:} US01} \\

      \addlinespace
      \textbf{Nombre:} Verificar la busqueda de lugares. \\

      \addlinespace
      \textbf{Descripción:} Validar que un usuario puede filtrar un lugar de la lista mediante el nombre. \\

      \addlinespace
      \textbf{Condiciones de Ejecución:} Ingresar en la base de datos el lugar con nombre ``MEMI''. \\

      \addlinespace
      \textbf{Entradas / Pasos de Ejecución:}  \\
      \tab \textbf{1.} Hacer tap sobre el botón \emph{menú} en la esquina superior-izquierda. \\
      \tab \textbf{2.} Seleccionar el menú \emph{lugares}.\\
      \tab \textbf{3.} Ingresar el nombre ``MEMI'' en el cajón de búsqueda.\\


      \addlinespace
      \textbf{Resultado Esperado:} Se debe mostrar un solo item en la lista de lugares con el nombre ``MEMI'' desplegado.\\

      \addlinespace
      \textbf{Evaluación de la Prueba:} Prueba exitosa. \\

      \bottomrule
    \end{tabularx}
    \caption{Prueba de Aceptación - CP002}
    \label{tab:CP002}
  \end{center}
\end{table}

\begin{table}[H]
  \begin{center}
    \begin{tabularx}{0.75\textwidth}{ X }
      \toprule
      \textbf{Codigo:} CP003
      \makebox[3cm][r]{}
      \makebox[6cm][r]{\textbf{Historia de Usuario:} US02} \\

      \addlinespace
      \textbf{Nombre:} Verificar la información de un lugar. \\

      \addlinespace
      \textbf{Descripción:} Validar que un usuario puede ver la descripción, el teléfono, el nivel y la foto de un lugar. \\

      \addlinespace
      \textbf{Condiciones de Ejecución:} Ingresar en la base de datos el lugar con nombre ``MEMI'' con su información completa. \\

      \addlinespace
      \textbf{Entradas / Pasos de Ejecución:}  \\
      \tab \textbf{1.} Hacer tap sobre el botón \emph{menú} en la esquina superior-izquierda. \\
      \tab \textbf{2.} Seleccionar el menú \emph{lugares}.\\
      \tab \textbf{3.} En la lista de lugares seleccionar el item ``MEMI''.\\

      \addlinespace
      \textbf{Resultado Esperado:} Se debe mostrar una pantalla con la foto del lugar ``MEMI'', su descripción, el teléfono y el nivel.\\

      \addlinespace
      \textbf{Evaluación de la Prueba:} Prueba exitosa. \\

      \bottomrule
    \end{tabularx}
    \caption{Prueba de Aceptación - CP003}
    \label{tab:CP003}
  \end{center}
\end{table}
