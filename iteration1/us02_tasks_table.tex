\begin{table}[H]
  \begin{center}
    \begin{tabularx}{\textwidth}{ c  X  C{2.3cm} }
      \toprule
        \textbf{Código} &
        \multicolumn{1}{c}{\textbf{Tarea}} &
        \textbf{Estimación [dias]}\\

      \midrule
        RF001
        &
        Crear un archivo shapefile con información inicial de lugares principales dentro el campus de la UMSS.
        &
        1 \\

      \addlinespace
        RF002
        &
        Crear una base de datos que pueda manejar información geoespacial.
        &
        1 \\

      \addlinespace
        RF003
        &
        Popular la base de datos creada en RF002 con la información de RF001.
        &
        0.5 \\

      \addlinespace
        RF004
        &
        El usuario puede ver una lista con los lugares creados.
        &
        2 \\

      \addlinespace
        RF005
        &
        El usuario deberá poder ingresar el nombre de un lugar para filtrar los lugares existentes.
        &
        0.5 \\

      \addlinespace
        RF006
        &
        El usuario deberá poder ver la información de un lugar al hacer tap sobre el nombre del lugar en la lista.
        &
        1 \\

      \addlinespace
        TS001
        &
        Crear pruebas de funcionalidad del US02.
        &
        1 \\

      \addlinespace
      \midrule
        & \multicolumn{1}{R{7cm}}{\textbf{Total: }}
        & 7 \\

      \bottomrule
    \end{tabularx}
    \caption{Tareas de la US02}
    \label{tab:us02_tasks}
  \end{center}
\end{table}
