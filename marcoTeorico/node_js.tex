\section{Node JS}
\label{sec:node_js}
  Node.js apareció en 2009 y está construido sobre el Motor de JavaScript de Google ``V8'' que fue sacado del browser y aplicado en el servidor.

  Para desarrollar en el lado del browser (cliente) el programador sólo tiene disponible JavaScript como lenguaje de desarrollo pero en el lado del servidor existen muchas alternativas (Ruby, C\#, Python, Java, etc.), JavaScript no estaba disponible.\\

  Node se beneficia del Motor de JavaScript ``V8'' ya que este es rápido y tiene integrado un sistema para manejar las instrucciones de forma asincrónica, pero el mayor beneficio y el porqué Node adquirió una gran popularidad es la facilidad de compartir codigo entre el cliente (browser) y el servidor.\\

  Node.js provee características pero estas pueden parecer complicadas o que necesitan más instrucciones de las necesarias para llevar a cabo acciones que ya son comunes en la creación de aplicación en lado del servidor, por ejemplo a la hora de crear un servidor web, Node se popularizó en gran medida por poder crear servidores web personalizables pero como ya dijimos esto tiene su grado de complejidad, acá es donde entra en acción \emph{Express.js}.
