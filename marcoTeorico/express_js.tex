\section{ExpressJS}
\label{sec:express_js}
  \emph{ExpressJS} es un framework que está construido sobre la funcionalidad de servidor web de \emph{NodeJS}, \emph{ExpressJS} ayuda a simplificar el API de Node y añadir nuevas características, diseñadas para mejorar y facilitar la organización de una aplicación \emph{Express}.\\

  El Cliente (navegador web, aplicación móvil, etc) envía una petición web y el servidor web de \emph{NodeJS} maneja los protocolos web, leyendolos y enviandolos a una aplicación \emph{ExpressJS} que se encarga de añadir características a la petición y espera la respuesta del ``Middleware Stack'', la función responde a la llamada y el servidor HTTP de Node envía la respuesta mediante los protocolos web al Cliente.\\

  Para escribir un servidor web con \emph{ExpressJS}  no es necesario una gran función para manejar un request, \emph{ExpressJS} contiene utilidades que permite escribir funciones más pequeñas para facilitar el manejo de las peticiones web, haciendo uso de ``middleware'' y ``routing''.

  \subsection{Middleware}
  \label{sub:middleware}
    \emph{NodeJS} maneja una función para trabajar con una petición web, en cambio \emph{ExpressJS} maneja la llamada con varias funciones, cada función se encarga de una pequeña parte del trabajo. Estas pequeñas funciones que manejan la petición web se denominan \emph{Middleware functions} o simplemente \emph{Middleware}.

  %  end sub section middleware

  \subsection{Routing}
  \label{sub:routing}
    Muy parecido al \emph{Middleware}, el \emph{Routing} se encarga de partir una petición web monolítica en pequeñas piezas, pero a diferencia del Middleware, estos manejadores de peticiones se ejecutan condicionalmente dependiendo del URL y la petición HTTP (GET, POST, DELETE) que el cliente envía.\\

  %  end sub section routing

  \emph{ExpressJS} es bastante extensible y cuenta con gran popularidad en la comunidad de desarrollo, la cual provee herramientas para renderizar dinámicamente HTML o interfaces para comunicarse con Bases de Datos, por ejemplo para manejar la conexión y las llamadas a la base de datos PostgreSQL en el presente proyecto se uso la libreria \emph{knex}.


  % \begin{verbatim}
  %   database.any("SELECT * FROM users WHERE id = $1", [userId])
  %     .then(function (data) {
  %         response.send(data.name);
  %     });
  % \end{verbatim}

% section Express JS (end)
