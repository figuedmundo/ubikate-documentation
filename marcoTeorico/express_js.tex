\section{Express JS}
\label{sec:express_js}
  Express.js es un framework que esta construido sobre la funcionalidad de servidor web de Node.js, Express.js ayuda a simplificar el API de Node y a\~nadir nuevas caracter\'isticas, dise\~nadas para mejorar y facilitar la organizaci\'on de una aplicaci\'on \emph{Express}.\\

  El Cliente (navegador web, aplicacion movil, etc) envia una peticion web y el servidor web de Node.js maneja los protocolos web, leyendolos y enviandolos a una aplicacion \emph{Express} que se encarga de a\~nadir caracteristicas a la peticion y espera la respuesta del ``Middleware Stack'', la funcion responde a la llamada y el servidor HTTP de Node envia la respuesta mediante los protocolos web al Cliente.\\

  Para escribir un servidor web con Express no es necesario una gran funcion para manejar un request, Express contiene utilidades que permite escribir funciones mas peque\~nas para facilitar el manejo de las peticiones web, asiendo uso de ``middleware'' y ``routing''.

  \subsection{Middleware}
  \label{sub:middleware}
    Node.js maneja una funci\'on para trabajar con una peticion web, encambio \emph{Express} maneja la llamada con varias funciones, cada funcion se encarga de una peque\~na parte del trabajo. Estas peque\~nas funciones que manejan la peticion web se denomina \emph{Middleware functions} o Middleware.

  %  end sub section middleware

  \subsection{Routing}
  \label{sub:routing}
    Muy parecido al Middleware, el Routing se encarga de partir una funcion de peticion web monolitica en peque\~nas piezas, pero a diferencia del Middleware, estos menajadores peticiones se ejecutan condicionalmente dependidiendo del URL y el metodo HTTP (GET, POST, DELETE) que el cliente envia.\\

  %  end sub section routing

  Express.js es bastante extensible y cuenta con gran popularidad en la comunidad de desarrollo, la cual provee herramientas para renderizar dinamicamente HTML o interfaces para comunicarse con Bases de Datos, por ejemplo para manejar la coneccion y llamadas a la base de datos PostgreSQL se uso la libreria \emph{knex} .

  \begin{verbatim}
    database.any("SELECT * FROM users WHERE id = $1", [userId])
      .then(function (data) {
          response.send(data.name);
      });
  \end{verbatim}

% section Express JS (end)
