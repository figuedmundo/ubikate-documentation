\begin{table}[H]
  \begin{center}
    \begin{tabularx}{0.75\textwidth}{ X }
      \toprule
      \textbf{Número de Tarea:} T014
      \makebox[1cm][r]{}
      \makebox[6cm][r]{\textbf{Historia de Usuario:} US04} \\

      \addlinespace
      \textbf{Descripción:} Crear un archivo shapefile con las rutas  dentro el campus de la UMSS. \\

      \addlinespace
      \textbf{Tipo de Tarea:} Desarrollo
      \makebox[6cm][r]{\textbf{Estimación [dias]:} 2} \\

      \addlinespace
      \textbf{Programador Responsable:} Edmundo Figueroa \\

      \bottomrule
    \end{tabularx}
    \caption{Tarea de Ingeniería - T014}
    \label{tab:T014}
  \end{center}
\end{table}


\begin{table}[H]
  \begin{center}
    \begin{tabularx}{0.75\textwidth}{ X }
      \toprule
      \textbf{Número de Tarea:} T015
      \makebox[1cm][r]{}
      \makebox[6cm][r]{\textbf{Historia de Usuario:} US04} \\

      \addlinespace
      \textbf{Descripción:} Preparar la base de datos para manejar información geográfica de rutas. \\

      \addlinespace
      \textbf{Tipo de Tarea:} Desarrollo
      % \makebox[1cm][r]{}
      \makebox[6cm][r]{\textbf{Estimación [dias]:} 1} \\

      \addlinespace
      \textbf{Programador Responsable:} Edmundo Figueroa \\

      \bottomrule
    \end{tabularx}
    \caption{Tarea de Ingeniería - T015}
    \label{tab:T015}
  \end{center}
\end{table}

\begin{table}[H]
  \begin{center}
    \begin{tabularx}{0.75\textwidth}{ X }
      \toprule
      \textbf{Número de Tarea:} T016
      \makebox[1cm][r]{}
      \makebox[6cm][r]{\textbf{Historia de Usuario:} US04} \\

      \addlinespace
      \textbf{Descripción:} Mostrar un marcador en el mapa sobre la posicion actual del usuario. \\

      \addlinespace
      \textbf{Tipo de Tarea:} Desarrollo
      \makebox[6cm][r]{\textbf{Estimación [dias]:} 0.5} \\

      \addlinespace
      \textbf{Programador Responsable:} Edmundo Figueroa \\

      \bottomrule
    \end{tabularx}
    \caption{Tarea de Ingeniería - T016}
    \label{tab:T016}
  \end{center}
\end{table}

\begin{table}[H]
  \begin{center}
    \begin{tabularx}{0.75\textwidth}{ X }
      \toprule
      \textbf{Número de Tarea:} T017
      \makebox[1cm][r]{}
      \makebox[6cm][r]{\textbf{Historia de Usuario:} US04} \\

      \addlinespace
      \textbf{Descripción:} Desarrollar un módulo que encuentra la ruta más corta usando la base de datos con información geográfica ruteable. \\

      \addlinespace
      \textbf{Tipo de Tarea:} Desarrollo
      \makebox[6cm][r]{\textbf{Estimación [dias]:} 2} \\

      \addlinespace
      \textbf{Programador Responsable:} Edmundo Figueroa \\

      \bottomrule
    \end{tabularx}
    \caption{Tarea de Ingeniería - T017}
    \label{tab:T017}
  \end{center}
\end{table}


\begin{table}[H]
  \begin{center}
    \begin{tabularx}{0.75\textwidth}{ X }
      \toprule
      \textbf{Número de Tarea:} T018
      \makebox[1cm][r]{}
      \makebox[6cm][r]{\textbf{Historia de Usuario:} US04} \\

      \addlinespace
      \textbf{Descripción:} Mostrar una línea roja que une el marcador de la posición del usuario con el marcador del lugar. \\

      \addlinespace
      \textbf{Tipo de Tarea:} Desarrollo
      \makebox[6cm][r]{\textbf{Estimación [dias]:} 1} \\

      \addlinespace
      \textbf{Programador Responsable:} Edmundo Figueroa \\

      \bottomrule
    \end{tabularx}
    \caption{Tarea de Ingeniería - T018}
    \label{tab:T018}
  \end{center}
\end{table}
