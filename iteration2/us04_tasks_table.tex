\begin{table}[H]
  \begin{center}
    \begin{tabularx}{\textwidth}{ c  X  C{2.3cm} }
      \toprule
        \textbf{Código} &
        \multicolumn{1}{c}{\textbf{Tarea}} &
        \textbf{Estimación [dias]}\\

      \midrule
        T011
        &
        Crear un archivo shapefile con información inicial de lugares principales dentro el campus de la UMSS.
        &
        2 \\

      \addlinespace
        T012
        &
        Preparar la base de datos para manejar información geográfica de rutas.
        &
        1 \\
      %
      % \addlinespace
      %   T013
      %   &
      %   Investigar e instalar una herramienta que permita usar un servicio de mapas.
      %   &
      %   1 \\

      % \addlinespace
      %   T014
      %   &
      %   El usuario puede ver un mapa usando un servicio del campus de la UMSS.
      %   &
      %   0.5 \\

      % \addlinespace
      %   T015
      %   &
      %   El usuario puede ver un marcador sobre el lugar.
      %   &
      %   0.5 \\
      %
      % \addlinespace
      %   T016
      %   &
      %   El marcador tiene información básica del lugar, nombre, piso.
      %   &
      %   0.5 \\

      \addlinespace
        T017
        &
        El usuario puede ver un marcador mostrando el lugar actual donde se encuentra (el usuario).
        &
        0.5 \\

      \addlinespace
        T018
        &
        Desarrollar un módulo que encuentra la ruta más corta usando la base de datos con información geográfica ruteable de T012.
        &
        2 \\

      \addlinespace
        T019
        &
        El usuario puede ver una línea roja que une el marcador de la posición del usuario con el marcador del lugar.
        &
        1 \\

      % \addlinespace
      %   TS003
      %   &
      %   Crear pruebas de funcionalidad del US04.
      %   &
      %   1 \\

      \addlinespace
      \midrule
        & \multicolumn{1}{R{7cm}}{\textbf{Total: }}
        & 10 \\

      \bottomrule
    \end{tabularx}
    \caption{Tareas del US04}
    \label{tab:us04_tasks}
  \end{center}
\end{table}
