\subsubsection{Resultado de las pruebas de la Iteración 2}

Al finalizar la Iteración 2, se ejecutaron todas las pruebas escritas durante la presente y la anterior iteración, en el cuadro \ref{tab:regresion_2} se puede ver el detalle.


\begin{table}[H]
  \begin{center}
    \begin{tabularx}{0.8\textwidth}{ c  X  c }
      \toprule
        \textbf{Código} &
        \multicolumn{1}{c}{\textbf{Título de la Prueba}} &
        \textbf{Resultado}\\

\midrule
CP001
&
Verificar la lista de lugares.
&
Exitoso \\

% \addlinespace
CP002
&
Verificar la busqueda de lugares.
&
Exitoso \\

% \addlinespace
CP003
&
Verificar la información de un lugar.
&
Exitoso \\

% \addlinespace
CP004
&
Verificar la lista de lugares cuando se busca un lugar no registrado.
&
Exitoso \\

% \addlinespace
CP005
&
Verificar la información de un lugar mediante el URI.
&
Exitoso \\

% \addlinespace
CP006
&
Verificar que el \emph{Menú} sea desplegado dinámicamente.
&
Exitoso \\

% \addlinespace
CP007
&
Verificar el marcador del lugar sobre el mapa del campus Universitario.
&
Exitoso \\

% \addlinespace
CP008
&
Verificar que el marcador del lugar muestre la información.
&
Exitoso \\


% \addlinespace
CP009
&
Verificar que una línea roja muestra la ruta óptima hacia el lugar.
&
Exitoso \\

% \addlinespace
CP010
&
Verificar que la ruta óptima se muestre dentro del campus Universitario.
&
Exitoso \\


% \addlinespace
CP011
&
Verificar que se puede cerrar el mapa.
&
Exitoso \\


      \bottomrule
    \end{tabularx}
    \caption{Pruebas de regresión de la Iteración 2}
    \label{tab:regresion_2}
  \end{center}
\end{table}


\begin{itemize}
  \item Se ejecutaron 6 pruebas de funcionalidad positiva, todas pasaron exitosamente.
  \item Se ejecutaron 3 pruebas de funcionalidad negativa, todas pasaron exitosamente.
  \item Se ejecutaron 2 prueba de usabilidad, todas pasaron exitosamente.
\end{itemize}
