\begin{table}[H]
  \begin{center}
    \begin{tabularx}{0.75\textwidth}{ X }
      \toprule
      \textbf{Codigo:} CP007
      \makebox[3cm][r]{}
      \makebox[6cm][r]{\textbf{Historia de Usuario:} US03} \\

      \addlinespace
      \textbf{Tipo:} Prueba de Funcionalidad - Positiva. \\

      \addlinespace
      \textbf{Nombre:} Verificar el marcador del lugar sobre el mapa del campus Universitario. \\

      \addlinespace
      \textbf{Descripción:} Validar que un usuario puede ver un marcador sobre el mapa indicando la posición del lugar dentro del campus Universitario cuando se pide la ruta óptima al lugar. \\

      \addlinespace
      \textbf{Condiciones de Ejecución:}
      Un lugar registrado dentro del campus Universitario. \\

      \addlinespace
      \textbf{Entradas / Pasos de Ejecución:}  \\
      \tab \textbf{1.} Seleccionar el menú \emph{Lugares}. \\
      \tab \textbf{2.} Buscar el lugar registrado.\\
      \tab \textbf{3.} Seleccionar la opción para buscar la ruta al lugar. \\


      \addlinespace
      \textbf{Resultado Esperado:} Una vista del mapa del campus Universitario debe ser desplegado con un marcador sobre la posición del lugar en el centro de la pantalla.  \\

      \addlinespace
      \textbf{Evaluación de la Prueba:} Prueba exitosa. \\

      \bottomrule
    \end{tabularx}
    \caption{Prueba de Aceptación - CP007}
    \label{tab:CP007}
  \end{center}
\end{table}


\begin{table}[H]
  \begin{center}
    \begin{tabularx}{0.75\textwidth}{ X }
      \toprule
      \textbf{Codigo:} CP008
      \makebox[3cm][r]{}
      \makebox[6cm][r]{\textbf{Historia de Usuario:} US03} \\

      \addlinespace
      \textbf{Tipo:} Prueba de Funcionalidad - Positiva. \\

      \addlinespace
      \textbf{Nombre:} Verificar que el marcador del lugar muestre la información. \\

      \addlinespace
      \textbf{Descripción:} Validar que un usuario puede ver un \emph{tooltip} sobre el marcador  mostrando el nombre del lugar, teléfono y nivel. \\

      \addlinespace
      \textbf{Condiciones de Ejecución:}
      Un lugar registrado dentro del campus Universitario. \\

      \addlinespace
      \textbf{Entradas / Pasos de Ejecución:}  \\
      \tab \textbf{1.} Seleccionar el menú \emph{Lugares}. \\
      \tab \textbf{2.} Buscar el lugar registrado.\\
      \tab \textbf{3.} Seleccionar la opción para buscar la ruta al lugar. \\
      \tab \textbf{4.} Presionar el marcador mostrado sobre el mapa. \\


      \addlinespace
      \textbf{Resultado Esperado:} Un \emph{tooltip} o cajon de información debe de aparecer sobre el marcador mostrando el nombre, teléfono y nivel del lugar.  \\

      \addlinespace
      \textbf{Evaluación de la Prueba:} Prueba exitosa. \\

      \bottomrule
    \end{tabularx}
    \caption{Prueba de Aceptación - CP008}
    \label{tab:CP008}
  \end{center}
\end{table}




\begin{table}[H]
  \begin{center}
    \begin{tabularx}{0.75\textwidth}{ X }
      \toprule
      \textbf{Codigo:} CP009
      \makebox[3cm][r]{}
      \makebox[6cm][r]{\textbf{Historia de Usuario:} US04} \\

      \addlinespace
      \textbf{Tipo:} Prueba de Funcionalidad - Positiva. \\

      \addlinespace
      \textbf{Nombre:} Verificar que una línea roja muestra la ruta óptima hacia el lugar. \\

      \addlinespace
      \textbf{Descripción:} Validar que el usuario puede ver un una línea roja mostrando la ruta óptima entre la posición del usuario y el lugar. \\

      \addlinespace
      \textbf{Condiciones de Ejecución:}
      Registrar un lugar dentro del campus Universitario. \\

      \addlinespace
      \textbf{Entradas / Pasos de Ejecución:}  \\
      \tab \textbf{1.} Seleccionar el menú \emph{Lugares}. \\
      \tab \textbf{2.} Buscar el lugar registrado.\\
      \tab \textbf{3.} Seleccionar la opción para buscar la ruta al lugar. \\


      \addlinespace
      \textbf{Resultado Esperado:} Una vista del mapa del campus Universitario debe ser desplegado con una línea roja uniendo el marcador sobre la posición del usuario y la posición del lugar.  \\

      \addlinespace
      \textbf{Evaluación de la Prueba:} Prueba exitosa. \\

      \bottomrule
    \end{tabularx}
    \caption{Prueba de Aceptación - CP009}
    \label{tab:CP009}
  \end{center}
\end{table}


\begin{table}[H]
  \begin{center}
    \begin{tabularx}{0.75\textwidth}{ X }
      \toprule
      \textbf{Codigo:} CP010
      \makebox[3cm][r]{}
      \makebox[6cm][r]{\textbf{Historia de Usuario:} US04} \\

      \addlinespace
      \textbf{Tipo:} Prueba de Funcionalidad - Negativa. \\

      \addlinespace
      \textbf{Nombre:} Verificar que la ruta óptima se muestre dentro del campus Universitario. \\

      \addlinespace
      \textbf{Descripción:} Validar que la ruta óptima sea visible dentro del campus Universitario. \\

      \addlinespace
      \textbf{Condiciones de Ejecución:}
      El usuario debe estar fuera del campus Universitario. \\

      \addlinespace
      \textbf{Entradas / Pasos de Ejecución:}  \\
      \tab \textbf{1.} Seleccionar el menú \emph{Lugares}. \\
      \tab \textbf{2.} Buscar el lugar registrado.\\
      \tab \textbf{3.} Seleccionar la opción para buscar la ruta al lugar. \\


      \addlinespace
      \textbf{Resultado Esperado:} La ruta óptima debe ser visible solamente dentro del campus Universitario.  \\

      \addlinespace
      \textbf{Evaluación de la Prueba:} Prueba exitosa. \\

      \bottomrule
    \end{tabularx}
    \caption{Prueba de Aceptación - CP010}
    \label{tab:CP010}
  \end{center}
\end{table}



\begin{table}[H]
  \begin{center}
    \begin{tabularx}{0.75\textwidth}{ X }
      \toprule
      \textbf{Codigo:} CP011
      \makebox[3cm][r]{}
      \makebox[6cm][r]{\textbf{Historia de Usuario:} US03} \\

      \addlinespace
      \textbf{Tipo:} Prueba de Usabilidad. \\

      \addlinespace
      \textbf{Nombre:} Verificar que se puede cerrar el mapa. \\

      \addlinespace
      \textbf{Descripción:} Validar que la vista del mapa pueda ser cerrada y la vista de la información del lugar sea visible nuevamente. \\

      \addlinespace
      \textbf{Condiciones de Ejecución:}
      Un lugar debe estar registrado en el sistema. \\

      \addlinespace
      \textbf{Entradas / Pasos de Ejecución:}  \\
      \tab \textbf{1.} Seleccionar el menú \emph{Lugares}. \\
      \tab \textbf{2.} Buscar el lugar registrado.\\
      \tab \textbf{3.} Seleccionar la opción para buscar la ruta al lugar. \\
      \tab \textbf{4.} Presionar el botón cerrar disponible sobre el mapa. \\


      \addlinespace
      \textbf{Resultado Esperado:} La vista del mapa debe ser cerrada y la información del lugar ser visible. \\

      \addlinespace
      \textbf{Evaluación de la Prueba:} Prueba exitosa. \\

      \bottomrule
    \end{tabularx}
    \caption{Prueba de Aceptación - CP011}
    \label{tab:CP011}
  \end{center}
\end{table}
