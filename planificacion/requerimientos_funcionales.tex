% \begin{table}[H]
%   \begin{center}
    \begin{longtable}{ c  X }
      \toprule
        \textbf{Requerimiento} &
        \multicolumn{1}{c}{\textbf{Descripción}}\\

      \midrule
      \endhead

\addlinespace
\textbf{RF001}
&
Los usuarios pueden ver los lugares registrados en el sistema en una lista
\\

\addlinespace
\textbf{RF002}
&
Los usuarios pueden filtrar los lugares de acuerdo al nombre del lugar
\\

\addlinespace
\textbf{RF003}
&
Dentro de la información de un lugar se puede observar una foto del lugar, así como el teléfono, una descripción y en qué piso se encuentra.
\\

\addlinespace
\textbf{RF004}
&
El usuario puede ver la ruta óptima hacia un lugar en específico
\\

\addlinespace
\textbf{RF005}
&
Un usuario visitante puede empezar el proceso de registro al sistema
\\


\addlinespace
\textbf{RF006}
&
El sistema tendrá un módulo de autenticación y autorización de usuarios
\\

\addlinespace
\textbf{RF007}
&
Un usuario registrado puede editar la información de un lugar
\\

\addlinespace
\textbf{RF008}
&
Un usuario registrado puede agregar una foto al lugar para una mejor identificación de este.
\\


\addlinespace
\textbf{RF009}
&
Un usuario registrado puede eliminar un lugar de sistema
\\


\addlinespace
\textbf{RF010}
&
Un usuario Administrador puede ver una lista con los usuarios registrados
\\


\addlinespace
\textbf{RF011}
&
Un Administrador puede ver y aceptar la solicitud de registra de parte de un usuario
\\


\addlinespace
\textbf{RF012}
&
Un Administrador puede eliminar un usuario del sistema.
\\

\addlinespace
\textbf{RF013}
&
Un Administrador puede ver los reportes que el sistema ofrezca.
\\


      \bottomrule
      \caption{Requerimientos Funcionales}
      \label{tab:requerimientos funcionales}
    \end{longtable}
%   \end{center}
% \end{table}
