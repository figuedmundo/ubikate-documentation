\section{Iteración 1}
\label{sec:iteracion_1}

% Para la primera iteración se implementaron las historias de usuario con más relevancia dentro de la lógica de negocio del cliente, estas son generalmente las que tienen mayor impacto en el sistema a desarrollar. \\

%
% \subsection{Iteration Planning Meeting}
% \label{sub:Iteration Planning Meeting}

\subsection{Planificación de la Iteración 1}

En esta etapa se analizaran las Historias de Usuario seleccionadas para esta iteración, y se las dividirá en \emph{tareas de ingeniería}. \\

% \begin{itemize}
%   \item \textbf{Planificación de la Iteración 1:} En esta etapa se analizan las Historias de Usuario seleccionadas para esta iteración, y se las divide en \emph{tareas de ingeniería}.
% \end{itemize}
  %
  % \subsection{Exploración y Planeación}
  % \label{subs:Exploración y Planeación}


% En un equipo de desarrollo formado por varias personas, las fases de Exploración y Planeación se las realiza por separado, primeramente en la fase de \emph{exploración} los desarrolladores se apropian de alguna de las historias de usuario planeadas para la iteración y procede a dividir la historia de usuario en \emph{Tareas de Ingeniería}, posteriormente en la fase de la \emph{Planeación}, todos los desarrolladores estiman las tareas de acuerdo a criterio propio. \\
%
% Tomando en cuenta que el equipo de desarrollo está compuesto solo por mi persona, para la implementación del presente proyecto de grado, las fases de Exploración y Planeación se las realizó al mismo tiempo. \\
%
% Para la primera iteración se determinó que las historias de usuario a implementar serían la 1 y la 2.   \\
% %
% Posteriormente como tarea del desarrollador se procede a dividir las historias de usuario en Tareas de Ingeniería, en la tabla se determinaron las Tareas pertenecientes a la historia de usuario 2, dentro lo que es la planeación se debe repartir las tareas entre los desarrolladores, pero ya que el equipo de desarrollo se traduce a mi persona, todas las tareas recaen sobre mi responsabilidad, como parte de la planeación es necesario estimar las tareas,

% En esta etapa se analizan las Historias de Usuario seleccionadas para esta iteración, y se las divide en \emph{tareas de ingeniería}.

  % \subsubsection{Tareas del US01}
  % \label{sub:us01_tasks}

En primer lugar se analizará la \emph{historia de usuario} US01, tal como se puede ver en el cuadro \ref{tab:US01}.

  
\begin{table}[H]
  \begin{center}
    \begin{tabularx}{0.75\textwidth}{ X }
      \toprule
      \textbf{Historia de Usuario:} US01
      \makebox[6cm][r]{\textbf{Prioridad:} Alta \space} \\
      \makebox[4cm][r]{}
      \makebox[6cm][r]{\textbf{Riesgo:} Medio} \\

      \addlinespace
      \textbf{Nombre:} Implementar la lista de lugares.\\


      \addlinespace
      \textbf{Descripción:} \\
      \tab Yo como visitante\\
      \tab Deseo ver una lista de lugares \\
      \tab Para encontrar el lugar al que deseo ir\\

      \addlinespace
      \textbf{Criterios de Aceptación:} \\
      \tab Quiero tener los lugares en una base de datos \\
      \tab Quiero ver una lista de lugares\\
      \tab Quiero filtrar la lista de lugares por el nombre o parte de este\\

      \bottomrule
    \end{tabularx}
    \caption{Historia de Usuario - US01}
    \label{tab:US01}
  \end{center}
\end{table}


    \begin{table}[H]
  \begin{center}
    \begin{tabularx}{0.75\textwidth}{ X }
      \toprule
      \textbf{Número de Tarea:} T001
      \makebox[1cm][r]{}
      \makebox[6cm][r]{\textbf{Historia de Usuario:} US01} \\

      \addlinespace
      \textbf{Descripción:} Crear un archivo shapefile con información inicial de lugares principales dentro el campus de la UMSS. \\

      \addlinespace
      \textbf{Tipo de Tarea:} Desarrollo
      % \makebox[1cm][r]{}
      \makebox[6cm][r]{\textbf{Estimación [dias]:} 1} \\

      \addlinespace
      \textbf{Programador Responsable:} Edmundo Figueroa \\

      \bottomrule
    \end{tabularx}
    \caption{Tarea de Ingeniería - T001}
    \label{tab:T001}
  \end{center}
\end{table}


\begin{table}[H]
  \begin{center}
    \begin{tabularx}{0.75\textwidth}{ X }
      \toprule
      \textbf{Número de Tarea:} T002
      \makebox[1cm][r]{}
      \makebox[6cm][r]{\textbf{Historia de Usuario:} US01} \\

      \addlinespace
      \textbf{Descripción:} Crear una base de datos que pueda manejar información geoespacial. \\

      \addlinespace
      \textbf{Tipo de Tarea:} Desarrollo
      % \makebox[1cm][r]{}
      \makebox[6cm][r]{\textbf{Estimación [dias]:} 1} \\

      \addlinespace
      \textbf{Programador Responsable:} Edmundo Figueroa \\

      \bottomrule
    \end{tabularx}
    \caption{Tarea de Ingeniería - T002}
    \label{tab:T002}
  \end{center}
\end{table}

\begin{table}[H]
  \begin{center}
    \begin{tabularx}{0.75\textwidth}{ X }
      \toprule
      \textbf{Número de Tarea:} T003
      \makebox[1cm][r]{}
      \makebox[6cm][r]{\textbf{Historia de Usuario:} US01} \\

      \addlinespace
      \textbf{Descripción:} Popular la base de datos creada en T002 con la información de T001. \\

      \addlinespace
      \textbf{Tipo de Tarea:} Desarrollo
      \makebox[6cm][r]{\textbf{Estimación [dias]:} 0.5} \\

      \addlinespace
      \textbf{Programador Responsable:} Edmundo Figueroa \\

      \bottomrule
    \end{tabularx}
    \caption{Tarea de Ingeniería - T003}
    \label{tab:T003}
  \end{center}
\end{table}

\begin{table}[H]
  \begin{center}
    \begin{tabularx}{0.75\textwidth}{ X }
      \toprule
      \textbf{Número de Tarea:} T004
      \makebox[1cm][r]{}
      \makebox[6cm][r]{\textbf{Historia de Usuario:} US01} \\

      \addlinespace
      \textbf{Descripción:} Mostrar una lista de los lugares. \\

      \addlinespace
      \textbf{Tipo de Tarea:} Desarrollo
      \makebox[6cm][r]{\textbf{Estimación [dias]:} 2} \\

      \addlinespace
      \textbf{Programador Responsable:} Edmundo Figueroa \\

      \bottomrule
    \end{tabularx}
    \caption{Tarea de Ingeniería - T004}
    \label{tab:T004}
  \end{center}
\end{table}


\begin{table}[H]
  \begin{center}
    \begin{tabularx}{0.75\textwidth}{ X }
      \toprule
      \textbf{Número de Tarea:} T005
      \makebox[1cm][r]{}
      \makebox[6cm][r]{\textbf{Historia de Usuario:} US01} \\

      \addlinespace
      \textbf{Descripción:} Filtrar los lugares ingresando el nombre o parte de este. \\

      \addlinespace
      \textbf{Tipo de Tarea:} Desarrollo
      \makebox[6cm][r]{\textbf{Estimación [dias]:} 1} \\

      \addlinespace
      \textbf{Programador Responsable:} Edmundo Figueroa \\

      \bottomrule
    \end{tabularx}
    \caption{Tarea de Ingeniería - T005}
    \label{tab:T005}
  \end{center}
\end{table}


  % \subsubsection{Tareas del US02}
  % \label{sub:us02_tasks}

Posteriormente se analizará la la \emph{historia de usuario} US02, ver el cuadro \ref{tab:US02}.

  
\begin{table}[H]
 \begin{center}
   \begin{tabularx}{0.75\textwidth}{ X }
     \toprule
     \textbf{Historia de Usuario:} US02
     \makebox[6cm][r]{\textbf{Prioridad:} Baja} \\
     \makebox[4cm][r]{}
     \makebox[6cm][r]{\textbf{Riesgo:} Alto} \\

     \addlinespace
     \textbf{Nombre:} Implementar la vista de la información del lugar.\\
     
     \addlinespace
     \textbf{Descripción:} \\
     \tab Yo como visitante\\
     \tab Deseo ver la información de un lugar\\
     % & Deseo ingresar el nombre de un lugar\\
     \tab Para decidir si es el lugar que estoy buscando\\

     \addlinespace
     \textbf{Criterios de Aceptación:} \\
     \tab Quiero leer una descripción del lugar \\
     \tab Quiero ver un teléfono asociado al lugar\\
     \tab Quiero ver en qué piso se encuentra el lugar\\

     \bottomrule
   \end{tabularx}
   \caption{Historia de Usuario - US02}
   \label{tab:US02}
 \end{center}
\end{table}


    \begin{table}[H]
  \begin{center}
    \begin{tabularx}{0.75\textwidth}{ X }
      \toprule
      \textbf{Número de Tarea:} T006
      \makebox[1cm][r]{}
      \makebox[6cm][r]{\textbf{Historia de Usuario:} US02} \\

      \addlinespace
      \textbf{Descripción:} Mostrar la Descripcion del lugar. \\

      \addlinespace
      \textbf{Tipo de Tarea:} Desarrollo
      \makebox[6cm][r]{\textbf{Estimación [dias]:} 0.5} \\

      \addlinespace
      \textbf{Programador Responsable:} Edmundo Figueroa \\

      \bottomrule
    \end{tabularx}
    \caption{Tarea de Ingeniería - T006}
    \label{tab:T006}
  \end{center}
\end{table}


\begin{table}[H]
  \begin{center}
    \begin{tabularx}{0.75\textwidth}{ X }
      \toprule
      \textbf{Número de Tarea:} T007
      \makebox[1cm][r]{}
      \makebox[6cm][r]{\textbf{Historia de Usuario:} US02} \\

      \addlinespace
      \textbf{Descripción:} Mostrar el telefono del lugar. \\

      \addlinespace
      \textbf{Tipo de Tarea:} Desarrollo
      % \makebox[1cm][r]{}
      \makebox[6cm][r]{\textbf{Estimación [dias]:} 0.5} \\

      \addlinespace
      \textbf{Programador Responsable:} Edmundo Figueroa \\

      \bottomrule
    \end{tabularx}
    \caption{Tarea de Ingeniería - T007}
    \label{tab:T007}
  \end{center}
\end{table}

\begin{table}[H]
  \begin{center}
    \begin{tabularx}{0.75\textwidth}{ X }
      \toprule
      \textbf{Número de Tarea:} T008
      \makebox[1cm][r]{}
      \makebox[6cm][r]{\textbf{Historia de Usuario:} US02} \\

      \addlinespace
      \textbf{Descripción:} Mostrar el nivel o el piso del lugar. \\

      \addlinespace
      \textbf{Tipo de Tarea:} Desarrollo
      \makebox[6cm][r]{\textbf{Estimación [dias]:} 0.5} \\

      \addlinespace
      \textbf{Programador Responsable:} Edmundo Figueroa \\

      \bottomrule
    \end{tabularx}
    \caption{Tarea de Ingeniería - T008}
    \label{tab:T008}
  \end{center}
\end{table}

\begin{table}[H]
  \begin{center}
    \begin{tabularx}{0.75\textwidth}{ X }
      \toprule
      \textbf{Número de Tarea:} T009
      \makebox[1cm][r]{}
      \makebox[6cm][r]{\textbf{Historia de Usuario:} US02} \\

      \addlinespace
      \textbf{Descripción:} Mostrar una imagen o foto del lugar. \\

      \addlinespace
      \textbf{Tipo de Tarea:} Desarrollo
      \makebox[6cm][r]{\textbf{Estimación [dias]:} 2} \\

      \addlinespace
      \textbf{Programador Responsable:} Edmundo Figueroa \\

      \bottomrule
    \end{tabularx}
    \caption{Tarea de Ingeniería - T009}
    \label{tab:T009}
  \end{center}
\end{table}


    %
    % \begin{itemize}
    %   \item \textbf{Implementación de la Iteración 1:}
    % \end{itemize}

\subsection{Implementación de la Iteración 1}
% \label{sub:implementacion_iteracion_1}

    % \subsubsection{Recolección de la información de los lugares}
% % \label{subs:Los lugares}
%
% En primer lugar se recolectó la información de los lugares que la aplicación contendrá  de forma inicial, al igual que para recolectar las rutas se hizo uso de un \emph{GPS Garmin Nuvi 1300}, el cual cuenta con la opción de guardar locaciones como favoritos, entonces solo fue necesario estar cerca del lugar que se desea guardar y activar esa opción del GPS, este guarda la información en un archivo \emph{.gpx} y con la ayuda de \emph{QGIS} se genero el archivo shapefile correspondiente.\\
%
% Posteriormente es necesario pasar la información geoespacial del shapefile a la base de datos, para esta tarea se hizo uso de una herramienta disponible para postgres, \emph{shp2pgsql}, que permite la conversión de un archivo shapefile a un archivo sql.
%
% % $ shp2pgsql -s 4326 -I -S -c -d ~/Documents/places.shp > places.sql
% \begin{verbatim}
%   $ shp2pgsql -s 3785 -I -S -c -d ~/Documents/places.shp > places.sql
% \end{verbatim}
%
% Con el anterior comando se tiene como resultado un archivo \emph{.sql}, el cual es ingresado en la base de datos ya configurada, de esta forma nuestra base de datos para a contener una tabla geoespacial con datos de tipo \emph{POINT}, los cuales representan los lugares dentro del campus de la UMSS.\\
%
% % \begin{verbatim}
% %   $ shp2pgsql -s 4326 -I -S -c -d ~/Documents/ways.shp > ways.sql
% % \end{verbatim}
% %
% % De la misma forma es necesario pasar la información de las rutas contenidas en un archivo shapefile a un archivo sql, en este caso creará una tabla \emph{WAYS}.\\
%
% El archivo \emph{sql} resultante es usado para popular la base de datos con la información inicial de los lugares que contiene el campus universitario, para tal tarea se usó el siguiente comando.\\
% % Los archivos resultantes \emph{sql} son usados para popular la base de datos .\\
%
% \begin{verbatim}
%   $ psql -d db_ubikate -U db_admin -f /Documents/places.sql
% \end{verbatim}
%
% \begin{figure}[H]
%   \begin{center}
%     \includegraphics[width=1\textwidth]{iteration1/postgres_places}
%     \caption{Herramienta gráfica de PostgreSQL (\emph{pgAdmin}).}
%     \label{fig:postgres_places}
%     \caption*{Fuente: Elaboración propia}
%   \end{center}
% \end{figure}
%  % con la tabla de Lugares desplegada.
%
% En la figura \ref{fig:postgres_places} se puede observar que la columna \emph{Elevation} contiene datos que el GPS Garmin Nuvi 1300 genera al momento de guardar un punto, en el presente caso es irrelevante.\\
%
%
% Una vez que se tiene populada la base de datos con la información de los lugares es necesario implementar el cómo se comunicara el backend con el frontend, este como ya se explicó se implementará un Servicio Web basado en un API REST.\\

% En la primera Iteración se implementó la lista de lugares que la aplicación mostrará a los usuarios, para lo cual se describe a continuación los pasos que se siguieron. \\


\subsubsection{Acceder a la Información de los lugares mediante un REST API }


La información de un lugar está almacenada en una base de datos, entonces para poder acceder a esta informacion se implement\'o un REST API, que como ya se especificaron sus características, es la más apropiada para construir una aplicación web. \\

Las peticiones al API llegan a través del protocolo HTTP, en forma de URIs por lo que el servidor implementado con \emph{ExpressJS} tiene que escuchar estas peticiones y actuar en relación al tipo de petición HTTP está llegando, por lo cual es necesario ``mapear'' las URIs con acciones o métodos, como se puede ver en el codigo \ref{express_api}, los cuales contienen el código necesario para comunicarse con la base de datos. \\


\begin{center}
  \begin{lstlisting}[label=express_api,caption=Declarando API REST con ExpressJS]

        const router = express.Router();
        router.get('/', places.getAll);
        router.get('/:id', places.getPlace);
        router.post('/', places.newPlace);
        router.put('/:id', places.editPlace);
        router.delete('/:id', places.deletePlace);

        app.use('/api/v1/places', router);

  \end{lstlisting}
\end{center}


En realidad, \emph{ExpressJS} no tiene restricciones a la hora de mapear las URIs, pero para una mejor comprensión del API que se está desarrollando, es necesario seguir convenciones que aseguran que cualquier aplicación pueda consumir la información que el API pueda ofrecer, y un REST API cumple con estas características. \\

A cada URI mapeada se lo conoce como \emph{endpoint}, por lo que para obtener la lista de lugares que están registradas en el sistema, se usará el \emph{endpoint}: \verb|router.get('/', places.getAll);|, el cual es el indicado para obtener todos los lugares, según la implementación del REST API visto en la tabla \ref{tab:rest}. \\

El método asociado al \emph{endpoint} contendrá la lógica para comunicarse con la base de datos, por lo que es necesario implementarla. \\


\subsubsection{Implementación de la base de datos para almacenar los lugares}

La característica especial que tiene esta base de datos, es la de manejar datos geoespaciales, en el caso de los \emph{lugares} estos son representados por el tipo POINT.

La base de datos PostgreSQL por defecto no maneja datos geoespaciales, pero como ya se explicó, añadiendo la extensión PostGIS, ya es posible el manejo de datos geoespaciales, pero es nuestra tarea el comprobar que realmente esté manejando los tipos de datos geoespaciales.

Entonces se procedió a recolectar información de un conjunto de lugares, para esta tarea se utilizó un \emph{GPS Garmin Nuvi 1300}, el cual es un dispositivo de posicionamiento global, que cuenta con la opción de guardar locaciones, el dispositivo GPS almacena su información en un archivo \emph{gpx} (que básicamente es un fichero XML estándar usado para compartir datos entre GPS's) y con la ayuda de \emph{QGIS} se gener\'o el archivo shapefile que se utilizó para popular la base de datos con información geoespacial de algunos \emph{lugares} del campus Universitario.

Posteriormente se convierte la información geoespacial almacenada en el shapefile a la base de datos, para lo cual se esta hizo uso de una herramienta propia de PostgreSQL, que permite la conversión de un archivo shapefile a un archivo sql, el cual se puede ver en el siguiente código.

% % $ shp2pgsql -s 4326 -I -S -c -d ~/Documents/places.shp > places.sql
\begin{verbatim}
  $ shp2pgsql -s 3785 -I -S -c -d ~/Documents/places.shp > places.sql
\end{verbatim}

Con el anterior comando se genera un archivo \emph{sql}, el cual es usado para popular la base de datos ya preparada para contener datos geoespaciales, para ingresar la información a la base de datos se utilizó el siguiente comando, propio de PostgreSQL.

% \begin{verbatim}
%   $ shp2pgsql -s 3785 -I -S -c -d ~/Documents/ways.shp > ways.sql
% \end{verbatim}

% % De la misma forma es necesario pasar la información de las rutas contenidas en un archivo shapefile a un archivo sql, en este caso creará una tabla \emph{WAYS}.\\
%
% El archivo \emph{sql} resultante es usado para popular la base de datos con la información inicial de los lugares que contiene el campus universitario, para tal tarea se usó el siguiente comando.\\
% % Los archivos resultantes \emph{sql} son usados para popular la base de datos .\\
%
\begin{verbatim}
  $ psql -d db_ubikate -U db_admin -f /Documents/places.sql
\end{verbatim}

Al finalizar este proceso, se puede ver en la figura \ref{fig:postgres_places}, que la tabla correspondiente a los \emph{lugares} está correctamente populada con la información geoespacial de los lugares registrados con el dispositivo GPS.

\begin{figure}[H]
  \begin{center}
    \includegraphics[width=1\textwidth]{iteration1/postgres_places}
    \caption{Herramienta gráfica de PostgreSQL (\emph{pgAdmin}).}
    \label{fig:postgres_places}
    \caption*{Fuente: Elaboración propia}
  \end{center}
\end{figure}
%  % con la tabla de Lugares desplegada.
%
% En la figura \ref{fig:postgres_places} se puede observar que la columna \emph{Elevation} contiene datos que el GPS Garmin Nuvi 1300 genera al momento de guardar un punto, en el presente caso es irrelevante.\\


Una vez implementado el servicio web, necesitamos empezar con el desarrollo del frontend de la aplicación, que como ya se explicó se usará \emph{EmberJS}. \\


\subsubsection{Mostrar la lista de lugares}


\emph{EmberJS} consume la información del API implementado, por lo tanto se hará una llamada \emph{GET} al URI \emph{places/}, que dentro de la estructura de \emph{EmberJS} se tiene que implementar en el \emph{router} dedicado al URI correspondiente. El método mostrado en el codigo \ref{model_places_index}, es el encargado de hacer generar la petición GET que el API está listo para responder con la lista de los lugares registrados en el sistema. \\

\begin{center}
  \begin{lstlisting}[label=model_places_index,caption=Método para obtener la lista de lugares del API]

    model() {
        var url = (ENV.APP.API_HOST || '') + '/api/v1/places/';
        return jQuery.ajax({
          url: url,
          type: 'GET'
        });
      }

  \end{lstlisting}
\end{center}

Una vez que se obtiene la lista de lugares del servidor, es necesario desplegarlo en el navegador, para tal efecto se implement\'o el método mostrado en el codigo \ref{template_places_index}, y se utilizó el template correspondiente al URI \emph{templates/places/index.hbs}.

\begin{center}
  \begin{lstlisting}[label=template_places_index,caption=Template de la lista de lugares]

    {{#paper-list}}
      {{#each model.data as |place|}}
        {{#paper-item class="md-1-line" onClick=(transition-to 'places.show' place)}}
            <div class="md-list-item-text">
                <span>{{place.name}}</span>
            </div>
        {{/paper-item}}
        {{paper-divider}}
      {{/each}}
    {{/paper-list}}

  \end{lstlisting}
\end{center}

En la anterior implementación se hizo uso de \emph{ember-paper}, que como ya se explicó ayudará en el \emph{look and feel} de la aplicación, tal como se puede observar en la figura \ref{fig:places_index}. \\
 % tiene toda la apariencia  mejorada para su uso en dispositivos móviles.


\begin{figure}[H]
  \begin{center}
    \includegraphics[width=0.3\textwidth]{iteration1/places_index_2}
    \caption{Lista de Lugares}
    \label{fig:places_index}
    \caption*{Fuente: Elaboración propia}
  \end{center}
\end{figure}


\subsubsection{Búsqueda de los lugares}
\label{subs:busqueda de los lugares}

Uno de los criterios de aceptación para la implementación de la \emph{historia de usuario} 2, correspondiente a la búsqueda de los lugares, es que sea posible la búsqueda usando el nombre del lugar o parte del mismo. Al implementar esta \emph{historia de usuario} se a\~nadi\'o un \emph{endpoint} adicional a nuestro servicio web, ver el codigo \ref{endpoint_search_place}, que basicamente realiza una consulta SQL la cual obtiene de la base de datos los lugares que concuerden con el criterio de búsqueda. \\
 % a continuación se puede ver el URI implementado en la servicio web. \\

\begin{center}
  \begin{lstlisting}[label=endpoint_search_place,caption=Implementación de la búsqueda de lugares en el Servicio Web]

    router.get('places/search/:name', places.getPlacesByName);

  \end{lstlisting}
\end{center}


\subsubsection{Mostrar la información del lugar}
\label{subs:Mostrar información del lugar}

Para mostrar la información de un \emph{lugar}, es necesario realizar una consulta al URI \emph{places/:id} usando el verbo HTTP \emph{GET}, el cual obtiene la información del \emph{lugar} en formato JSON la cual es mostrarda en el navegador, para tal efecto se emplea el template correspondiente al URI,  \emph{templates/places/show}. \\

La implementación del template se puede ver en el codigo \ref{template_places_show}. \\

\begin{center}
  \begin{lstlisting}[label=template_places_show,caption=Template para mostrar la información de un lugar]

      {{#text.headline}}{{model.name}}{{/text.headline}}
      {{#card.content}}
          {{#paper-list}}
              {{model.description}}
              {{/paper-item}}
                  {{paper-icon "local_phone"}} {{model.phone}}
              {{/paper-item}}
              {{#paper-item class="md-2-line" }}
                  {{paper-icon "layers"}} Piso N# {{model.level}}
              {{/paper-item}}
          {{/paper-list}}
      {{/card.content}}

  \end{lstlisting}
\end{center}

El resultado en el navegador del template implementado se puede apreciar en la figura \ref{fig:place_show}. \\

\begin{figure}[H]
  \begin{center}
    \includegraphics[width=0.3\textwidth]{iteration1/place_show}
    \caption{Vista de la Información de un Lugar.}
    \label{fig:place_show}
    \caption*{Fuente: Elaboración propia}
  \end{center}
\end{figure}




% En la figura \ref{fig:places_index}, se puede observar un


\subsection{Pruebas de Aceptación de la Iteración 1}

% \begin{itemize}
%   \item \textbf{Pruebas de Aceptación de la Iteracion 1:}
% \end{itemize}


\begin{table}[H]
  \begin{center}
    \begin{tabularx}{0.75\textwidth}{ X }
      \toprule
      \textbf{Codigo:} CP001
      \makebox[3cm][r]{}
      \makebox[6cm][r]{\textbf{Historia de Usuario:} US01} \\

      \addlinespace
      \textbf{Nombre:} Verificar la lista de lugares. \\

      \addlinespace
      \textbf{Descripción:} Validar que un usuario visitante puede ver la lista de lugares cuando ingresa al menú \emph{lugares}. \\

      \addlinespace
      \textbf{Condiciones de Ejecución:} \\
      \tab \textbf{a.} El usuario no debe estar registrado. \\
      \tab \textbf{b.} Deben existir lugares registrados en el sistema.\\

      \addlinespace
      \textbf{Entradas / Pasos de Ejecución:}  \\
      \tab \textbf{1.} Hacer tap sobre el botón \emph{menú} en la esquina superior-izquierda. \\
      \tab \textbf{2.} Seleccionar el menú \emph{lugares}.\\

      \addlinespace
      \textbf{Resultado Esperado:} El Usuario debe ver una lista con los lugares registrados en la Condición de Ejecución \emph{b}.  \\

      \addlinespace
      \textbf{Evaluación de la Prueba:} Prueba exitosa. \\

      \bottomrule
    \end{tabularx}
    \caption{Prueba de Aceptación - CP001}
    \label{tab:CP001}
  \end{center}
\end{table}

\begin{table}[H]
  \begin{center}
    \begin{tabularx}{0.75\textwidth}{ X }
      \toprule
      \textbf{Codigo:} CP002
      \makebox[3cm][r]{}
      \makebox[6cm][r]{\textbf{Historia de Usuario:} US01} \\

      \addlinespace
      \textbf{Nombre:} Verificar la busqueda de lugares. \\

      \addlinespace
      \textbf{Descripción:} Validar que un usuario puede filtrar un lugar de la lista mediante el nombre. \\

      \addlinespace
      \textbf{Condiciones de Ejecución:} Ingresar en la base de datos el lugar con nombre ``MEMI''. \\

      \addlinespace
      \textbf{Entradas / Pasos de Ejecución:}  \\
      \tab \textbf{1.} Hacer tap sobre el botón \emph{menú} en la esquina superior-izquierda. \\
      \tab \textbf{2.} Seleccionar el menú \emph{lugares}.\\
      \tab \textbf{3.} Ingresar el nombre ``MEMI'' en el cajón de búsqueda.\\


      \addlinespace
      \textbf{Resultado Esperado:} Se debe mostrar un solo item en la lista de lugares con el nombre ``MEMI'' desplegado.\\

      \addlinespace
      \textbf{Evaluación de la Prueba:} Prueba exitosa. \\

      \bottomrule
    \end{tabularx}
    \caption{Prueba de Aceptación - CP002}
    \label{tab:CP002}
  \end{center}
\end{table}

\begin{table}[H]
  \begin{center}
    \begin{tabularx}{0.75\textwidth}{ X }
      \toprule
      \textbf{Codigo:} CP003
      \makebox[3cm][r]{}
      \makebox[6cm][r]{\textbf{Historia de Usuario:} US02} \\

      \addlinespace
      \textbf{Nombre:} Verificar la información de un lugar. \\

      \addlinespace
      \textbf{Descripción:} Validar que un usuario puede ver la descripción, el teléfono, el nivel y la foto de un lugar. \\

      \addlinespace
      \textbf{Condiciones de Ejecución:} Ingresar en la base de datos el lugar con nombre ``MEMI'' con su información completa. \\

      \addlinespace
      \textbf{Entradas / Pasos de Ejecución:}  \\
      \tab \textbf{1.} Hacer tap sobre el botón \emph{menú} en la esquina superior-izquierda. \\
      \tab \textbf{2.} Seleccionar el menú \emph{lugares}.\\
      \tab \textbf{3.} En la lista de lugares seleccionar el item ``MEMI''.\\

      \addlinespace
      \textbf{Resultado Esperado:} Se debe mostrar una pantalla con la foto del lugar ``MEMI'', su descripción, el teléfono y el nivel.\\

      \addlinespace
      \textbf{Evaluación de la Prueba:} Prueba exitosa. \\

      \bottomrule
    \end{tabularx}
    \caption{Prueba de Aceptación - CP003}
    \label{tab:CP003}
  \end{center}
\end{table}

