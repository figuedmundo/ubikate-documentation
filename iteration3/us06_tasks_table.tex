\begin{table}[H]
  \begin{center}
    \begin{tabularx}{\textwidth}{ c  X  C{2.3cm} }
      \toprule
        \textbf{Código} &
        \multicolumn{1}{c}{\textbf{Tarea}} &
        \textbf{Estimación [dias]}\\

      \midrule
        T020
        &
        Implementar en el backend el \emph{endpoint} para añadir y editar usuarios.
        &
        0.5 \\

        \addlinespace
        T020
        &
        Mostrar un botón solamente a usuarios registrados para añadir lugares.   %El usuario registrado podrá ver un link hacia el formulario para añadir más lugares desde la lista de lugares existentes.
        &
        0.5 \\


      \addlinespace
        T021
        &
        Mostrar un formulario para añadir más lugares, con el nombre del lugar, descripción, teléfono y nivel. %El usuario podrá ver un formulario para añadir un lugar con información básica. por ejemplo, el nombre del lugar, descripción, teléfono.
        &
        1 \\

      \addlinespace
        T022
        &
        Registrar las coordenadas del lugar al momento de crear un nuevo lugar % El usuario deberá estar cerca del lugar que desea añadir para poder georeferenciarlo.
        &
        1 \\

      \addlinespace
        T023
        &
        Mostrar un botón para poder añadir una foto al lugar.
        &
        1 \\


      \addlinespace
        P004
        &
        Crear pruebas de funcionalidad del US05.
        &
        0.5 \\

      \addlinespace
      \midrule
        & \multicolumn{1}{R{7cm}}{\textbf{Total: }}
        & 4 \\

      \bottomrule
    \end{tabularx}
    \caption{Tareas del US05}
    \label{tab:us05_tasks}
  \end{center}
\end{table}
