\subsubsection{Resultado de las pruebas de la Iteración 3}

Al finalizar la Iteración 3, se ejecutaron todas las pruebas escritas durante la presente y las anteriores iteraciones, en el cuadro \ref{tab:regresion_3} se puede ver el detalle.

\LTXtable{0.8\textwidth}{iteration3/regresion_table}

%
% \begin{table}[H]
%   \begin{center}
%     \begin{tabularx}{0.8\textwidth}{ c  X  c }
%       \toprule
%         \textbf{Código} &
%         \multicolumn{1}{c}{\textbf{Título de la Prueba}} &
%         \textbf{Resultado}\\
%
% \midrule
% CP001
% &
% Verificar la lista de lugares.
% &
% Exitoso \\
%
% % \addlinespace
% CP002
% &
% Verificar la busqueda de lugares.
% &
% Exitoso \\
%
% % \addlinespace
% CP003
% &
% Verificar la información de un lugar.
% &
% Exitoso \\
%
% % \addlinespace
% CP004
% &
% Verificar la lista de lugares cuando se busca un lugar no registrado.
% &
% Exitoso \\
%
% % \addlinespace
% CP005
% &
% Verificar la información de un lugar mediante el URI.
% &
% Exitoso \\
%
% % \addlinespace
% CP006
% &
% Verificar que el \emph{Menú} sea desplegado dinámicamente.
% &
% Exitoso \\
%
% % \addlinespace
% CP007
% &
% Verificar el marcador del lugar sobre el mapa del campus Universitario.
% &
% Exitoso \\
%
% % \addlinespace
% CP008
% &
% Verificar que el marcador del lugar muestre la información.
% &
% Exitoso \\
%
%
% % \addlinespace
% CP009
% &
% Verificar que una línea roja muestra la ruta óptima hacia el lugar.
% &
% Exitoso \\
%
% % \addlinespace
% CP010
% &
% Verificar que la ruta óptima se muestre dentro del campus Universitario.
% &
% Exitoso \\
%
%
% % \addlinespace
% CP011
% &
% Verificar que se puede cerrar el mapa.
% &
% Exitoso \\
%
% CP012
% &
% Verificar el registro de lugares.
% &
% Exitoso \\
%
% CP013
% &
% Verificar la edición de la información de un lugar.
% &
% Exitoso \\
%
%
% CP014
% &
% Verificar que no pueda registrar un lugar fuera del campus Universitario.
% &
% Fallada \\
%
% CP015
% &
% Verificar los campos de texto de los formularios de registro y edición.
% &
% Exitoso \\
%
%
%       \bottomrule
%     \end{tabularx}
%     \caption{Pruebas de regresión de la Iteración 3}
%     \label{tab:regresion_3}
%   \end{center}
% \end{table}


\begin{itemize}
  \item Se ejecutaron 8 pruebas de funcionalidad positiva, todas pasaron exitosamente.
  \item Se ejecutaron 4 pruebas de funcionalidad negativa, falló la prueba CP014 pero que al ser una prueba negativa se lo documentara como un \emph{problema conocido}.
  \item Se ejecutaron 3 prueba de usabilidad, todas pasaron exitosamente.
\end{itemize}
