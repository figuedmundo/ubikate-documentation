\begin{table}[H]
  \begin{center}
    \begin{tabularx}{0.75\textwidth}{ X }
      \toprule
      \textbf{Número de Tarea:} T019
      \makebox[1cm][r]{}
      \makebox[6cm][r]{\textbf{Historia de Usuario:} US06} \\

      \addlinespace
      \textbf{Descripción:} Implementar en el backend el \emph{endpoint} para añadir y editar usuarios. \\

      \addlinespace
      \textbf{Tipo de Tarea:} Desarrollo
      \makebox[6cm][r]{\textbf{Estimación [dias]:} 1} \\

      \addlinespace
      \textbf{Programador Responsable:} Edmundo Figueroa \\

      \bottomrule
    \end{tabularx}
    \caption{Tarea de Ingeniería - T019}
    \label{tab:T019}
  \end{center}
\end{table}


\begin{table}[H]
  \begin{center}
    \begin{tabularx}{0.75\textwidth}{ X }
      \toprule
      \textbf{Número de Tarea:} T020
      \makebox[1cm][r]{}
      \makebox[6cm][r]{\textbf{Historia de Usuario:} US06} \\

      \addlinespace
      \textbf{Descripción:} Mostrar un botón para añadir lugares solamente a usuarios registrados. \\

      \addlinespace
      \textbf{Tipo de Tarea:} Desarrollo
      % \makebox[1cm][r]{}
      \makebox[6cm][r]{\textbf{Estimación [dias]:} 0.5} \\

      \addlinespace
      \textbf{Programador Responsable:} Edmundo Figueroa \\

      \bottomrule
    \end{tabularx}
    \caption{Tarea de Ingeniería - T020}
    \label{tab:T020}
  \end{center}
\end{table}

\begin{table}[H]
  \begin{center}
    \begin{tabularx}{0.75\textwidth}{ X }
      \toprule
      \textbf{Número de Tarea:} T021
      \makebox[1cm][r]{}
      \makebox[6cm][r]{\textbf{Historia de Usuario:} US06} \\

      \addlinespace
      \textbf{Descripción:} Mostrar un formulario para añadir más lugares, con el nombre del lugar, descripción, teléfono y nivel. \\

      \addlinespace
      \textbf{Tipo de Tarea:} Desarrollo
      \makebox[6cm][r]{\textbf{Estimación [dias]:} 0.5} \\

      \addlinespace
      \textbf{Programador Responsable:} Edmundo Figueroa \\

      \bottomrule
    \end{tabularx}
    \caption{Tarea de Ingeniería - T021}
    \label{tab:T021}
  \end{center}
\end{table}

\begin{table}[H]
  \begin{center}
    \begin{tabularx}{0.75\textwidth}{ X }
      \toprule
      \textbf{Número de Tarea:} T022
      \makebox[1cm][r]{}
      \makebox[6cm][r]{\textbf{Historia de Usuario:} US06} \\

      \addlinespace
      \textbf{Descripción:} Registrar las coordenadas del lugar al momento de crear un nuevo lugar. \\

      \addlinespace
      \textbf{Tipo de Tarea:} Desarrollo
      \makebox[6cm][r]{\textbf{Estimación [dias]:} 1} \\

      \addlinespace
      \textbf{Programador Responsable:} Edmundo Figueroa \\

      \bottomrule
    \end{tabularx}
    \caption{Tarea de Ingeniería - T022}
    \label{tab:T022}
  \end{center}
\end{table}


\begin{table}[H]
  \begin{center}
    \begin{tabularx}{0.75\textwidth}{ X }
      \toprule
      \textbf{Número de Tarea:} T023
      \makebox[1cm][r]{}
      \makebox[6cm][r]{\textbf{Historia de Usuario:} US06} \\

      \addlinespace
      \textbf{Descripción:} Implementar el modulo para poder agregar una foto a un lugar. \\

      \addlinespace
      \textbf{Tipo de Tarea:} Desarrollo
      \makebox[6cm][r]{\textbf{Estimación [dias]:} 1} \\

      \addlinespace
      \textbf{Programador Responsable:} Edmundo Figueroa \\

      \bottomrule
    \end{tabularx}
    \caption{Tarea de Ingeniería - T023}
    \label{tab:T023}
  \end{center}
\end{table}
