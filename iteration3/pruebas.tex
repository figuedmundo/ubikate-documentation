\begin{table}[H]
  \begin{center}
    \begin{tabularx}{0.75\textwidth}{ X }
      \toprule
      \textbf{Codigo:} CP006
      \makebox[3cm][r]{}
      \makebox[6cm][r]{\textbf{Historia de Usuario:} US06} \\

      \addlinespace
      \textbf{Nombre:} Verificar el registro de lugares. \\

      \addlinespace
      \textbf{Descripción:} Validar que un usuario registrado puede añadir un lugar nuevo al sistema. \\

      \addlinespace
      \textbf{Condiciones de Ejecución:} \\
      \tab \textbf{a.} El usuario debe estar registrado. \\
      % \tab \textbf{b.} Registrar un lugar dentro del campus Universitario.\\

      \addlinespace
      \textbf{Entradas / Pasos de Ejecución:}  \\
      \tab \textbf{1.} Seleccionar el menú \emph{Lugares}. \\
      \tab \textbf{2.} Hacer tap sobre el icono de registro de lugares.\\
      \tab \textbf{3.} Ingresar el Nombre del lugar.\\
      \tab \textbf{4.} Ingresar una Descripción del lugar.\\
      \tab \textbf{5.} Ingresar el Teléfono del lugar.\\
      \tab \textbf{6.} Ingresar el Nivel del lugar.\\
      \tab \textbf{7.} Aceptar el formulario.\\


      \addlinespace
      \textbf{Resultado Esperado:} Una vista de la información del lugar es desplegada con los datos ingresados en los pasos.  \\

      \addlinespace
      \textbf{Evaluación de la Prueba:} Prueba exitosa. \\

      \bottomrule
    \end{tabularx}
    \caption{Prueba de Aceptación - CP006}
    \label{tab:CP006}
  \end{center}
\end{table}

\begin{table}[H]
  \begin{center}
    \begin{tabularx}{0.75\textwidth}{ X }
      \toprule
      \textbf{Codigo:} CP007
      \makebox[3cm][r]{}
      \makebox[6cm][r]{\textbf{Historia de Usuario:} US07} \\

      \addlinespace
      \textbf{Nombre:} Verificar la edición de la información de un lugar. \\

      \addlinespace
      \textbf{Descripción:} Validar que un usuario registrado puede editar la información de un lugar. \\

      \addlinespace
      \textbf{Condiciones de Ejecución:} \\
      \tab \textbf{a.} El usuario debe estar registrado. \\
      \tab \textbf{b.} Registrar un lugar dentro del campus Universitario.\\

      \addlinespace
      \textbf{Entradas / Pasos de Ejecución:}  \\
      \tab \textbf{1.} Seleccionar el menú \emph{Lugares}. \\
      \tab \textbf{2.} Buscar el lugar registrado en las condiciones.\\
      \tab \textbf{3.} Editar el Nombre del lugar.\\
      \tab \textbf{4.} Editar la Descripción del lugar.\\
      \tab \textbf{5.} Editar el Teléfono del lugar.\\
      \tab \textbf{6.} Editar el Nivel del lugar.\\
      \tab \textbf{7.} Aceptar el formulario.\\


      \addlinespace
      \textbf{Resultado Esperado:} Una vista de la información del lugar es desplegada con la información editada en los pasos.  \\

      \addlinespace
      \textbf{Evaluación de la Prueba:} Prueba exitosa. \\

      \bottomrule
    \end{tabularx}
    \caption{Prueba de Aceptación - CP007}
    \label{tab:CP007}
  \end{center}
\end{table}
