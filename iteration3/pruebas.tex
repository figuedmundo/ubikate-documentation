\begin{table}[H]
  \begin{center}
    \begin{tabularx}{0.75\textwidth}{ X }
      \toprule
      \textbf{Código:} CP012
      \makebox[3cm][r]{}
      \makebox[6cm][r]{\textbf{Historia de Usuario:} US06} \\

      \addlinespace
      \textbf{Tipo:} Prueba de Funcionalidad - Positiva. \\

      \addlinespace
      \textbf{Nombre:} Verificar el registro de lugares. \\

      \addlinespace
      \textbf{Descripción:} Validar que un usuario registrado puede añadir un lugar nuevo al sistema. \\

      \addlinespace
      \textbf{Condiciones de Ejecución:} \\
      \tab \textbf{a.} El usuario debe estar registrado. \\
      % \tab \textbf{b.} Registrar un lugar dentro del campus Universitario.\\

      \addlinespace
      \textbf{Entradas / Pasos de Ejecución:}  \\
      \tab \textbf{1.} Seleccionar el menú \emph{Lugares}. \\
      \tab \textbf{2.} Hacer tap sobre el icono de registro de lugares.\\
      \tab \textbf{3.} Ingresar el Nombre del lugar.\\
      \tab \textbf{4.} Ingresar una Descripción del lugar.\\
      \tab \textbf{5.} Ingresar el Teléfono del lugar.\\
      \tab \textbf{6.} Ingresar el Nivel del lugar.\\
      \tab \textbf{7.} Aceptar el formulario.\\


      \addlinespace
      \textbf{Resultado Esperado:} Una vista de la información del lugar es desplegada con los datos ingresados en los pasos.  \\

      \addlinespace
      \textbf{Evaluación de la Prueba:} Prueba exitosa. \\

      \bottomrule
    \end{tabularx}
    \caption{Prueba de Aceptación - CP012}
    \label{tab:CP012}
  \end{center}
\end{table}

\begin{table}[H]
  \begin{center}
    \begin{tabularx}{0.75\textwidth}{ X }
      \toprule
      \textbf{Código:} CP013
      \makebox[3cm][r]{}
      \makebox[6cm][r]{\textbf{Historia de Usuario:} US07} \\

      \addlinespace
      \textbf{Tipo:} Prueba de Funcionalidad - Positiva. \\

      \addlinespace
      \textbf{Nombre:} Verificar la edición de la información de un lugar. \\

      \addlinespace
      \textbf{Descripción:} Validar que un usuario registrado puede editar la información de un lugar. \\

      \addlinespace
      \textbf{Condiciones de Ejecución:} \\
      \tab \textbf{a.} El usuario debe estar registrado. \\
      \tab \textbf{b.} Registrar un lugar dentro del campus Universitario.\\

      \addlinespace
      \textbf{Entradas / Pasos de Ejecución:}  \\
      \tab \textbf{1.} Seleccionar el menú \emph{Lugares}. \\
      \tab \textbf{2.} Buscar el lugar registrado en las condiciones.\\
      \tab \textbf{3.} Editar el Nombre del lugar.\\
      \tab \textbf{4.} Editar la Descripción del lugar.\\
      \tab \textbf{5.} Editar el Teléfono del lugar.\\
      \tab \textbf{6.} Editar el Nivel del lugar.\\
      \tab \textbf{7.} Aceptar el formulario.\\


      \addlinespace
      \textbf{Resultado Esperado:} Una vista de la información del lugar es desplegada con la información editada en los pasos.  \\

      \addlinespace
      \textbf{Evaluación de la Prueba:} Prueba exitosa. \\

      \bottomrule
    \end{tabularx}
    \caption{Prueba de Aceptación - CP013}
    \label{tab:CP013}
  \end{center}
\end{table}


\begin{table}[H]
  \begin{center}
    \begin{tabularx}{0.75\textwidth}{ X }
      \toprule
      \textbf{Código:} CP014
      \makebox[3cm][r]{}
      \makebox[6cm][r]{\textbf{Historia de Usuario:} US06} \\

      \addlinespace
      \textbf{Tipo:} Prueba de Funcionalidad - Negativa. \\

      \addlinespace
      \textbf{Nombre:} Verificar que no pueda registrar un lugar fuera del campus Universitario. \\

      \addlinespace
      \textbf{Descripción:} Validar que los lugares que están fuera del campus Universitario no pueda ser registrado en el sistema. \\

      \addlinespace
      \textbf{Condiciones de Ejecución:} \\
      \tab \textbf{a.} El usuario debe estar registrado. \\
      \tab \textbf{b.} El usuario debe estar fuera de los predios del Campus Universitario.\\

      \addlinespace
      \textbf{Entradas / Pasos de Ejecución:}  \\
      \tab \textbf{1.} Seleccionar el menú \emph{Lugares}. \\
      \tab \textbf{2.} Hacer tap sobre el icono de registro de lugares.\\

      \addlinespace
      \textbf{Resultado Esperado:} Un mensaje debería ser desplegado informando que no se pueden registrar Lugares fuera del Campus Universitario.  \\

      \addlinespace
      \textbf{Evaluación de la Prueba:} Prueba fallada. \\

      \bottomrule
    \end{tabularx}
    \caption{Prueba de Aceptación - CP014}
    \label{tab:CP014}
  \end{center}
\end{table}


\begin{table}[H]
  \begin{center}
    \begin{tabularx}{0.75\textwidth}{ X }
      \toprule
      \textbf{Código:} CP015
      \makebox[3cm][r]{}
      \makebox[6cm][r]{\textbf{Historia de Usuario:} US06, US07} \\

      \addlinespace
      \textbf{Tipo:} Prueba de Usabilidad. \\

      \addlinespace
      \textbf{Nombre:} Verificar los campos de texto de los formularios de registro y edición. \\

      \addlinespace
      \textbf{Descripción:} Validar que los campos de texto dentro los formularios de registro y edición de lugares no exceda la longitud de la pantalla del dispositivo móvil. \\

      \addlinespace
      \textbf{Condiciones de Ejecución:} \\
      \tab \textbf{a.} El usuario debe estar registrado. \\
      % \tab \textbf{b.} El usuario debe estar fuera de los predios del Campus Universitario.\\

      \addlinespace
      \textbf{Entradas / Pasos de Ejecución:}  \\
      \tab \textbf{1.} Seleccionar el menú \emph{Lugares}. \\
      \tab \textbf{2.} Presionar el icono de registro de lugares.\\
      \tab \textbf{3.} Insertar texto en los campos de descripción, teléfono y nombre.\\
      \tab \textbf{4.} Repetir para el formulario de edición de lugares.\\

      \addlinespace
      \textbf{Resultado Esperado:} El texto escrito no debería sobrepasar la longitud del dispositivo móvil en los formularios de registro y edición de lugares.  \\

      \addlinespace
      \textbf{Evaluación de la Prueba:} Prueba exitosa. \\

      \bottomrule
    \end{tabularx}
    \caption{Prueba de Aceptación - CP015}
    \label{tab:CP015}
  \end{center}
\end{table}
