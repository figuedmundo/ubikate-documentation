
\chapter{Conclusiones y Recomendaciones}

\section{Sobre la geolocalizacion}


    Es importante entender las diferencias entre los distintos tipos de sistemas de coordenadas porque computacionalmente realizar operaciones sobre los sistemas de coordenadas tiene costo. \\

          Si se usara el sistema de coordenadas geográfico (WSG84) este es el más apropiado si se necesitaría usar grandes extensiones de la superficie terrestre, que al ser una estructura elipsoidal el costo computacional para realizar las operaciones matemáticas de calcular distancias, intersecciones, etc. es más elevado. En cambio el uso de un sistema de coordenadas proyectado (Mercator Projection) tiene un costo computacional más bajo, ya que se estaría trabajando con un sistema geométrico.\\

          % Por otro lado,
          También hay tomar en cuenta la base de datos, ya que será esta la que se encargará de manejar los datos espaciales. Al estar usando PostGIS, se puede ver que en su documentacion\footnote{ http://postgis.org/documentation/manual-1.5/ch04.html} que claramente exorta el uso de un sistema geometrico sobre el uso de un sistema geografico si  se va trabajar con datos que cubran una pequena area geografica. Tomando en cuenta esta recomendación y el tamaño del área de estudio (el campus de la UMSS), se procedió a implementar en la base de datos el uso de la proyección Mercator. Se va usar Mercator sobre las otras proyecciones porque aparte de las ventajas que se mencionaron con anterioridad, Google Maps usa esta proyección y ya que se usará este mapa lo más correcto es trabajar con la misma proyección.
