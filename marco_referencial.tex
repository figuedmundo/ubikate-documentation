\chapter{Marco Referencial} % (fold)
\label{cha:marco_referncial}

  \section{Node JS}
  \label{sec:node_js}
    Node.js aparecio en 2009 y esta construido sobre el Motor de JavaScript de Google ``V8'' que fue sacado del browser y aplicado en el servidor.

    Para desarrollar en el lado del browser (cliente) el programador solo tiene disponible JavaScript como lenguaje de desarrollo pero en el lado del servidor existen muchas alternativas (Ruby, C\#, Phtyon, Java, etc.), JavaScript no estaba disponible.\\

    Node se beneficia del Motor de JavaScript ``V8'' ya que \'este es r\'apido y tiene integrado un sistema para manejar las instrucciones de forma asyncr\'onica, pero el mayor beneficio y el porqu\'e Node adquiri\'o una gran popularidad es la facilidad de compartir c\'odigo entre el cliente (browser) y el servidor.\\

    Node.js provee caracter\'isticas pero estas pueden parecer complicadas o que necesitan mas instrucciones de las necesarias para llevar a cabo acciones que ya son comunes en la creacion de aplicacion en lado del servidor, por ejemplo a la hora de crear un servidor web, Node se popularizo en gran medida por poder crear servidores web personalizables pero como ya dijimos esto tiene su grado de complejidad, aca es donde entra en accion Express.js.

  % end section node js
  \section{Express JS}
  \label{sec:express_js}
    Express.js es un framework que esta construido sobre la funcionalidad de servidor web de Node.js, Express.js ayuda a simplificar el API de Node y a\~nadir nuevas caracter\'isticas, dise\~nadas para mejorar y facilitar la organizaci\'on de una aplicaci\'on \emph{Express}.\\

    El Cliente (navegador web, aplicacion movil, etc) envia una peticion web y el servidor web de Node.js maneja los protocolos web, leyendolos y enviandolos a una aplicacion \emph{Express} que se encarga de a\~nadir caracteristicas a la peticion y espera la respuesta del ``Middleware Stack'', la funcion responde a la llamada y el servidor HTTP de Node envia la respuesta mediante los protocolos web al Cliente.\\

    Para escribir un servidor web con Express no es necesario una gran funcion para manejar un request, Express contiene utilidades que permite escribir funciones mas peque\~nas para facilitar el manejo de las peticiones web, asiendo uso de ``middleware'' y ``routing''.

    \subsection{Middleware}
    \label{sub:middleware}
      Node.js maneja una funci\'on para trabajar con una peticion web, encambio \emph{Express} maneja la llamada con varias funciones, cada funcion se encarga de una peque\~na parte del trabajo. Estas peque\~nas funciones que manejan la peticion web se denomina \emph{Middleware functions} o Middleware.

    %  end sub section middleware

    \subsection{Routing}
    \label{sub:routing}
      Muy parecido al Middleware, el Routing se encarga de partir una funcion de peticion web monolitica en peque\~nas piezas, pero a diferencia del Middleware, estos menajadores peticiones se ejecutan condicionalmente dependidiendo del URL y el metodo HTTP (GET, POST, DELETE) que el cliente envia.\\

    %  end sub section routing

    Express.js es bastante extensible y cuenta con gran popularidad en la comunidad de desarrollo, la cual provee herramientas para renderizar dinamicamente HTML o interfaces para comunicarse con Bases de Datos, por ejemplo para manejar la coneccion y llamadas a una base de datos en PostgreSQL se uso la libreria \emph{pg-promise}.

    \begin{lstlisting}[language=Java]
      database.any("SELECT * FROM users WHERE id = $1", [userId])
        .then(function (data) {
            response.send(data.name);
        });
    \end{lstlisting}


  % section Express JS (end)

  % \section{Introducci\'on} % (fold)
  % \label{sec:Introduccion}

  % % section Introduccion (end)

  % \section{Ruby on Rails} % (fold)
  % \label{sec:ruby_on_rails}
  %   Ruby on Rails es un framework dise\~nado para desarrollar aplicaciones web,
  %   y está construido sobre el lenguaje de programación Ruby, Ruby  fue creado
  %   alrededor de 1993 por Yukihiro ``Matz'' Matsumuto de Jap\'on, y liberado al
  %   público en 1995, y desde entonces fue ganando  popularidad y
  %   reputaci\'on gracias al aporte de una gran variedad de programadores, que
  %   realzan la sintaxis elegante y el código limpio que se genera, Ruby es un
  %   lenguaje de programación multiparadigma ya que implementa programaci\'on
  %   Orientado a Objetos, programaci\'on Funcional así como también
  %   programaci\'on Imperativa. \\
  %
  %   Ruby on Rails fue creado en el 2004 por David Heinemeier Hansson durante el desarrollo de Basecamp, una aplicación de gesti\'on de proyectos, y una vez que se necesitó para otros proyectos, el equipo de desarrollo extrajo el core de funcionalidad el cual fue presentado al público en julio del 2004 con el nombre de Ruby on Rails, como proyecto Open Source bajo una licencia MIT, desde entonces tuvo un gran  crecimiento impulsado por la comunidad de usuarios que continuamente están desarrollando nuevas características, limpiando bugs y creando gemas\footnote{ Los plugins o complementos, en el lenguaje Ruby son llamados \textbf{gemas}}. La última versión de Rails es la 3.2 publicado en enero del 2012 y actualmente está en su revisi\'on 3.2.8 presentado en agosto del 2012,  demostrando que el equipo de desarrollo de Rails esta trabajando constantemente en mejorar este framework que actualmente está entre los mejores en desarrollo web.\\
  %    % y se prevee que la versión 4 de Rails sea lanzada a finales del 2012, pero no  \\
  %   % y para el proyecto se utilizó la versión 3.2.3
  %
  %   El núcleo de funcionalidad de Rails es un conjunto de funciones llamadas \emph{Railties}:
  %   \begin{description}
  %     \item[Active Record] Es una implementaci\'on del patrón
  %       Object-Relational Mapping(ORM), que mapea las tablas de la Base de datos relacional en clases, filas en objetos y
  %       columnas en atributos de los objetos.
  %     \item[Active Support] Es el componente de Rails responsable de proporcionar extensiones del lenguaje Ruby, utilitarios y funciones primordiales a la hora de realizar cualquier tarea en el desarrollo de la aplicaci\'on.
  %     \item[Action Mailer] Permite enviar correo electrónico (email) desde la aplicaci\'on usando un  modelo y vistas.
  %     \item[Action Pack] Es el responsable de manejar y responder los request del navegador web. Provee las herramientas para el \textbf{routing}, define los \textbf{controladores}, y genera las respuestas renderizando las \textbf{vistas}. En resumen, Action Pack maneja las capas de la vista y el controlador del paradigma MVC.
  %   \end{description}
  %   Otra de las características de Rails es la facilidad para escribir Pruebas, en realidad Rails sugiere el modelo de Desarrollo guiado por Pruebas(TDD\footnote{Test-Driven Development, por sus siglas en Inglés}), que consiste en 3 pasos
  %   \begin{enumerate}
  %     \item \textbf{Rojo}, la prueba falla
  %     \item \textbf{Verde}, la prueba pasa
  %     \item \textbf{Refactorizar}, limpiar el código
  %   \end{enumerate}
  %   Para este proceso, Rails ofrece primeramente el módulo Test::Unit, pero también se pueden encontrar variadas herramientas para llevar a cabo está tarea.\\
  %
  %   Ruby on Rails es un framework MVC, que implementa los principios REST, No te Repitas\footnote{DRY, Don't Repit Yourself}, Convenci\'on sobre Configuraci\'on, estas características de Rails están explicadas con mayor detalle en el capítulo \ref{cha:ruby_on_rails_y_patrones_web_2_0}.


  % section ruby_on_rails (end)

  \section{Base de Datos} % (fold)
  \label{sec:base_de_datos}

    En una aplicación web es necesario alguna forma de persistencia de datos, en especial si se están usando datos complejos y en gran cantidad, para realizar está tarea, la base de datos es un factor primordial.
    Rails maneja la base de datos mediante  un ORM, por lo tanto la base de datos que se utilice no es tan excluyente, en este caso se utilizó  \emph{PostgreSQL} como base de datos relacional.\\

    \subsection{PostgreSQL} % (fold)
    \label{sec:postgres}

      PostgreSQL es un sistema de gestión de bases de datos objeto-relacional, Open Source y distribuido bajo licencia BSD.
      PostgreSQL utiliza un modelo cliente/servidor y usa multiprocesos en vez de multihilos para garantizar la estabilidad del sistema. Un fallo en uno de los procesos no afectará el resto y el sistema continuará funcionando.
      La última versi\'on de PostgreSQL es la 9.5, su desarrollo comenz\'o hace más de 16 años, y cuenta con una gran comunidad que aporta con el desarrollo, testeo de nuevas versiones.
      PostgreSQL  está considerada como una de los mejores \emph{Sistemas de gesti\'on de bases de datos}, es muy completo y está muy bien documentado\footnote{ http://www.postgresql.org/docs/9.5/static/}.
      Entre sus características se pueden nombrar las siguientes.
      \begin{itemize}
        \item Es una base de datos 100\% ACID\footnote{  ACID es un acrónimo de Atomicity, Consistency, Isolation and Durability}
        \item Integridad referencial
        \item Replicación asincrónica/sincrónica
        \item Múltiples métodos de autentificación
        \item Disponible para Linux y UNIX en todas sus variantes
        \item Funciones/procedimientos almacenados
        \item Soporte a la especificaci\'on SQL
      \end{itemize}

      Personalmente se escogió trabajar con  PostgreSQL como DBMS
      porque cuenta con una extensa documentación,  y gracias a su caracter ``Open Source'', y su gran flexibilidad en poder definir nuevos tipos de datos,
      se hace posible que empresas como \textbf{Refractions Research} puedan crear recursos como PostGIS, necesario para trabajar con datos geográficos \'o espaciales.

      % Entre sus principales  características se puede nombrar que es
      % \footnote{ DBMS, DataBase Management System}
      % y durante este tiempo, estabilidad, potencia, robustez, facilidad de administración e implementación de estándares han sido las características que más se han tenido en cuenta durante su desarrollo. PostgreSQL funciona muy bien con grandes cantidades de datos y una alta concurrencia de usuarios accediendo a la vez a el sistema.

    % section postgres (end)

    \subsection{PostGis} % (fold)
    \label{sec:postgis}

      PostGIS es un módulo  que a\~nade soporte de objetos geográficos al DBMS PostgreSQL, convirtiéndola en una base de datos espacial para su utilización en un Sistema de Informaci\'on Geografica(SIG\footnote{ Es bastante común utilizar el acrónimo en Inglés, Geographic Information System (GIS), de hay viene el término de PostGIS = Postgres + GIS}).

      El desarrollo de PostGIS está a cargo de \textbf{Refractions Research}, está liberada con la \emph{Licencia pública general de GNU}, declarandola como software libre y lo protege de cualquier intento de apropiaci\'on.\\

      PostGIS implementa la especificaci\'on ``SFSQL'' (Simple Features for SQL, define los tipos y funciones que necesita implementar cualquier base de datos espacial) de la OGC (Open Geospatial Consortium, es un consorcio internacional, formado por un conjunto de empresas, agencias gubernamentales y universidades, dedicado a desarrollar especificaciones de interfaces para promover y facilitar el uso global de la información espacial).\\

      PostGIS al igual que PostgreSQL tiene una documentaci\'on bastante extensa, y cuenta con equipo de desarrollo que continuamente va sacando nuevas versiones, actualmente se encuentra la versi\'on 2.0.1, pero para el desarrollo de la aplicaci\'on se hizo uso de la versi\'on 1.5.5.

      PostGIS es gratis, pero no por ello es una herramienta de baja calidad, al contrario se la considera una herramienta de nivel empresarial, y muchas instituciones la est\'an usando de manera exitosa\footnote{ http://www.postgis.org/documentation/casestudies/}, aparte de numerosas aplicaciones \\

      Manejar los datos geográficos con PostGIS es sencillo y muy eficiente, por está raz\'on se utilizó está herramienta, pero para conseguir la ruta óptima entre 2 puntos se necesitaba el uso del algoritmo de Dijkstra y para PostGIS existe el módulo \textbf{PgRouting}, que tiene implementado este algoritmo.

      \subsubsection{pgRouting} % (fold)
      \label{sec:pgrouting}
        pgRouting es una extensi\'on  de  PostGIS para proveer funcionalidades de ruteo espacial. pgRouting es un desarrollo posterior de pgDijkstra y actualmente está siendo mantenido por Georepublic, la última versi\'on estable es la 2.1, y es la que fue usada para desarrollar el sistema.\\

        Las ventajas del ruteo en la base de datos son:
        \begin{itemize}
          \item Los datos y atributos pueden ser modificados desde varios clientes, como Quantum GIS y uDig a través de JDBC, ODBC, o directamente usando Pl/pgSQL. Los clientes pueden ser PCs o dispositivos móviles.
          \item Los cambios pueden ser reflejados instantáneamente a través del motor de ruteo. No hay necesidad de hacer cálculos previos.
          \item El parámetro de ``costo'' puede ser calculado dinámicamente a través de SQL y su valor puede provenir de múltiples campos y tablas.
        \end{itemize}

        pgRouting provee funciones para:
        \begin{itemize}
          \item Camino mínimo (Dijkstra): algoritmo de ruteo sin heurística
          \item Camino mínimo (A-Star): routeo para conjunto de datos grandes (con heurística)
          \item Camino mínimo (Shooting-Star): ruteo con restricciones de giro (con heurística)
          \item El problema del viajante (TSP: Traveling Salesperon Problem)
          \item Cálculo de ruta (Isolíneas)
        \end{itemize}

        % Uses PostGIS for its geographic data format, which in turn uses OGC’s data format Well Konwn Text (WKT) and Well Known Binary (WKB)
      % section pgrouting (end)
    % section postgis (end)
  % section base_de_datos (end)
% chapter marco_referncial (end)
