
\chapter{Manual Instalación}

El Manual de instalación tomará en cuenta que todos los comandos y las herramientas son ejecutados sobre un servidor Linux Ubuntu 16.04 LTS.

\section{Instalación de la base de datos}


\begin{itemize}
  \item \textbf{PostgreSQL:} Se usará la versión 9.4.8.

  \begin{center}
    \begin{lstlisting}[label=postgres_install,caption=Comandos usados para instalar PostgreSQL.]

      $ sudo apt-get update
      $ sudo apt-get install -y postgresql postgresql-contrib
    \end{lstlisting}
  \end{center}

  \item \textbf{PostGIS:} Se usará la versión 2.1.

  \begin{center}
    \begin{lstlisting}[label=postgis_install,caption=Comandos usados para instalar PostGIS.]

      $ sudo add-apt-repository ppa:ubuntugis/ubuntugis-unstable
      $ sudo apt-get install -y postgis postgresql-9.4-postgis-2.1
    \end{lstlisting}
  \end{center}

  \item \textbf{PgRouting:} Se usará la versión 2.2.

  \begin{center}
    \begin{lstlisting}[label=pgRouting_install,caption=Comandos usados para instalar PgRouting.]

      $ sudo add-apt-repository ppa:georepublic/pgrouting-unstable
      $ sudo apt-get update
      $ sudo apt-get install postgresql-9.4-pgrouting
    \end{lstlisting}
  \end{center}




\end{itemize}

\section{Configuración de la base de datos}


\begin{itemize}
  \item Habilitar \emph{Admin pack} en PostgreSQL.
  \begin{center}
    \begin{lstlisting}[label=adminpack,caption=Habilitar Admin Pack.]

      $ sudo -u postgres psql
      > CREATE EXTENSION adminpack;
      > \q
    \end{lstlisting}
  \end{center}

  \item Crear el Usuario de la Base de Datos y crear la Base de Datos usando el usuario creado.
  \begin{center}
    \begin{lstlisting}[label=adminpack,caption=Comandos para crear el usuario y la base de datos.]

      $ sudo -u postgres createuser -d -E -i -l -P -r -s USER_NAME
      $ sudo -u postgres createdb -O USER_NAME DATABASE_NAME
    \end{lstlisting}
  \end{center}


\end{itemize}


\subsection{Añadir las funciones de PostGIS y pgRouting a la base de datos}


PostGIS y pgRouting son extensiones a la base de datos PostgreSQL, y al instalar sus paquetes, todas las funciones que vienen con las extensiones están disponibles pero para utilizarlas hay que habilitarlas específicamente en la base de datos que las va a utilizar.


% Para habilitar la base de datos es necesario correr el siguiente comando en la terminal.
Ejecutar el siguiente comando para habilitar las funciones de PostGIS y pgRouting en la base de datos.

\begin{center}
  \begin{lstlisting}[label=enable_extensions,caption=Habilitación de las Extensiones.]

        $ sudo -u postgres psql -c ``
            CREATE EXTENSION adminpack; CREATE EXTENSION postgis;
            CREATE EXTENSION postgis_topology;
            CREATE EXTENSION pgrouting;
            CREATE EXTENSION fuzzystrmatch;
            CREATE EXTENSION postgis_tiger_geocoder;
         `` DATABASE_NAME
  \end{lstlisting}
\end{center}


\begin{itemize}
  \item Se puede verificar que las extensiones instaladas están correctamente habilitadas en la base de datos.
  \begin{center}
    \begin{lstlisting}[label=enable_extensions,caption=Habilitación de las Extensiones.]

          $ psql -h localhost -U USER_NAME DATABASE_NAME

          > SELECT postgis_full_version();
          > SELECT * FROM pgr_version();
    \end{lstlisting}
  \end{center}

\end{itemize}


\subsection{Habilitar el acceso a la base de datos}

PostgreSQL por defecto viene con la configuración habilitada para conectarse a la base de datos usando conexiones seguras. Pero  para el presente proyecto se debe ampliar los permisos de acceso a la base de datos para que el servicio REST API desarrollado no presenta problemas en la conexión a la base de datos.

\begin{itemize}
  \item Editar el archivo \emph{pg\_hba.conf} ubicado en el directorio de instalación de PostgreSQL. La última sección del archivo debe quedar como en la siguiente tabla.

  \begin{center}
    \begin{tabularx}{0.75\textwidth}{ X X X X X}
      \toprule
      hostssl &  all      &       all       &     0.0.0.0/0       &        md5 \\
      local &  all       &      postgres    &                     &       trust \\
      local &  all       &     all         &                      &      trust \\
      host  &  all        &     all        &     127.0.0.1/32      &      trust \\
      host  &  all       &      all        &     ::1/128           &      trust \\
      \bottomrule
    \end{tabularx}
    % \caption{Tarea de Ingeniería - T037}
    % \label{tab:T037}
  \end{center}

\item Reiniciar el servicio de PostgreSQL para que los cambios a la configuración tomen efecto.
\begin{center}
  \begin{lstlisting}[label=postgresql_restart,caption=Comando para reiniciar el servicio PostgreSQL.]

        $ sudo service postgresql restart
  \end{lstlisting}
\end{center}


\end{itemize}



\section{Configuration de la aplicación UBIKATE UMSS}

Para el presente proyecto se utilizaron los siguientes datos para configurar la base de datos.

\begin{itemize}
  \item USER\_NAME: db\_admin
  \item DATABASE\_NAME: db\_ubikate
\end{itemize}


\subsection{Instalación de las dependencias}

% A continuación se detallara los paquetes necesarios para correcta ejecución de la aplicación.
Instalar los siguientes paquetes para la correcta ejecución de la aplicación.

\begin{itemize}
  \item \textbf{NodeJS:} Se usará la versión 6.2.2 o superior.
  \begin{center}
    \begin{lstlisting}[label=node_install,caption=Comando para instalar NodeJS.]

          $ sudo apt-get install nodejs
    \end{lstlisting}
  \end{center}


  \item \textbf{Knex:} Instalar la última versión.
  \begin{center}
    \begin{lstlisting}[label=knex_install,caption=Commando para instalar Knex.]

          $ npm install knex -g
    \end{lstlisting}
  \end{center}

\end{itemize}


\subsection{Copiar los archivos del proyecto UBIKATE UMSS}

Los archivos de proyecto irán dentro de una carpeta, localizarla en alguna ubicación donde el usuario tenga permisos de lectura y escritura.

\subsection{Configurar el mapa topológico del campus Universitario}

El \emph{shapefile} con las rutas están ubicadas dentro de la carpeta \emph{resources} ubicada en la raíz del proyecto.

Para configurar la base de datos con los rutas se debe seguir los siguientes pasos:
\begin{enumerate}
  \item Popular la tabla correspondiente a las rutas.

\begin{center}

  \begin{lstlisting}[label=psql_ways,caption=Comando popular la base de datos.]

        $ psql -d db_ubikate -U db_admin -f resources/ways.sql
  \end{lstlisting}
\end{center}

      \item Añadir las columnas \emph{source} y \emph{target} a la tabla \emph{ways}, necesarias para agregar la topología.


\begin{center}

  \begin{lstlisting}[label=add_columns,caption=Agregar las columnas \emph{source} y \emph{target}.]

        $ psql -h localhost -U db_admin db_ubikate
        > alter table ways add column "source" integer;
        > alter table ways add column "target" integer;
  \end{lstlisting}
\end{center}

  \item Crear la topología usando pgRouting.

\begin{center}
  \begin{lstlisting}[label=add_topology,caption=Añadir la topología.]

        > select pgr_createTopology('ways', 0.00000001, 'geom', 'gid');
  \end{lstlisting}
\end{center}

\end{enumerate}

\subsection{Instalar las dependencias del proyecto UBIKATE UMSS}

Una vez los archivos hayan sido copiados, es necesario instalar todas las dependencias utilizadas por el proyecto.


\begin{center}
  \begin{lstlisting}[label=npm_install,caption=Instalar dependencias de la aplicación.]

        $ npm install
  \end{lstlisting}
\end{center}

\subsection{Iniciar el servidor \emph{ExpressJS}}

Finalmente se inicia el servidor \emph{ExpressJS} y la aplicación es mandada al cliente visitando la dirección del servidor.

\begin{center}
  \begin{lstlisting}[label=npm_start,caption=Iniciar el servidor.]

        $ npm start
  \end{lstlisting}
\end{center}
