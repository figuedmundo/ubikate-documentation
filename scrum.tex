\subsection{SCRUM}
\label{sub:scrum}

  \begin{quote}

    Un marco de trabajo por el cual las personas pueden acometer problemas complejos adaptativos, a la vez que entregar productos del máximo valor posible productiva y creativamente.\footnote{http://www.scrumguides.org/docs/scrumguide/v1/Scrum-Guide-ES.pdf}

  \end{quote}

  Scrum fue creado en 1993 por Jeff Sutherland, el termino \emph{scrum} fue tomado de un estudio de 1986 de Takeuchi y Nonaka \footnote{https://hbr.org/1986/01/the-new-new-product-development-game} en el cual comparan equipos de alto rendimiento y multidisiplinarios con la formacion \emph{scrum} usado por equipos de Rugby.\cite{why_scrum}.\\

  Scrum es un marco de trabajo el cual especifica roles, procesos, actvidades y artefactos pero eso no quiere decir que sea una receta que haya que seguir al pie de la letra, encambio es una guia que nos permitara organizarnos, a continuacion se describira las caracteristicas que se usaran en el desarrollo del presente proyecto de grado.


  % The Scrum framework in 30 seconds
  % A product owner creates a prioritized wish list called a product backlog.
  % During sprint planning, the team pulls a small chunk from the top of that wish list, a sprint backlog, and decides how to implement those pieces.
  % The team has a certain amount of time — a sprint (usually two to four weeks) — to complete its work, but it meets each day to assess its progress (daily Scrum).
  % Along the way, the ScrumMaster keeps the team focused on its goal.
  % At the end of the sprint, the work should be potentially shippable: ready to hand to a customer, put on a store shelf, or show to a stakeholder.
  % The sprint ends with a sprint review and retrospective.
  % As the next sprint begins, the team chooses another chunk of the product backlog and begins working again.
  % - See more at: https://www.scrumalliance.org/why-scrum#sthash.U15K98Xq.dpuf
  %

  \subsubsection{El Equipo Scrum (Scrum Team)}
  \label{subs:scrum_team}

  El Equipo Scrum consiste en un Dueño de Producto (Product Owner), el Equipo de Desarrollo       (Development Team) y un Scrum Master. Los Equipos Scrum son autoorganizados y      multifuncionales. Los equipos autoorganizados eligen la mejor forma de llevar a cabo su trabajo y no son dirigidos por personas externas al equipo. Los equipos multifuncionales tienen todas las competencias necesarias para llevar a cabo el trabajo sin depender de otras personas que no son parte del equipo. El modelo de equipo en Scrum está diseñado para optimizar la flexibilidad, la creatividad y la productividad.

  \begin{description}
    \item[Due\~no de Producto (Product Owner)] Debe ser una persona con vision, autoridad y disponibilidad. Es el experto del producto y las prioridades y necesidades del cliente. Trabaja con el equipo de desarrollo aclarando dudas acerca de los requerimientos. Es la persona encargada de charlar con los clientes y traducir sus necesidades para el equipo de desarrollo.\\
    El rol de \emph{product owner} estara a cargo del docente de la materia de Proyecto Final.

    \item[El equipo de Desarrollo] Es el grupo de personas que hacen el trabajo de crear el producto, dentro de este rol entran los programadores, testers, escritores, cualquier persona que forme parte del desarrolo del producto es parte del \emph{equipo de desarollo}.\\
    El rol de \emph{equipo de desarrollo} estara a cargo de mi persona.

    \item[Scrum Master] Es la persona responsable de asegurase que el equipo de desarrollo entiende e implementa \emph{Scrum}. El Scrum Master no maneja el equipo de desarrollo encambio lo ayuda a organizarse maximizando el valor creado por el equipo de desarrollo.\\
    El rol de \emph{Scrum Master} estara a cargo del Tutor.\\

    Dentro del equipo Scrum tambien se reconocen a los \emph{Stakeholders} que son personas que tienen algun interes y/o estan afectadas por el producto y pueden dar feedback hacerca de este pero no son responsables del producto. Este rol no sera implementado.

  \end{description}

  % end scrum_team

  \subsubsection{Eventos de Scrum}
  \label{subs:scrum_events}
  Eventos o actividades, estos procesos estan dise\~nados para mejorar la eficiencia  y performance del equipo scrum.

  \begin{description}
    \item[Sprint]
    El corazón de Scrum es el Sprint, es un bloque de tiempo (time-box) generalmente consistente en 2 semanas durante  cual se crea un incremento de producto “Terminado”, utilizable y potencialmente desplegable. Es más conveniente si la duración de los Sprints es consistente a lo largo del esfuerzo de desarrollo. Cada nuevo Sprint comienza inmediatamente después de la finalización del Sprint previo.

    Los Sprints contienen y consisten de la Reunión de Planificación del Sprint (Sprint Planning Meeting), los Scrums Diarios (Daily Scrums), el trabajo de desarrollo, la Revisión del Sprint (Sprint Review), y la Retrospectiva del Sprint (Sprint Retrospective).

    \item[Daily Scrum]

    El Scrum Diario es una reunión con un bloque de tiempo de 15 minutos para que el Equipo de Desarrollo sincronice sus actividades y cree un plan para las siguientes 24 horas. Esto se lleva a cabo inspeccionando el trabajo avanzado desde el último Scrum Diario y haciendo una proyección acerca del trabajo que podría completarse antes del siguiente. \\
    Durante la reunión, cada miembro del Equipo de Desarrollo explica:
    \begin{itemize}
      \item Que hize ayer?
      \item Que voy a hacer hoy?
      \item Tengo algun problema?
    \end{itemize}

  \end{description}

  % end scrum_events

  \subsubsection{Artefactos de Scrum}
  \label{subs:artefactos_de_scrum}
    Los artefactos de Scrum representan trabajo o valor en diversas formas.

    \begin{description}
      \item[Lista de Producto (Product Backlog)]

      La Lista de Producto es una lista ordenada de todo lo que podría ser necesario en el producto, y es la única fuente de requisitos para cualquier cambio a realizarse en el producto. El Dueño de Producto (Product Owner) es el responsable de la Lista de Producto, incluyendo su contenido, disponibilidad y ordenación.

      % La Lista de Producto no esta escrita en piedra, se puede modificar y a\~nadir  // pero  no eliminar.  // requisitos durante el ciclo de desarrollo de software.

      \item[Lista de pendientes del Sprint (Sprint Backlog)]

      La Lista de Pendientes del Sprint es el conjunto de elementos de la Lista de Producto seleccionados para el Sprint, más un plan para entregar el Incremento de producto y conseguir el Objetivo del Sprint. La Lista de Pendientes del Sprint es una predicción hecha por el Equipo de Desarrollo acerca de qué funcionalidad formará parte del próximo Incremento y del trabajo necesario para entregar esa funcionalidad en un Incremento “Terminado”.

    \end{description}
  % end artefactos_de_scrum
% end scrum
