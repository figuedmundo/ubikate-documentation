\chapter*{Resumen} % si no queremos que añada la palabra "Capitulo"
\addcontentsline{toc}{chapter}{Resumen} % si queremos que aparezca en el índice
\markboth{RESUMEN}{RESUMEN} % encabezado

% Una bonita historia

% \newpage
% \pdfbookmark[0]{Resumen}{Resumen} % Sets a PDF bookmark for the abstract
% \chapter*{Resumen}
% % \textsc{The celebrated number} -17 was discovered in Manchester in 1989 ...


% El presente projecto de grado, es

El presente proyecto de grado se trabajó sobre el problema que presenta la Universidad Mayor de San Simón en relación a la deficiente localización de los lugares que se encuentran dentro del campus Universitario ``Las Cuadras'', esto se debe a varios factores, entre los cuales podemos mencionar los escasos mapas con ubicaciones de lugares con que cuentan las distintas facultades, la reestructuración que cada cierto tiempo se lleva a cabo pero los mapas no son actualizados.

Con la aplicación desarrollada se trata de mitigar el problema planteado, con la implementación de una aplicación web móvil que pueda localizar un lugar dentro del campus Universitario, mostrado su ubicación en un mapa y además la posición actual del usuario, para así poder mostrar la ruta óptima que el usuario puede seguir para poder llegar a su destino, usando geolocalización además de las diversas características con que cuentan los dispositivos móviles actuales.

Durante la ejecución del proyecto se analizaron y utilizaron diversas herramientas, como por ejemplo: \emph{EmberJS}, el cual es un entorno de desarrollo \emph{JavaScript}, especializado en el \emph{frontend} o la capa de presentación de la aplicación, \emph{PostgreSQL} como base de datos con \emph{PostGIS} para el procesamiento de datos Geoespaciales, \emph{ExpressJS} para implementar un API o \emph{backend} que maneje la lógica del negocio.

 % para la correcta implementación del proceso de desarrollo de software se utilizó la \emph{Programación Extrema} como metodología de desarrollo, la cual así como todas herramientas utilizadas son desglosadas y analizadas  con más detalle en el Marco Teórico del proyecto de grado.

En una primera etapa se construyó un mapa de las rutas internas dentro el campus Universitario,  utilizando un dispositivo GPS y recorriendo los caminos peatonales, este mapa resultante tiene las características de un \emph{grafo no dirigido}, sobre el cual se aplica el algoritmo de Dijkstra para obtener el camino más corto entre 2 nodos del grafo, los cuales corresponden al lugar de interés que se está buscando y a la posición del usuario respectivamente.

Posteriormente se implementó la lógica necesaria para poder mostrar al usuario de forma gráfica los lugares disponibles dentro del Campus Universitario, porque al final los datos geoespaciales de los lugares o de la ruta óptima son fácilmente apreciados de forma visual sobre un mapa.

% Finalizando la implementación de la aplicación se enfocó la forma de manejar los usuarios que pueden aumentar y mejorar la base de datos de lugares disponibles, pero ya que se cuenta con información costosa, es necesario un tratamiento especial de los usuarios en el sistema.

La información obtenida por el uso de la aplicación es bastante valiosa, ya que refleja la relación de personas que desean localizar algún lugar en específico, con la consiguiente posibilidad de  tomar decisiones según la cantidad de personas que se mueven por los distintos puntos de interés con los que cuenta la Universidad Mayor de San Simón.
