\chapter{Implementación del Proyecto}
\label{chap:planificacion}


%   \section{Análisis}
%   \label{sec:Analisis}
% %
%     % La programación extrema no contempla específicamente el diseño de casos de uso
%     Dentro del análisis se realizó el diseño de Casos de Uso como parte del proceso para entender el comportamiento del sistema. \cite{casos_uso}
% %
%
% \begin{figure}[H]
%   \begin{center}
%     \includegraphics[width=0.45\textwidth]{casos_uso/cu_lugares}
%   \end{center}
%   \caption{Diagrama Casos de Uso - Gestión de Lugares }
%   \label{fig:cu_lugares}
%   \caption*{Fuente: Elaboración propia}
% \end{figure}
%
% \begin{table}[H]
  \begin{center}
    \begin{tabularx}{0.75\textwidth}{ X X  }
      \toprule
      \multicolumn{2}{l}{\textbf{Caso de Uso:} Buscar Lugar} \\
      \multicolumn{2}{l}{\textbf{Actor:} Visitante} \\
      \multicolumn{2}{l}{\textbf{Precondición:} El Visitante accede al sistema.} \\
      \addlinespace
      \textbf{Secuencia Principal} & \textbf{Alternativas} \\
      \midrule
      1. El Visitante accede a la sección de ``Lugares''. \\
      \addlinespace
      2. El visitante ingresa el nombre del lugar en un \emph{search box}.  \\
      \addlinespace
      3. El Visitante ve la lista de Lugares registrados que coinciden con la búsqueda. &
      3.1 El lugar no existe en la base de datos, retorna una lista vacía. \\

      \midrule
      \multicolumn{2}{l}{\textbf{Postcondición:} El Visitante encuentra un lugar} \\

      \bottomrule
    \end{tabularx}
    \caption{Caso de Uso - Buscar Lugar}
    \label{tab:cu_buscar_lugar}
  \end{center}
\end{table}


% \begin{table}[H]
%   \begin{center}
%     \begin{tabularx}{0.75\textwidth}{  L{3cm} X  }
%       \toprule
%       \textbf{Caso de Uso:} & Buscar Lugar \\
%       \textbf{Actor:} & Visitante \\
%       \textbf{Precondición:} & El Visitante accede al sistema. \\
%       \textbf{Secuencia Principal:} & 1. El Visitante accede a la sección de ``Lugares''. \\
%       \addlinespace
%       & 2. El visitante ingresa el nombre del lugar en un \emph{search box}.  \\
%       \addlinespace
%       & 3. El Visitante ve la lista de Lugares registrados que coinciden con la búsqueda. \\
%
%       \addlinespace
%       \textbf{Postcondición:} & El Visitante encuentra un lugar. \\
%       \bottomrule
%     \end{tabularx}
%     \caption{Caso de Uso - Buscar Lugar}
%     \label{tab:cu_buscar_lugar}
%   \end{center}
% \end{table}




\begin{table}[H]
  \begin{center}
    \begin{tabularx}{0.75\textwidth}{ X X  }
      \toprule
      \multicolumn{2}{l}{\textbf{Caso de Uso:} Ver Información del Lugar} \\
      \multicolumn{2}{l}{\textbf{Actor:} Visitante} \\
      \multicolumn{2}{l}{\textbf{Precondición:} El Visitante ha buscado un lugar} \\
      \addlinespace
      \textbf{Secuencia Principal} & \textbf{Alternativas} \\
      \midrule
      1. El Visitante hace un tap sobre el nombre del lugar en la lista de búsqueda. \\
      \addlinespace
      2. El visitante puede ver una imagen del lugar. &
      2.1. Si el lugar no tiene una imagen asociada, se desplegará una imagen genérica de la UMSS.\\
      \addlinespace
      3. El Visitante ve la descripción del lugar. & \\

      \midrule
      \multicolumn{2}{l}{\textbf{Postcondición:} El Visitante puede ver la información del lugar.} \\
      \bottomrule
    \end{tabularx}
    \caption{Caso de Uso - Ver Información del lugar}
    \label{tab:cu_info_lugar}
  \end{center}
\end{table}


\begin{table}[H]
  \begin{center}
    \begin{tabularx}{0.75\textwidth}{ X X  }
      \toprule
      \multicolumn{2}{l}{\textbf{Caso de Uso:} Ver Ruta Óptima} \\
      \multicolumn{2}{l}{\textbf{Actor:} Visitante} \\
      \multicolumn{2}{l}{\textbf{Precondición:} El Visitante observa la información del lugar.} \\
      \addlinespace
      \textbf{Secuencia Principal} & \textbf{Alternativas} \\
      \midrule
      1. El Visitante selecciona el botón que muestra la ruta más corta al lugar. & \\
      \addlinespace
      2. El sistema pregunta si el usuario quiere compartir su ubicación geográfica. & \\
      \addlinespace
      3. El Visitante acepta la pregunta. &
      3.1. Si el usuario niega la pregunta, el sistema no puede mostrar la ruta óptima. \\

      \midrule
      \multicolumn{2}{L{11cm}}{\textbf{Postcondición:} El Visitante ve un mapa con un marcador y una línea mostrando la ruta óptima. } \\

      \bottomrule
    \end{tabularx}
    \caption{Caso de Uso - Ver Ruta Óptima}
    \label{tab:cu_ruta_optima}
  \end{center}
\end{table}




% \begin{table}[H]
%   \begin{center}
%     \begin{tabularx}{0.75\textwidth}{ X X  }
%       \toprule
%       \multicolumn{2}{l}{\textbf{Caso de Uso:} Ver Ruta Óptima} \\
%       \multicolumn{2}{l}{\textbf{Actor:} Visitante} \\
%       \multicolumn{2}{l}{\textbf{Precondición:} El Visitante está viendo la información del lugar.} \\
%       \addlinespace
%       \textbf{Secuencia Principal} & \textbf{Alternativas} \\
%       \midrule
%       1. El Visitante selecciona el botón que muestra la ruta más corta al lugar. & \\
%       \addlinespace
%       2. El sistema pregunta si el usuario quiere compartir su ubicación geográfica. & \\
%       \addlinespace
%       3. El Visitante acepta la pregunta. &
%       3.1. Si el usuario niega la pregunta, el sistema no puede mostrar la ruta óptima.\\
%
%       \midrule
%       \multicolumn{2}{l}{\textbf{Postcondición:} El Visitante ve un mapa con el lugar marcado y una línea que conecta la posición actual del usuario con el lugar.} \\
%
%       \bottomrule
%     \end{tabularx}
%     \caption{Caso de Uso - Ver Ruta Óptima}
%     \label{tab:cu_ruta_optima}
%   \end{center}
% \end{table}
%
%



\begin{table}[H]
  \begin{center}
    \begin{tabularx}{0.75\textwidth}{ X X  }
      \toprule
      \multicolumn{2}{l}{\textbf{Caso de Uso:} Editar Información del Lugar} \\
      \multicolumn{2}{l}{\textbf{Actor:} Visitante Registrado} \\
      \multicolumn{2}{L{12cm}}{\textbf{Precondición:} El Visitante Registrado esta viendo la información del lugar.} \\
      \addlinespace
      \textbf{Secuencia Principal} & \textbf{Alternativas} \\
      \midrule
      1. El Visitante Registrado selecciona el boton de Edicion. \\
      \addlinespace
      2. El sistema muestra un formulario con la información actual del Lugar.& \\
      \addlinespace
      3. El Visitante Registrado selecciona el botón ``Aceptar''. & \\


      \midrule
      \multicolumn{2}{L{11cm}}{\textbf{Postcondición:} El sistema muestra la información del Lugar actualizada.} \\

      \bottomrule
    \end{tabularx}
    \caption{Caso de Uso - Editar Información del Lugar}
    \label{tab:cu_edit_place}
  \end{center}
\end{table}


\begin{table}[H]
  \begin{center}
    \begin{tabularx}{0.75\textwidth}{ X X  }
      \toprule
      \multicolumn{2}{l}{\textbf{Caso de Uso:} Añadir Lugar} \\
      \multicolumn{2}{l}{\textbf{Actor:} Visitante Registrado} \\
      \multicolumn{2}{L{12cm}}{\textbf{Precondición:} El Visitante Registrado necesita desplazarse hasta el lugar para añadirlo.} \\
      \addlinespace
      \textbf{Secuencia Principal} & \textbf{Alternativas} \\
      \midrule
      1. El Visitante Registrado ha accedido a la sección de ``Lugares''. & \\
      \addlinespace
      2. El Visitante Registrado puede ver un botón de adición al final de la lista de lugares. &\\
      \addlinespace
      3. El sistema pregunta si se desea compartir la ubicación geográfica.
      \addlinespace
      3. El Visitante Registrado acepta la pregunta. &
      3.1. El visitante niega la pregunta.\\
      \addlinespace
      4. El Visitante Registrado ve un formulario para ingresar la información del lugar. &
      4.1. El Visitante Registrado no puede anadir un nuevo lugar. \\
      \addlinespace
      5. El Visitante Registrado acepta el formulario. & \\

      \midrule
      \multicolumn{2}{l}{\textbf{Postcondición:} El sistema muestra el lugar en la lista de lugares.} \\

      \bottomrule
    \end{tabularx}
    \caption{Caso de Uso - Añadir Lugar}
    \label{tab:cu_add_place}
  \end{center}
\end{table}

%
% \begin{figure}[H]
%   \begin{center}
%     \includegraphics[width=0.45\textwidth]{casos_uso/cu_usuarios}
%   \end{center}
%   \caption{Diagrama Casos de Uso - Gestion de Usuarios }
%   \label{fig:cu_usuarios}
%   \caption*{Fuente: Elaboración propia}
% \end{figure}
%
% \begin{table}[H]
  \begin{center}
    \begin{tabularx}{0.75\textwidth}{ X X  }
      \toprule
      \multicolumn{2}{l}{\textbf{Caso de Uso:} Solicitar Registro} \\
      \multicolumn{2}{l}{\textbf{Actor:} Visitante} \\
      \multicolumn{2}{l}{\textbf{Precondición:} El Visitante accede al sistema.} \\
      \addlinespace
      \textbf{Secuencia Principal} & \textbf{Alternativas} \\
      \midrule
      1. El Visitante accede a la sección de ``Registro''. \\
      \addlinespace
      2. El visitante ingresa su nombre y una direccion email.  \\

      \midrule
      \multicolumn{2}{l}{\textbf{Postcondición:} El Visitante ve un mensaje de confirmacion} \\

      \bottomrule
    \end{tabularx}
    \caption{Caso de Uso - Solicitar Registro}
    \label{tab:cu_solicitar_registro}
  \end{center}
\end{table}


\begin{table}[H]
  \begin{center}
    \begin{tabularx}{0.75\textwidth}{ X X  }
      \toprule
      \multicolumn{2}{l}{\textbf{Caso de Uso:} Registrar Usuario} \\
      \multicolumn{2}{l}{\textbf{Actor:} Administrador} \\
      \multicolumn{2}{l}{\textbf{Precondición:} El Administrador accede al sistema.} \\
      \addlinespace
      \textbf{Secuencia Principal} & \textbf{Alternativas} \\
      \midrule
      1. El Administrador accede a la sección de ``Registro de Usuarios''. \\
      \addlinespace
      2. El Administrador ve una lista con de usuarios a registrar.  \\
      \addlinespace
      3. El Administrador acepta el registro del usuario.  \\
      \addlinespace
      3. El sistema manda un email con instrucciones al usuario.  \\
      \addlinespace
      4. Un mensaje confirmando el envio del email es mostrado. & 4.1. Si el email no existe, un mensaje es mostrado y el registro del usuario es cancelado.  \\

      \midrule
      \multicolumn{2}{L{12cm}}{\textbf{Postcondición:} El Visitante deberia recibir un email con instrucciones de registro.} \\

      \bottomrule
    \end{tabularx}
    \caption{Caso de Uso - Registrar Usuario}
    \label{tab:cu_solicitar_registro}
  \end{center}
\end{table}



\begin{table}[H]
  \begin{center}
    \begin{tabularx}{0.75\textwidth}{ X X  }
      \toprule
      \multicolumn{2}{l}{\textbf{Caso de Uso:} Eliminar Usuarios Registrados} \\
      \multicolumn{2}{l}{\textbf{Actor:} Administrador} \\
      \multicolumn{2}{l}{\textbf{Precondición:} El Administrador accede al sistema.} \\
      \addlinespace
      \textbf{Secuencia Principal} & \textbf{Alternativas} \\
      \midrule
      1. El Administrador accede a la sección de ``Usuarios Registrados''. \\
      \addlinespace
      2. El Administrador elimina un usuario registrado.  \\
      \addlinespace
      3. Un mensaje confirmando la eliminacion del usuario es mostrado.\\

      \midrule
      \multicolumn{2}{L{11.5cm}}{\textbf{Postcondición:} El Usuario eliminado ya no deberia ser visible en la lista de Usuarios registrados.} \\

      \bottomrule
    \end{tabularx}
    \caption{Caso de Uso - Eliminar Usuarios Registrados}
    \label{tab:cu_solicitar_registro}
  \end{center}
\end{table}

%
%
% \begin{figure}[H]
%   \begin{center}
%     \includegraphics[width=0.45\textwidth]{casos_uso/cu_reporte}
%   \end{center}
%   \caption{Diagrama Casos de Uso - Reporte }
%   \label{fig:cu_reporte}
%   \caption*{Fuente: Elaboración propia}
% \end{figure}
%
% \begin{table}[H]
  \begin{center}
    \begin{tabularx}{0.75\textwidth}{ X X  }
      \toprule
      \multicolumn{2}{l}{\textbf{Caso de Uso:} Ver Reporte} \\
      \multicolumn{2}{l}{\textbf{Actor:} Administrador} \\
      \multicolumn{2}{l}{\textbf{Precondición:} El Administrador accede al sistema.} \\
      \addlinespace
      \textbf{Secuencia Principal} & \textbf{Alternativas} \\
      \midrule
      1. El Administrador accede a la sección de ``Reporte''. \\
      \addlinespace
      2. El Sistema genera una grafica en relacion a los lugares mas visitados.  \\
      \addlinespace
      3. Se muestra un mensaje preguntado si se desea guardar el reporte.\\

      \midrule
      \multicolumn{2}{l}{\textbf{Postcondición:} El Adminstrador puede guardar un archivo con el ``Reporte''} \\

      \bottomrule
    \end{tabularx}
    \caption{Caso de Uso - Ver Reporte}
    \label{tab:cu_ver_reporte}
  \end{center}
\end{table}



% Siguiendo la metodología XP, el presente proyecto de grado debe empezar por la primera etapa del Planning Game.\\


% Exploración - Entrega y estimación de las historias de usuario


\section{Exploración}
\label{sec:requerimientos}


Como parte del proceso de Exploración se empezó identificando los usuarios que tomarán parte del sistema y los requerimientos funcionales que el sistema deberá proveer, para posteriormente definir las \emph{historias de usuario}.

% Como parte del análisis, se procedió a elaborar la lista de requerimientos funcionales y no-funcionales, los requerimientos funcionales vienen a ser las características o servicios que el sistema proveerá a sus usuarios y los requerimientos no funcionales son las propiedades o restricciones del sistema.

\subsection{Identificación de Usuarios}

Se puede identificar 3 tipos de usuario en el sistema:

\begin{itemize}

\item \textbf{Usuario Visitante:} Un usuario visitante puede ver los lugares que están registrados en el sistema, su información y la ruta óptima al lugar.

\item \textbf{Usuario Registrado:} Un usuario registrado tiene privilegios para agregar lugares y poder editarlos según se requiera.

\item \textbf{Usuario Administrador:} Un administrador tiene privilegios para remover usuarios registrados que estén haciendo uso indebido del sistema y además pueden ver los reportes que el sistema ofrezca.

\end{itemize}


\subsection{Requerimientos Funcionales}

\LTXtable{0.75\textwidth}{planificacion/requerimientos_funcionales}

% \subsection{Requerimientos No Funcionales}
%
% \LTXtable{0.75\textwidth}{implementacion/requerimientos_no_funcionales}




  \subsection{Historias de Usuario}
  \label{sub:historias_de_usuario}

    Las \emph{Historias de Usuario}, están escritas en lenguaje del cliente (no técnico) y serán la pauta para  determinar que los requerimientos del sistema están correctamente implementados, en la siguiente tabla \ref{tab:us_table}, están nombradas las \emph{historias de usuario} que serán implementadas en el presente proyecto de grado, cada historia de usuario será analizada con más detalle en la \emph{iteración} en la que será implementada. \\

    % % % user_story_01
% % \begin{table}[!ht]
% \begin{table}[H]
%   \begin{center}
%     \begin{tabular}{ L{3cm}  L{8cm} }
%       \toprule
%         \textbf{Historia} US01 &
%         % \textbf{Esfuerzo} 5 puntos \\
%         \makebox[6cm][r]{\textbf{Esfuerzo} 8 puntos} \\
%         % \makebox[4cm][r]{\textbf{Estimación} 3 días} \\
%
%       \midrule
%       \multirow{3}{*}{\textbf{Descripción}}
%         & Yo como visitante\\
%         & Deseo registrarme en el sistema\\
%         & Para poder acceder al sistema\\
%       \midrule
%         \multirow{3}{3cm}{\textbf{Criterios de Aceptación}}
%         & Quiero ver fácilmente que no estoy registrado\\
%         & Quiero que el registro solamente me pida un nombre de usuario y un password\\
%         & Quiero ver que al estar registrado pueda acceder a los lugares\\
%       \bottomrule
%     \end{tabular}
%     \caption{Historia de Usuario - US01}
%     \label{tab:user_story_01}
%   \end{center}
% \end{table}

% user_story_02

% user_story_03

\begin{table}[H]
  \begin{center}
    \begin{tabular}{ L{3cm}  L{8cm} }
      \toprule
        \textbf{Codigo:} & US01 \\
        \textbf{Prioridad:} & Alta \\
        \textbf{Riesgo:} & Alta \\

        % \addlinespace
      \midrule
        \multirow{3}{*}{\textbf{Descripción:}}
        & Yo como visitante\\
        & Deseo ver una lista de lugares \\
        % & Deseo ingresar el nombre de un lugar\\
        & Para encontrar el lugar al que deseo ir\\
        \addlinespace
      % \midrule
        \multirow{3}{3cm}{\textbf{Criterios de Aceptación:}}
        & Quiero tener los lugares en una base de datos \\
        & Quiero ver una lista de lugares\\
        & Quiero filtrar la lista de lugares por el nombre o parte de este\\
        % & Quiero encontrar un lugar
        % & Quiero encontrar un lugar y poder ver su información\\
      \bottomrule
    \end{tabular}
    \caption{Historia de Usuario - US01}
    \label{tab:user_story_01}
  \end{center}
\end{table}

\begin{table}[H]
  \begin{center}
    \begin{tabular}{ L{3cm}  L{8cm} }
      \toprule
        \textbf{Historia} US02 &
        % \textbf{Esfuerzo} 5 puntos \\
        \makebox[6cm][r]{\textbf{Esfuerzo} 8 puntos} \\
        % \makebox[4cm][r]{\textbf{Estimación} 3 días} \\

      \midrule
        \multirow{3}{*}{\textbf{Descripción}}
        & Yo como visitante\\
        & Deseo ver la información de un lugar\\
        & Para decidir si es el lugar que estoy buscando\\
      \midrule
        \multirow{3}{3cm}{\textbf{Criterios de Aceptación}}
        & Quiero leer una descripción del lugar\\
        & Quiero ver un teléfono asociado al lugar\\
        & Quiero ver en qué piso se encuentra el lugar\\
      \bottomrule
    \end{tabular}
    \caption{Historia de Usuario - US02}
    \label{tab:user_story_02}
  \end{center}
\end{table}



% user_story_03

\begin{table}[H]
  \begin{center}
    \begin{tabular}{ L{3cm}  L{8cm} }
      \toprule
        \textbf{Historia} US03 &
        % \textbf{Esfuerzo} 5 puntos \\
        \makebox[6cm][r]{\textbf{Esfuerzo} 8 puntos} \\
        % \makebox[4cm][r]{\textbf{Estimación} 3 días} \\

      \midrule
        \multirow{3}{*}{\textbf{Descripción}}
        & Yo como visitante\\
        & Deseo ver el lugar en un mapa\\
        & Para saber en que parte del campus se encuentra el lugar\\
      \midrule
        \multirow{3}{3cm}{\textbf{Criterios de Aceptación}}
        & Deseo ver sobre un mapa un punto del lugar buscado a donde quiero ir\\
        & Quiero ver un marcador sobre el lugar que estoy buscando con alguna información para asegurarme que es a donde quiero ir\\

      \bottomrule
    \end{tabular}
    \caption{Historia de Usuario - US03}
    \label{tab:user_story_03}
  \end{center}
\end{table}


% user_story_04

\begin{table}[H]
  \begin{center}
    \begin{tabular}{ L{3cm}  L{8cm} }
      \toprule
        \textbf{Historia} US04 &
        % \textbf{Esfuerzo} 5 puntos \\
        \makebox[6cm][r]{\textbf{Esfuerzo} 8 puntos} \\
        % \makebox[4cm][r]{\textbf{Estimación} 3 días} \\

      \midrule
        \multirow{3}{*}{\textbf{Descripción}}
        & Yo como visitante\\
        & Deseo ver una ruta sobre el mapa\\
        & Para encontrar el lugar de forma rapida\\
      \midrule
        \multirow{1}{3cm}{\textbf{Criterios de Aceptación}}
        & Deseo ver sobre un mapa un punto del lugar actual donde me encuentro\\

        & Deseo ver una línea roja que muestre la ruta más corta para llegar de mi ubicación al lugar donde quiero ir\\

      \bottomrule
    \end{tabular}
    \caption{Historia de Usuario - US04}
    \label{tab:user_story_04}
  \end{center}
\end{table}


% user_story_04

\begin{table}[H]
  \begin{center}
    \begin{tabular}{ L{3cm}  L{8cm} }
      \toprule
        \textbf{Historia} US05 &
        % \textbf{Esfuerzo} 5 puntos \\
        \makebox[6cm][r]{\textbf{Esfuerzo} 8 puntos} \\
        % \makebox[4cm][r]{\textbf{Estimación} 3 días} \\

      \midrule
        \multirow{3}{*}{\textbf{Descripción}}
        & Yo como visitante\\
        & Deseo registrarme\\
        & Para tener mas opciones dentro el sistema\\
      \midrule
        \multirow{3}{3cm}{\textbf{Criterios de Aceptación}}
        & Quiero ver un formulario donde me pueda registrar.\\
        & Una vez registrado quiero poder ingresar al sistema con mis credenciales.\\
        & Quiero ver tener la posibilidad de editar mis datos.\\
      \bottomrule
    \end{tabular}
    \caption{Historia de Usuario - US05}
    \label{tab:user_story_05}
  \end{center}
\end{table}



% user_story_05

\begin{table}[H]
  \begin{center}
    \begin{tabular}{ L{3cm}  L{8cm} }
      \toprule
        \textbf{Historia} US06 &
        % \textbf{Esfuerzo} 5 puntos \\
        \makebox[6cm][r]{\textbf{Esfuerzo} 8 puntos} \\
        % \makebox[4cm][r]{\textbf{Estimación} 3 días} \\

      \midrule
        \multirow{3}{*}{\textbf{Descripción}}
        & Yo como usuario registrado\\
        & Deseo añadir más lugares al sistema\\
        & Para mejorar los criterios de busqueda\\
      \midrule
        \multirow{3}{3cm}{\textbf{Criterios de Aceptación}}
        & Quiero que sea posible anadir un lugar si no lo encuentro en la lista de lugares\\
        & Quiero ver un formulario para poder ingresar los datos de un nuevo lugar.\\
        & Quiero pararme cerca o en el lugar que necesito añadir para geo-referenciarlo\\
        % & Al añadir un lugar necesito ingresar alguna descripción y/o teléfono si fuera necesario\\
      \bottomrule
    \end{tabular}
    \caption{Historia de Usuario - US06}
    \label{tab:user_story_06}
  \end{center}
\end{table}



% user_story_06

\begin{table}[H]
  \begin{center}
    \begin{tabular}{ L{3cm}  L{8cm} }
      \toprule
        \textbf{Historia} US07 &
        % \textbf{Esfuerzo} 5 puntos \\
        \makebox[6cm][r]{\textbf{Esfuerzo} 8 puntos} \\
        % \makebox[4cm][r]{\textbf{Estimación} 3 días} \\

      \midrule
        \multirow{3}{*}{\textbf{Descripción}}
        & Yo como usuario registrado\\
        & Deseo editar la información de un lugar\\
        & Para mejorar o corregir la información de ese lugar\\
      \midrule
        \multirow{3}{3cm}{\textbf{Criterios de Aceptación}}
        & Al entrar a la información de un lugar quiero ser el único que vea un icono para poder entrar a la edición de los datos\\
        & Quiero acceder a un formulario que muestre la información actual del lugar y poder editar la información mostrada\\

      \bottomrule
    \end{tabular}
    \caption{Historia de Usuario - US07}
    \label{tab:user_story_07}
  \end{center}
\end{table}

% user_story_07

\begin{table}[H]
  \begin{center}
    \begin{tabular}{ L{3cm}  L{8cm} }
      \toprule
        \textbf{Historia} US08 &
        % \textbf{Esfuerzo} 5 puntos \\
        \makebox[6cm][r]{\textbf{Esfuerzo} 8 puntos} \\
        % \makebox[4cm][r]{\textbf{Estimación} 3 días} \\

      \midrule
        \multirow{3}{*}{\textbf{Descripción}}
        & Yo como usuario administrador\\
        & Deseo administrar usuarios\\
        & Para anadir o remover usuarios del sistema\\
      \midrule
        \multirow{3}{3cm}{\textbf{Criterios de Aceptación}}
        & Quiero ver los usuarios que desean registrarse en el sistema\\
        & Quiero aceptar o rechazar solicitudes de registro\\
        & Quiero eliminar usuarios que no usen el sistema de forma adecuada.\\

      \bottomrule
    \end{tabular}
    \caption{Historia de Usuario - US08}
    \label{tab:user_story_08}
  \end{center}
\end{table}


% user_story_08

\begin{table}[H]
  \begin{center}
    \begin{tabular}{ L{3cm}  L{8cm} }
      \toprule
        \textbf{Historia} US09 &
        % \textbf{Esfuerzo} 5 puntos \\
        \makebox[6cm][r]{\textbf{Esfuerzo} 13 puntos} \\
        % \makebox[4cm][r]{\textbf{Estimación} 3 días} \\

      \midrule
        \multirow{3}{*}{\textbf{Descripción}}
        & Yo como usuario administrador\\
        & Deseo ver los lugares más visitados\\
        & Para obtener información y estadísticas de los lugares dentro del campus Universitario\\
      \midrule
        \multirow{3}{3cm}{\textbf{Criterios de Aceptación}}
        & Quiero apreciar de forma sencilla la cantidad de veces que los usuarios buscan un lugar\\
        & Quiero poder guardarla el reporte\\

      \bottomrule
    \end{tabular}
    \caption{Historia de Usuario - US09}
    \label{tab:user_story_09}
  \end{center}
\end{table}

    \begin{table}[H]

  \begin{center}
    \begin{tabularx}{0.75\textwidth}{ c X }

    %% \begin{longtable}{ c X }
      \toprule
        \textbf{C\'odigo} &
        \textbf{Nombre de la Historia de Usuario} \\

      \midrule
      \addlinespace
      US01 & Implementar la lista de lugares.\\

      \addlinespace
      US02 & Implementar la vista de la información del lugar.\\

      \addlinespace
      US03 & Georeferenciar un lugar sobre el mapa del campus Universitario.\\
      \addlinespace
      US04 & Encontrar la ruta óptima entre 2 puntos dentro del campus Universitario.\\
      \addlinespace
      US05 & Implementar el módulo de Registro de un Visitante.\\
      \addlinespace
      US07 & Editar la información de un lugar.\\
      \addlinespace
      US06 & Añadir más lugares al sistema.\\
      \addlinespace
      US08 & Implementar el módulo para Administrar Usuarios.\\
      \addlinespace
      US09 & Implementar el reporte de los lugares más visitados.\\
      \addlinespace

      \bottomrule
    \end{tabularx}

    \caption{Historias de Usuario}
    \label{tab:us_table}
%
  \end{center}
\end{table}

    % \LTXtable{0.75\textwidth}{planificacion/us_table}

  % end historias_de_usuario


\\
  \section{Planificación de la Entrega}
  \label{sub:Planificación de la Entrega}


  \subsection{Plan de Entregas}

  Como parte del proceso XP, el equipo de desarrollo estima el esfuerzo que requerirá la implementación de las \emph{historias de usuario}, tal como se puede apreciar en la tabla \ref{tab:estimation_user_stories}.


\begin{table}[H]

  \begin{center}
    \begin{tabularx}{0.5\textwidth}{ XXc }
      \toprule
        \textbf{Codigo} &
        \textbf{Prioridad} &
        \textbf{Esfuerzo} \\
        &&\textbf{[puntos]} \\

      \midrule
      US01 & Alta & 8 \\
      US02 & Media & 3 \\
      US03 & Alta & 8 \\
      US04 & Alta & 13 \\
      US05 & Baja & 3 \\
      US06 & Media & 8 \\
      US07 & Media & 3 \\
      US08 & Alta & 3 \\
      US09 & Alta & 8 \\


      \bottomrule
    \end{tabularx}
    \caption{Estimacion de las historias de usuario}
    \label{tab:estimation_user_stories}
  \end{center}
\end{table}


    Como parte del plan de entregas, se definió que cada iteración será de 2 semanas, como se puede ver en la figura \ref{fig:calendario_entregas}.

      % A continuación el calendario de entregas del proyecto, el cual de acuerdo de las historias de usuario recogidas se estimó para unas 4 Iteraciones, y cada Iteración de 2 semanas.\\

\input{planificacion/calendario_entregas}

       También se definió el orden de implementación de las  \emph{historias de usuario} según la  prioridad, el esfuerzo y el valor de negocio que el cliente le asignó a las \emph{historias de usuario}, se puede ver en la tabla \ref{tab:user_stories_order},

      \begin{table}[H]

        \begin{center}
          \begin{tabular}{ c  c  c }
            \toprule
              \textbf{Iteración} &
              \textbf{Historia de Usuario} &
              \textbf{Estimación [dias]}\\

            \midrule
              \multirow{2}{*}{Iteración 1}
              & US01 & 6\\
              & US02 & 4\\

            \addlinespace
            \multirow{2}{*}{Iteración 2}
            & US03 & 5\\
            & US04 & 5\\

            % \midrule
            \addlinespace
              \multirow{2}{*}{Iteración 3}
              & US06 & 3\\
              & US07 & 4\\
            %
            % \midrule
            \addlinespace
              \multirow{3}{*}{Iteración 4}
              & US05 & 4\\
              & US08 & 4\\
              & US09 & 3\\

            \bottomrule
          \end{tabular}
          \caption{Estimación de la implementación de las Historias de Usuario.}
          \label{tab:user_stories_order}
        \end{center}
      \end{table}

% end implementacion
% \section{Iteraciones}

Una vez que la fase de \emph{planificación de entrega} es completado, se empieza con la fase de las \emph{iteraciones}.

\chapter{Iteración 1}
\label{chap:iteracion_1}

Para la primera iteración se implementaran las historias de usuario que tengan más relevancia dentro de la lógica de negocios para el cliente, generalmente son las que tienen mayor impacto en el sistema a desarrollar.


\section{Iteration Planning Meeting}
\label{sec:Iteration Planning Meeting}


Tomando en cuenta que el equipo de desarrollo está compuesto solo por mi persona, para el desarrollo del presente proyecto de grado la fase de Exploración y Planeación se procedió a realizarlas en la misma fase.

  \subsection{Exploración y Planeación}
  \label{sub:Exploración y Planeación}

Para la primera iteración se llevó a cabo una reunión para determinar las historias de usuario que se implementaran, y de acuerdo del impacto en el producto se determinó que las historias de usuario 2 y 3 serán las primeras en implementarse. \\

Posteriormente como tarea del desarrollador se procede a dividir las historias de usuario en Tareas de Ingeniería, en la tabla se determinaron las Tareas pertenecientes a la historia de usuario 2, dentro lo que es la planeación se debe repartir las tareas entre los desarrolladores, pero ya que el equipo de desarrollo se traduce a mi persona, todas las tareas recaen sobre mi responsabilidad, como parte de la planeación es necesario estimar las tareas, para lo cual se presenta la tabla \ref{tab:us02_tasks}. \\


  \subsection{Tareas del US02}
  \label{sub:us02_tasks}

    \begin{table}[H]
  \begin{center}
    \begin{tabularx}{\textwidth}{ c  X  C{2.3cm} }
      \toprule
        \textbf{Código} &
        \multicolumn{1}{c}{\textbf{Tarea}} &
        \textbf{Estimación [dias]}\\

      \midrule
        RF001
        &
        Crear un archivo shapefile con información inicial de lugares principales dentro el campus de la UMSS.
        &
        1 \\

      \addlinespace
        RF002
        &
        Crear una base de datos que pueda manejar información geoespacial.
        &
        1 \\

      \addlinespace
        RF003
        &
        Popular la base de datos creada en RF002 con la información de RF001.
        &
        0.5 \\

      \addlinespace
        RF004
        &
        El usuario puede ver una lista con los lugares creados.
        &
        2 \\

      \addlinespace
        RF005
        &
        El usuario deberá poder ingresar el nombre de un lugar para filtrar los lugares existentes.
        &
        0.5 \\

      \addlinespace
        RF006
        &
        El usuario deberá poder ver la información de un lugar al hacer tap sobre el nombre del lugar en la lista.
        &
        1 \\

      \addlinespace
        TS001
        &
        Crear pruebas de funcionalidad del US02.
        &
        1 \\

      \addlinespace
      \midrule
        & \multicolumn{1}{R{7cm}}{\textbf{Total: }}
        & 7 \\

      \bottomrule
    \end{tabularx}
    \caption{Tareas de la US02}
    \label{tab:us02_tasks}
  \end{center}
\end{table}


  \subsection{Tareas del US03}
  \label{sub:us03_tasks}

    
\begin{table}[H]
  \begin{center}
    \begin{tabularx}{\textwidth}{ c  X  C{2.3cm} }
      \toprule
        \textbf{Código} &
        \multicolumn{1}{c}{\textbf{Tarea}} &
        \textbf{Estimación [dias]}\\

      \midrule
        RF007
        &
        El usuario puede ver la descripción del lugar.
        &
        0.5 \\

      \addlinespace
        RF008
        &
        El usuario puede ver el teléfono del lugar.
        &
        0.5 \\

      \addlinespace
        RF009
        &
        El usuario puede ver en qué piso se encuentra el lugar.
        &
        0.5 \\

      \addlinespace
        RF010
        &
        El usuario puede ver una imagen del lugar.
        &
        1 \\

      \addlinespace
        TS002
        &
        Crear pruebas de funcionalidad del US03.
        &
        0.5 \\

      \addlinespace
      \midrule
        & \multicolumn{1}{R{7cm}}{\textbf{Total: }}
        & 3 \\

      \bottomrule
    \end{tabularx}
    \caption{Tareas de la US03}
    \label{tab:us03_tasks}
  \end{center}
\end{table}



  \subsection{Calendario de Entregas}
  \label{subs:schedule_1}

    % \begin{table}[!ht]
%
% \end{table}
\begin{table}[H]

  \begin{center}

\begin{ganttchart}[
  canvas/.append style={fill=none, draw=black!5, line width=.75pt},
  hgrid style/.style={draw=black!5, line width=.75pt},
  vgrid={*1{draw=black!5, line width=.75pt}},
  %today=0,
  % today label=Semana 3,
  today rule/.style={
    draw=black!64,
    dash pattern=on 3.5pt off 4.5pt,
    line width=1.5pt
  },
  today label font=\small\bfseries,
  title/.style={draw=none, fill=none},
  title label font=\bfseries\footnotesize,
  title label node/.append style={below=7pt},
  include title in canvas=false,
  bar label font=\mdseries\small\color{black!70},
  bar label node/.append style={left=2cm},
  bar/.append style={draw=none, fill=black!63},
  bar incomplete/.append style={fill=barblue},
  bar progress label font=\mdseries\footnotesize\color{black!70},
  group incomplete/.append style={fill=groupblue},
    group left shift=0,
    group right shift=0,
    group height=.5,
    group peaks tip position=0,
    group label node/.append style={left=.6cm},
    group progress label font=\bfseries\small,
    link/.style={-latex, line width=1.5pt, linkred},
    link label font=\scriptsize\bfseries,
    link label node/.append style={below left=-2pt and 0pt},
  ]{1}{12}
  \gantttitle{Calendario de Entregasde de la Iteración 2}{7} \\[grid]
  \gantttitle{Semana 1}{7}
  \gantttitle{Semana 2}{7} \\
  % \gantttitle{Noviembre}{4} \\
  \gantttitle[title label node/.append style={below left=7pt and -3pt}]{D\'ia:\quad15}{0}
  \gantttitlelist{16,...,30}{1} \\
  % \ganttgroup[progress=0]{Historias de Usuario}{1}{8} \\
  \ganttbar[
    progress=0,
    name=bar1
  ]{\textbf{User Story 04}}{1}{12} \\
  % \ganttbar[
  %   progress=0,
  %   name=bar2
  % ]{\textbf{User Story 03}}{10}{12} \\
  % \ganttbar[
  %   progress=0,
  %   name=bar3
  % ]{\textbf{Iteración 3}}{5}{6} \\
  % \ganttbar[
  %   progress=0,
  %   name=bar4
  % ]{\textbf{Iteración 4}}{7}{8} \\
  % \ganttbar[
  %   progress=100,
  %   name=bar5
  % ]{\textbf{Actividad 5}}{5}{7} \\
  % \ganttbar[
  %   progress=80,
  % ]{\textbf{Actividad 6}}{8}{8} \\
  % \ganttbar[
  %   progress=49,
  % ]{\textbf{Actividad 7}}{9}{11} \\
  % \ganttmilestone{Hito 1}{11}{11}  \\
  % \ganttmilestone{Hito 2}{12}{12} \\
  %

  % \ganttmilestone{Q6 report}{24}{24} \\
  % \ganttmilestone{M1: Project finished}{8}{8}

  % \ganttlink[link type=f-s]{bar1}{bar2}
  % \ganttlink[link type=f-s]{bar2}{bar3}
  % \ganttlink[link type=f-s]{bar3}{bar4}

\end{ganttchart}

\caption{Calendario de Entregas de la Iteración 2}
\label{tab:calendario_entregas_iteracion_2}

\end{center}
\end{table}



\section{Implementación}
\label{sec:implementacion_iteracion_1}

  A continuación se detalla los resultados de la implementación de cada tarea asignada.

\subsection{RF001}
\label{sub:RF001}


Para la presente Tarea se procedió a recolectar la información geográfica de los principales lugares del campus de la Universidad Mayor de San Simón, para tal efecto se hizo uso
de un GPS Garmin Nuvi 1300, el mismo que fue usado  para obtener el mapa de rutas.\\

Con la ayuda de QGIS fue que se exporto el archivo gpx generado por el GPS a un archivo Shapefile, el cual se usará para popular información dentro de la base de datos.\\

\begin{figure}[H]
  \begin{center}
    \caption{Shapefile de Lugares desplegados en QGIS}
    \label{fig:qgis_places}
    \includegraphics[width=1\textwidth]{qgis_places}
    \caption*{Fuente: Elaboración propia}
  \end{center}
\end{figure}

\subsection{RF002}
\label{sub:RF002}

Para poder determinar una base de datos se investigó las diferentes opciones disponibles que tenga la capacidad de manejar información geoespacial y tras la investigación que se puede apreciar en \ref{sec:base_de_datos}, se instaló PostgreSQL 9.4.8 sobre Linux Ubuntu 15.10, y para manejar datos geoespaciales se necesitó instalar PostGIS 2.1.8, para poder acceder a las librerías necesarias para almacenar datos geo-referenciados.\\

El resultado de esta tarea se puede apreciar en el manual de instalación.\\

\subsection{RF003}
\label{sub:RF003}

Para esta tarea se hizo uso de una herramienta disponible para postgres, \emph{shp2pgsql}, que permite la conversion de un archivo shapefile a un archivo sql.

\begin{verbatim}
  $ shp2pgsql -s 4326 -I -S -c -d ~/Documents/places.shp > places.sql
\end{verbatim}

Con el anterior comando se tiene como resultado un archivo .sql, el cual es ingresado en la base de datos ya configurada, de esta forma nuestra base de datos ya contiene una tabla geoespacial con datos de tipo POINT, los cuales representan los lugares dentro del campus de la UMSS.\\

El archivo sql resultante es cargado a la base de datos con el siguiente comando.\\

\begin{verbatim}
  $ psql -d db_ubikate -U db_admin -f /Documents/places.sql
\end{verbatim}

\begin{figure}[H]
  \begin{center}
    \caption{Herramienta grafica de PostgreSQL (pgAdmin) con la tabla de Lugares desplegada.}
    \label{fig:postgres_places}
    \includegraphics[width=1\textwidth]{iteration1/postgres_places}
    \caption*{Fuente: Elaboración propia}
  \end{center}
\end{figure}

En la figura \ref{fig:postgres_places} se puede observar que la columna \emph{Elevation} contiene datos que el GPS Garmin Nuvi 1300 genera al momento de guardar un punto, en el presente caso es irrelevante.\\

\subsection{RF004}
\label{sub:RF004}

Para completar esta tarea se procedió a instalar y configurar el framework de desarrollo Ember JS, que nos ayudará en la implementación del frontend de la aplicación o la capa que interactúa con el usuaria, y Express JS, el cual manejara el backend de la aplicación, básicamente se encarga de la lógica del sistema y la comunicación con la base de datos.\\

\begin{figure}[H]
  \begin{center}
    \caption{Vista de la lista de Lugares registrados en el sistema.}
    \label{fig:places_index}
    \includegraphics[width=0.5\textwidth]{iteration1/places_index}
    \caption*{Fuente: Elaboración propia}
  \end{center}
\end{figure}

\subsection{RF005}
\label{sub:RF005}

\begin{figure}[H]
  \begin{center}
    \caption{Vista de la búsqueda de lugares a través de un cajón de búsqueda.}
    \label{fig:places_search}
    \includegraphics[width=0.5\textwidth]{iteration1/places_search}
    \caption*{Fuente: Elaboración propia}
  \end{center}
\end{figure}


\subsection{RF006}
\label{sub:RF006}

Para implementar esta funcionalidad del sistema fue necesario utilizar las funcionalidad de Ember JS.

\begin{verbatim}
  {{#paper-item class="md-1-line" onClick=(transition-to 'places.show' place)}}
      <div class="md-list-item-text">
          <span>{{place.name}}</span>
      </div>
  {{/paper-item}}
\end{verbatim}

\subsection{RF007}
\label{sub:RF007}



\begin{figure}[H]
  \begin{center}
    \caption{Vista de la Información de un Lugar.}
    \label{fig:place_show}
    \includegraphics[width=0.5\textwidth]{iteration1/place_show}
    \caption*{Fuente: Elaboración propia}
  \end{center}
\end{figure}


\subsection{RF008}
\label{sub:RF008}
% RF008: El usuario puede ver el teléfono del lugar

\subsection{RF009}
\label{sub:RF009}
% RF009: El usuario puede ver en qué piso se encuentra el lugar

\subsection{RF010}
\label{sub:RF010}
 % RF010: El usuario puede ver una imagen del lugar
 El hecho de visualizar las imagenes dentro de la aplicacion se tiene q pensar en donde se van a guardar las imagenes
 Para realizar esta tarea se hizo uso de Cloudinary

\subsection{TS001}
\label{sub:TS001}

Pruebas funcionales


\subsection{TS002}
\label{sub:TS002}

Pruebas funcionales

\section{Registrar el Avance}
\label{sec:iteracion1_avance}

\section{Verificación}
\label{sec:iteracion1_verificacion}


\section{Iteración 2}
\label{sec:iteracion_2}

Después de que finalizó la primera iteración y ya que todas las pruebas pasaron exitosamente, se continúa con la segunda iteración.


\subsection{Iteration Planning Meeting}
\label{sub:iteration2_planning_meeting}


Al igual que la primera iteración, las fases de exploración y planeación se realizan al mismo tiempo, en el sentido que no es necesario repartir las tareas resultantes de la exploración, por lo tanto al mismo tiempo en que se determinan las tareas se puede realizar la estimación de las mismas.

  \subsubsection{Exploración y Planeación}
  \label{subs:iteration2_exploracion_planeacion}

  Para la segunda iteración se toman en cuenta todas las historias de usuario restantes, y de acuerdo al criterio de escoger las siguientes historias más relevantes y de mayor valor para el producto, se escogió la historia de usuario #4.\\

  Como parte de la fase de Exploración se toma la historia de usuario #4 y la dividimos en las Tareas de Ingeniería, las cuales serán trabajadas en la fase de la Implementación.

  En la siguente tabla se especificaran las tareas correspondientes a la historia de usuario #4 \ref{tab:us04_tasks}. \\

  \subsection{Tareas del US04}
  \label{sub:us04_tasks}

    \begin{table}[H]
  \begin{center}
    \begin{tabularx}{\textwidth}{ c  X  C{2.3cm} }
      \toprule
        \textbf{Código} &
        \multicolumn{1}{c}{\textbf{Tarea}} &
        \textbf{Estimación [dias]}\\

      \midrule
        RF011
        &
        Crear un archivo shapefile con información inicial de lugares principales dentro el campus de la UMSS.
        &
        2 \\

      \addlinespace
        RF012
        &
        Preparar la base de datos para manejar información geográfica de rutas.
        &
        1 \\

      \addlinespace
        RF013
        &
        Investigar e instalar una herramienta que permita usar un servicio de mapas.
        &
        1 \\

      \addlinespace
        RF014
        &
        El usuario puede ver un mapa usando un servicio del campus de la UMSS.
        &
        0.5 \\

      \addlinespace
        RF015
        &
        El usuario puede ver un marcador sobre el lugar.
        &
        0.5 \\

      \addlinespace
        RF016
        &
        El marcador tiene información básica del lugar, nombre, piso.
        &
        0.5 \\

      \addlinespace
        RF017
        &
        El usuario puede ver un marcador mostrando el lugar actual donde se encuentra (el usuario).
        &
        0.5 \\

      \addlinespace
        RF018
        &
        Desarrollar un módulo que encuentra la ruta más corta usando la base de datos con información geográfica ruteable de RF012.
        &
        2 \\

      \addlinespace
        RF019
        &
        El usuario puede ver una línea roja que une el marcador de la posición del usuario con el marcador del lugar.
        &
        1 \\

      \addlinespace
        TS003
        &
        Crear pruebas de funcionalidad del US04.
        &
        1 \\

      \addlinespace
      \midrule
        & \multicolumn{1}{R{7cm}}{\textbf{Total: }}
        & 10 \\

      \bottomrule
    \end{tabularx}
    \caption{Tareas del US04}
    \label{tab:us04_tasks}
  \end{center}
\end{table}


  \subsection{Calendario de Entregas}
  \label{subs:schedule_2}

    % \begin{table}[!ht]
%
% \end{table}
\begin{table}[H]

  \begin{center}

\begin{ganttchart}[
  canvas/.append style={fill=none, draw=black!5, line width=.75pt},
  hgrid style/.style={draw=black!5, line width=.75pt},
  vgrid={*1{draw=black!5, line width=.75pt}},
  %today=0,
  % today label=Semana 3,
  today rule/.style={
    draw=black!64,
    dash pattern=on 3.5pt off 4.5pt,
    line width=1.5pt
  },
  today label font=\small\bfseries,
  title/.style={draw=none, fill=none},
  title label font=\bfseries\footnotesize,
  title label node/.append style={below=7pt},
  include title in canvas=false,
  bar label font=\mdseries\small\color{black!70},
  bar label node/.append style={left=2cm},
  bar/.append style={draw=none, fill=black!63},
  bar incomplete/.append style={fill=barblue},
  bar progress label font=\mdseries\footnotesize\color{black!70},
  group incomplete/.append style={fill=groupblue},
    group left shift=0,
    group right shift=0,
    group height=.5,
    group peaks tip position=0,
    group label node/.append style={left=.6cm},
    group progress label font=\bfseries\small,
    link/.style={-latex, line width=1.5pt, linkred},
    link label font=\scriptsize\bfseries,
    link label node/.append style={below left=-2pt and 0pt},
  ]{1}{12}
  \gantttitle{Calendario de Entregasde de la Iteración 2}{7} \\[grid]
  \gantttitle{Semana 1}{7}
  \gantttitle{Semana 2}{7} \\
  % \gantttitle{Noviembre}{4} \\
  \gantttitle[title label node/.append style={below left=7pt and -3pt}]{D\'ia:\quad15}{0}
  \gantttitlelist{16,...,30}{1} \\
  % \ganttgroup[progress=0]{Historias de Usuario}{1}{8} \\
  \ganttbar[
    progress=0,
    name=bar1
  ]{\textbf{User Story 04}}{1}{12} \\
  % \ganttbar[
  %   progress=0,
  %   name=bar2
  % ]{\textbf{User Story 03}}{10}{12} \\
  % \ganttbar[
  %   progress=0,
  %   name=bar3
  % ]{\textbf{Iteración 3}}{5}{6} \\
  % \ganttbar[
  %   progress=0,
  %   name=bar4
  % ]{\textbf{Iteración 4}}{7}{8} \\
  % \ganttbar[
  %   progress=100,
  %   name=bar5
  % ]{\textbf{Actividad 5}}{5}{7} \\
  % \ganttbar[
  %   progress=80,
  % ]{\textbf{Actividad 6}}{8}{8} \\
  % \ganttbar[
  %   progress=49,
  % ]{\textbf{Actividad 7}}{9}{11} \\
  % \ganttmilestone{Hito 1}{11}{11}  \\
  % \ganttmilestone{Hito 2}{12}{12} \\
  %

  % \ganttmilestone{Q6 report}{24}{24} \\
  % \ganttmilestone{M1: Project finished}{8}{8}

  % \ganttlink[link type=f-s]{bar1}{bar2}
  % \ganttlink[link type=f-s]{bar2}{bar3}
  % \ganttlink[link type=f-s]{bar3}{bar4}

\end{ganttchart}

\caption{Calendario de Entregas de la Iteración 2}
\label{tab:calendario_entregas_iteracion_2}

\end{center}
\end{table}



\subsection{Implementación}
\label{sub:implementacion_iteracion_1}

Durante esta fase es donde se implementaran las tareas especificadas en la tabla \ref{tab:us04_tasks}.

\subsubsection{RF011}
\label{subs:RF011}

% Crear un archivo shapefile con la información geográfica de las rutas internas del campus de la UMSS

% Como ya se explico en \ref{sec:ruta_corta_umss}, esta tarea se llevo a cabo recabando la informacion geoespacial con un dispositivo GPS y exportando los datos resultantes a un archivo shapefile, el cual se puede apreciar en \ref{fig:shapefile_umss_v1}.

\subsubsection{RF012}
\label{subs:RF012}

% Preparar la base de datos para manejar información geográfica de rutas

% Para realizar esta tarea se procedió a instalar pgRouting, el resultado de esta tarea se puede apreciar en el manual de instalación ## \\
%
% Una vez configurada la base de datos se procede a cargar la misma con la informacion obtenida en RF011, para tal efecto es necesario primeramente crear una tabla que contendra los LINESTRING contenidos en el shapefile, esta operacion es similar a la realizada en la tarea - RF003 (\ref{sub:RF003}). Una vez que ya se tiene la tabla a la llamamos \emph{ways}, se necesita ejecutar un query propio de \emph{pgRouting} el cual tiene como objetivo analizar los datos geo-espaciales de la tabla y a\~nadirle una \emph{topologia}.
%
% Dentro lo que es la \emph{topologia geoespacial} existe una aplicacion que se lo conoce como \emph{topología de red}. La \emph{topología de red} representa las relaciones entre segmentos en una red lineal o una colección de segmentos de línea\cite{osgeo_journal_topology}.
% En un \emph{SIG} la topologia ayuda a mejorar el analisis de datos geo-espaciales, para resolver el problema de la ruta corta \emph{pgRouting} genera una \emph{topología de red} usando los datos que existen en la tabla \emph{ways}, es necesario ejecutar una instruccion, la que se muestra a continiacion y \emph{pgRouting} se encarga de llenar los datos que se pueden observar en la figura \ref{fig:postgres_ways}, las columnas \emph{source} y \emph{target} son populadas con el analisis topologico y en la figura \ref{fig:postgres_vertices}, se puede observar que la tabla \emph{ways\_vertices\_pgr} es creada enteramente en la ejecucion de la instruccion.
%
% \begin{verbatim}
%   select pgr_createTopology('ways', 0.00000001, 'geom', 'gid');
% \end{verbatim}
%
% \begin{figure}[H]
%   \begin{center}
%     \caption{Vista de la tabla \emph{ways} en la base de datos PostgreSQL.}
%     \label{fig:postgres_ways}
%     \includegraphics[width=1\textwidth]{iteration2/postgres_ways}
%     \caption*{Fuente: Elaboración propia}
%   \end{center}
% \end{figure}
%
% En la figura \ref{fig:postgres_ways} se puede apreciar que cada fila es una parte de la línea original obtenida por el dispositivo GPS y explisionada por QGIS, hay que notar que las columnas \emph{source} y \emph{target} hacen referencia a los nodos o vertices que la primera linea tiene en sus extremos, la primera linea o fila esta identificada por la columna \emph{gid}.\\
%
% En la siguiente figura \ref{fig:postgres_vertices} se observa la tabla \emph{ways\_vertices\_pgr} que contiene los vertices creados a partir del analisis de los datos en la tabla \emph{ways}.
%
% \begin{figure}[H]
%   \begin{center}
%     \caption{Vista de la tabla \emph{ways\_vertices\_pgr} en la base de datos PostgreSQL.}
%     \label{fig:postgres_vertices}
%     \includegraphics[width=1\textwidth]{iteration2/postgres_vertices}
%     \caption*{Fuente: Elaboración propia}
%   \end{center}
% \end{figure}
%
% Para entender los datos generados hay leer la informacion de las 2 tablas, por ejemplo en la primera  fila (gid 1) de la tabla \emph{ways}, se observa que el contenido de la columna \emph{source} es igual a \textbf{2} y \emph{target} es igual a \textbf{3}, eso quiere decir que los vertices del LINESTRING de la fila 1 son los vertices con \textbf{id} 2 y 3 respectivamente de la tabla \emph{ways\_vertices\_pgr}.\\
%
%
% Todo el conjunto de vertices y lineas de estas tablas se podria representar con una Matriz de adyacencias, explicada en \ref{sub:representacion_de_un_grafo}, y usada en la resolucion de la ruta mas corta, mas especificamente con el algoritmo de Dijkstra.

\subsubsection{RF013}
\label{subs:RF013}

% Investigar e instalar una herramienta que permita usar un servicio de mapas
% Durante la investigacion de esta tarea se encontro \emph{ember-leaflet}, una libreria o plugin que contiene las herramientas para poder cargar y usar un servicio de mapas.\\
%
% Para instalar esta libreria solo se necesita ejecutar el siguiente comando y posteriormente ya se puede empezar a utilizarla.\\
%
% \begin{verbatim}
%   $ ember install ember-leaflet
% \end{verbatim}
%
% El resultado de la investigacion puede apreciar en el marco teórico, en la sección que describe la librería, \emph{ember-leaflet}. \ref{sec:ember_js}

\subsubsection{RF014}
\label{subs:RF014}

% El usuario puede ver un mapa usando un servicio del campus de la UMSS

Para completar esta tarea se hizo uso de la herramienta \emph{ember-leaflet}, con la cual se puede desplegar un mapa en el browser y optimizada para dispositivos moviles.\\

\begin{verbatim}
  {{#leaflet-map lat=lat lng=lng zoom=zoom}}
    {{tile-layer
      url="http://{s}.tile.openstreetmap.fr/hot/{z}/{x}/{y}.png"
    }}
  {{/leaflet-map}}
\end{verbatim}

Con la anterior instrucción se accede al servicio de \emph{Open Street Maps}, de la cual obtenemos los datos necesarios para renderizar un mapa en el browser. Los atributos de \emph{lat} y \emph{lng} se acceden de la capa del controlador de la aplicacion, son la latitud y longitud respectivamente, la convinacion de ambos datos es la locacion donde se va a ubicar el renderizado del mapa.\\

% Esta librería es la nos ayudará a insertar fácilmente los marcadores que irán sobre los lugares o la líneas que mostraran la ruta más corta
%
% Como resultado de esta tarea se puede apreciar la siguiente figura,

\subsubsection{RF015}
\label{subs:RF015}
% El usuario puede ver un marcador sobre el lugar

Para completar esta tarea se continuó usando la librería \emph{ember-leaflet}, la cual permite que con la siguiente instrucción se despliegue un marcador sobre el mapa renderizado del API de \emph{Open Street Maps}.

\begin{verbatim}
  {{#marker-layer location=userLocation}} {{/marker-layer}}
\end{verbatim}

El resultado de la tarea se puede observar en la figura \ref{fig:baquita_place}.

\subsubsection{RF016}
\label{subs:RF016}
% marcador se  tiene información básica del lugar, nombre, piso

Para poder mostrar la informacion del lugar sobre el marcador creado en RF015 se hizo uso de la librería \emph{ember-leaflet}, al igual que dicha tarea, solo se necesito de una instruccion para poder desplegar la informacion necesaria.

\begin{verbatim}
  {{#marker-layer location=location}}
    h3>{{model.name}}</h3>
    {{model.description}}
    <strong>telf:</strong> {{model.phone}}
    <strong>piso#</strong> {{model.level}}
  {{/marker-layer}}
\end{verbatim}

En la figura \ref{fig:baquita_place} se puede apreciar el marcador con la información desplegada del lugar ``Baquita''.

\begin{figure}[H]
  \begin{center}
    \caption{Tooltip con la información de un lugar.}
    \label{fig:baquita_place}
    \includegraphics[width=1\textwidth]{iteration2/baquita_place}
    \caption*{Fuente: Elaboración propia.}
  \end{center}
\end{figure}


\subsubsection{RF017}
\label{subs:RF017}
% El usuario puede ver un marcador mostrando el lugar actual donde se encuentra (el usuario)

Para encontrar la locación del usuario se uso el API de geolocalización propio de HTML5, que en un smarthphone puede acceder y usar los recursos nativos de un smartphone, es necesaria la aceptacion del usuario mediante un mensaje que el navegador desplega, la locacion es encontrada mediante la triangulacion de Coordenadas por GPS (el mas exacto a la hora de encontrar la locacion del dispositvo), Wi-Fi, GSM o CDMA. Solo es necesaria la ejecucion de la siguente linea para ponder obetener la posición actual del usuario usando el API de geolocalización de HTML5.

\begin{verbatim}
  var coords = Geolocation.getCurrentPosition();
  var latitud = coords.latitude;
  var longitud = coords.longitude;
\end{verbatim}

La \emph{latitud} y \emph{longitud} obtenidas es fácilmente trasladado al mapa usando \emph{ember-leaflet} mediante un marcador, como se puede apreciar en la siguiente figura.

\begin{figure}[H]
  \begin{center}
    \caption{Tooltip con la latitud y longitud de la posición actual del usuario.}
    \label{fig:location_marker}
    \includegraphics[width=0.5\textwidth]{iteration2/location_marker}
    \caption*{Fuente: Elaboración propia.}
  \end{center}
\end{figure}

\subsubsection{RF018}
\label{subs:RF018}
% Desarrollar un módulo que encuentra la ruta más corta usando la base de datos con información geográfica ruteable de RF012

Durante esta tarea se investigó la mejor forma de encontrar la ruta más corta y se llegó a la conclusión de usar la combinación de \emph{Postgres + Postgis + pgRouting}, esta investigación se puede apreciar en el marco teórico.\\

Para hallar la ruta más corta se necesita usar las características de la base de datos para poder analizar los datos geográficos almacenados en la Tarea RF012, como el análisis que se requiere hacer es caro osea el costo de procesador para realizar los calculos necesarios es elevado, lo mas recomendable es que este trabajo sea realizado en el backend de la aplicacion por la base de datos.\\

Tambien hay que tomar en cuenta que la tierra no es plana y las líneas que en un mapa parecen lineas rectas, realmente no son rectas, porque el planeta Tierra es un esferoide oblato\footnote{Un \emph{esferoide oblato} (o elipsoide oblato) es un elipsoide de revolución obtenido por rotación de una elipse alrededor de su eje más corto.} entonces las lineas tienen la curvatura natural del planeta Tierra. En distancias largas esto tiene un gran impacto cuando se manejan mapas projectados, pero es cierto que para una área pequeña como es el campus de la Universidad de San Simón este problema no tiene un gran impacto pero no está demás en tomar en cuenta esta característica del análisis de datos geoespaciales, como se explico en el capitulo \ref{cha:geolocalizacion}, para el presente proyecto se usara el proyeccion \emph{SRID 3857}.\\

Una vez que se tienen en cuanta estas variables es necesaria la resolucion del problema de la ruta más corta, \emph{pgRouting} tiene varios métodos implementados para el analisis de datos geo-espaciales en la resolucion de este problema, para el presente proyecto se usara el algoritmo de \emph{Dijkstra}, explicado en el capítulo \ref{cha:ruta_optima}.\\

Tomando en cuenta los conceptos aprendidos y las herramientas investigadas es que se desarrollo el modulo que encuentra la ruta mas corta.

% La siguiente SQL query está diseñado y explicado en la documentación de pgRouting {ref - link}, básicamente se necesita especificar el nodo inicio y el nodo destino y la base de datos se encarga de analizar la tabla creada en RF012, para que el algoritmo de Dijkstra funcione hay que darle un Costo a cada uno de los tramos pertenecientes a la Matriz, en este caso el costo será la longitud del tramo(st_length(geom) AS cost),  el costo de ir del punto A al punto B puede no ser la misma que de ir del punto B al punto A a pesar de ser una única línea, por ejemplo si la circulación fuera en un solo sentido  como en el caso de las rutas para automoviles, en este caso como es una ruta peatonal se simplifica un poco el problema, entonces como datos de entrada tenemos el punto donde se encuentra el usuario extraído por RF017 y el punto del lugar buscado extraído por RF0##, tomando en cuenta estos datos se armó el siguiente query,

% var raw = "SELECT seq, id1 AS node, id2 AS edge, cost " +
%             "FROM pgr_dijkstra('SELECT gid AS id, source::integer, target::integer, st_length(geom) AS cost " +
%             "                   FROM public.ways', " + targetId + ", " + sourceId + ", false, false);";

\subsubsection{RF019}
\label{subs:RF019}
% El usuario puede ver una línea roja que une el marcador de la posición del usuario con el marcador del lugar
Como resultado de la tarea RF018 se tiene un conjunto de datos en formato de latitud y longitud que conforman líneas, las cuales representan la ruta más corta, pero al final es solo un montón de números, útiles pero para el usuario esta información es difícil de procesar, el usuario necesita información que sea fácil de entender y no existe mejor herramienta disponible para esta tarea que mostrar la \emph{ruta} de forma visual, esto quiere decir que se necesita mostrar la ruta sobre un \emph{mapa}, en la aplicación se usará \emph{ember-leaflet} para desplegar el mapa ofrecido por los servicios de OpenStreetMaps y también para mostrar ruta más corta mediante una línea de color rojo.\\

Para resolver esta tarea se creó un servicio API usando ExpressJS, la cual se encarga obtener la información extraída de la base de datos y transformarla en un objeto JSON (GeoJSON), este objeto contiene la información geoespacial necesaria para ``dibujar'' la línea roja entre 2 puntos georeferenciados, uno de los cuales es el lugar al que se quiere llegar y el otro es la ubicación actual del usuario. \\

\begin{verbatim}
  ENV.APP.API_HOST + '/api/v1/ways/route/' + sourceData.id + '/' + targetData.id;

  GET /api/v1/ways/route/930/77 200 276.217 ms - 3911

  $ curl http://localhost:3000/api/v1/ways/route/930/77 | python -m json.tool                                                       [3:04:52]
  % Total    % Received % Xferd  Average Speed   Time    Time     Time  Current
                                 Dload  Upload   Total   Spent    Left  Speed
100  3911  100  3911    0     0   161k      0 --:--:-- --:--:-- --:--:--  166k
{
    "features": [
        {
            "geometry": {
                "coordinates": [
                    [
                        -66.1467397848201,
                        -17.3935321732846
                    ],
                    [
                        -66.1467190789842,
                        -17.3935294725234
                    ]
                ],
                "type": "LineString"
            },
            "type": "Feature"
        },

\end{verbatim}
% /api/v1/ways/route/

% Este objeto es representado en el mapa usando ember-leaflet con la siguiente instrucción,
%
% {{#geojson-layer geoJSON=currentGeoJSON color='red' }}
% {{/geojson-layer}}

Y se puede observar en el mapa una línea roja que representa la ruta más corta entre el punto donde se encuentra el usuario y el punto del lugar a buscar.

\begin{figure}[H]
  \begin{center}
    \caption{Ruta más corta dibujada con una línea roja.}
    \label{fig:short_way_place}
    \includegraphics[width=0.5\textwidth]{iteration2/short_way_place}
    \caption*{Fuente: Elaboración propia.}
  \end{center}
\end{figure}

\subsubsection{TS004}
\label{subs:TS004}

Pruebas funcionales

\subsection{Registrar el Avance}
\label{sub:iteracion2_avance}

\subsection{Verificación}
\label{sub:iteracion2_verificacion}


\input{iteracion_3}

\section{Iteración 4}
\label{sec:iteracion_4}

% Al igual que al principio de la segunda iteración, se esperan los resultados de las pruebas realizadas para poder empezar con la planificación de la tercera iteración.
%
% \subsection{Iteration Planning Meeting}
% \label{sub:iteration2_planning_meeting}
%
%
% Los resultados de las pruebas realizadas se analizan para determinar si los criterios de aceptación, de las historias de usuario trabajadas en la segunda iteración, se cumplen para poder continuar con las historias que continúan sin ser desarrolladas, en caso que las pruebas fallen, es necesario continuar con la implementación de las historias inconclusas.\\
%
% En el caso del presente proyecto, las pruebas pasaron exitosamente y se aceptaron los criterios de aceptación de las historias de usuario trabajadas, por lo tanto se procede con la primera fase del “Iteration Planning”. \\
%

\subsection{Planificación}

% Esta fase generalmente se realiza en 2 pasos pero será realizada al mismo tiempo, ya que en la exploración se definen las tareas a realizar y en la planeación se asigna estas tareas al equipo de desarrollo, el cual tiene que estimar las tareas, pero como el equipo de desarrollo está compuesto por mi persona, puedo definir las tareas y asignarles una estimación en el mismo paso.\\
%
% En la Iteración 4, se trabajaron las historias de usuario 5, 8 y 9. A continuacion se puede ver las tareas de ingenieria con su respectiva estimacion para las historias de usuario ya mencionadas.


% \subsubsection{Tareas del US05}
% \label{sub:us05_tasks}

A continuación se analizará la \emph{historia de usuario} US05, ver el cuadro \ref{tab:US05}.

  
\begin{table}[H]
\begin{center}
\begin{tabularx}{0.75\textwidth}{ X }
 \toprule
 \textbf{Historia de Usuario:} US05
 \makebox[6cm][r]{\textbf{Prioridad:} Alta} \\
 \makebox[4cm][r]{}
 \makebox[6cm][r]{\textbf{Riesgo:} Alto} \\

 \addlinespace
 \textbf{Nombre:} Implementar el módulo de Registro de un Visitante.\\

 \addlinespace
 \textbf{Descripción:} \\
 \tab Yo como visitante.\\
 \tab Deseo registrarme. \\
 \tab Para tener mas opciones dentro el sistema. \\

 \addlinespace
 \textbf{Criterios de Aceptación:} \\
 \tab Quiero ver un formulario donde me pueda registrar. \\
 \tab Una vez registrado quiero poder ingresar al sistema con mis credenciales. \\
 \tab Quiero ver tener la posibilidad de editar mis datos. \\

 \bottomrule
\end{tabularx}
\caption{Historia de Usuario - US05}
\label{tab:US05}
\end{center}
\end{table}


  \begin{table}[H]
  \begin{center}
    \begin{tabularx}{0.75\textwidth}{ X }
      \toprule
      \textbf{Número de Tarea:} T028
      \makebox[1cm][r]{}
      \makebox[6cm][r]{\textbf{Historia de Usuario:} US05} \\

      \addlinespace
      \textbf{Descripción:} Generar un link para acceder al formulario de registro. \\

      \addlinespace
      \textbf{Tipo de Tarea:} Desarrollo
      \makebox[6cm][r]{\textbf{Estimación [dias]:} 0.5} \\

      \addlinespace
      \textbf{Programador Responsable:} Edmundo Figueroa \\

      \bottomrule
    \end{tabularx}
    \caption{Tarea de Ingeniería - T028}
    \label{tab:T028}
  \end{center}
\end{table}


\begin{table}[H]
  \begin{center}
    \begin{tabularx}{0.75\textwidth}{ X }
      \toprule
      \textbf{Número de Tarea:} T029
      \makebox[1cm][r]{}
      \makebox[6cm][r]{\textbf{Historia de Usuario:} US05} \\

      \addlinespace
      \textbf{Descripción:} Mostrar un formulario con el nombre, el email y la razon para registrarse. \\

      \addlinespace
      \textbf{Tipo de Tarea:} Desarrollo
      % \makebox[1cm][r]{}
      \makebox[6cm][r]{\textbf{Estimación [dias]:} 0.5} \\

      \addlinespace
      \textbf{Programador Responsable:} Edmundo Figueroa \\

      \bottomrule
    \end{tabularx}
    \caption{Tarea de Ingeniería - T029}
    \label{tab:T029}
  \end{center}
\end{table}

\begin{table}[H]
  \begin{center}
    \begin{tabularx}{0.75\textwidth}{ X }
      \toprule
      \textbf{Número de Tarea:} T030
      \makebox[1cm][r]{}
      \makebox[6cm][r]{\textbf{Historia de Usuario:} US05} \\

      \addlinespace
      \textbf{Descripción:} Implementar un modulo de Registro de Usuarios. \\

      \addlinespace
      \textbf{Tipo de Tarea:} Desarrollo
      \makebox[6cm][r]{\textbf{Estimación [dias]:} 2} \\

      \addlinespace
      \textbf{Programador Responsable:} Edmundo Figueroa \\

      \bottomrule
    \end{tabularx}
    \caption{Tarea de Ingeniería - T030}
    \label{tab:T030}
  \end{center}
\end{table}


% \subsubsection{Tareas del US08}
% \label{sub:us08_tasks}

A continuación se analizará la \emph{historia de usuario} US08, ver el cuadro \ref{tab:US08}.

  
\begin{table}[H]
\begin{center}
\begin{tabularx}{0.75\textwidth}{ X }
 \toprule
 \textbf{Historia de Usuario:} US08
 \makebox[6cm][r]{\textbf{Prioridad:} Alta} \\
 \makebox[4cm][r]{}
 \makebox[6cm][r]{\textbf{Riesgo:} Alto} \\

 \addlinespace
 \textbf{Nombre:} Implementar el módulo para Administrar Usuarios.\\

 \addlinespace
 \textbf{Descripción:} \\
 \tab Yo como usuario administrador.\\
 \tab Deseo administrar usuarios. \\
 \tab Para añadir o remover usuarios del sistema. \\

 \addlinespace
 \textbf{Criterios de Aceptación:} \\
 \tab Quiero ver los usuarios que desean registrarse en el sistema. \\
 \tab Quiero aceptar o rechazar solicitudes de registro. \\
 \tab Quiero eliminar usuarios que no usen el sistema de forma adecuada. \\

 \bottomrule
\end{tabularx}
\caption{Historia de Usuario - US08}
\label{tab:US08}
\end{center}
\end{table}


  \begin{table}[H]
  \begin{center}
    \begin{tabularx}{0.75\textwidth}{ X }
      \toprule
      \textbf{Número de Tarea:} T031
      \makebox[1cm][r]{}
      \makebox[6cm][r]{\textbf{Historia de Usuario:} US08} \\

      \addlinespace
      \textbf{Descripción:} Mostrar el Menú de Registros solo para usuario Administrador. \\

      \addlinespace
      \textbf{Tipo de Tarea:} Desarrollo
      \makebox[6cm][r]{\textbf{Estimación [dias]:} 0.5} \\

      \addlinespace
      \textbf{Programador Responsable:} Edmundo Figueroa \\

      \bottomrule
    \end{tabularx}
    \caption{Tarea de Ingeniería - T031}
    \label{tab:T031}
  \end{center}
\end{table}


\begin{table}[H]
  \begin{center}
    \begin{tabularx}{0.75\textwidth}{ X }
      \toprule
      \textbf{Número de Tarea:} T032
      \makebox[1cm][r]{}
      \makebox[6cm][r]{\textbf{Historia de Usuario:} US08} \\

      \addlinespace
      \textbf{Descripción:} Mostrar una lista con los Usuarios a registrar. \\

      \addlinespace
      \textbf{Tipo de Tarea:} Desarrollo
      % \makebox[1cm][r]{}
      \makebox[6cm][r]{\textbf{Estimación [dias]:} 1} \\

      \addlinespace
      \textbf{Programador Responsable:} Edmundo Figueroa \\

      \bottomrule
    \end{tabularx}
    \caption{Tarea de Ingeniería - T032}
    \label{tab:T032}
  \end{center}
\end{table}

\begin{table}[H]
  \begin{center}
    \begin{tabularx}{0.75\textwidth}{ X }
      \toprule
      \textbf{Número de Tarea:} T033
      \makebox[1cm][r]{}
      \makebox[6cm][r]{\textbf{Historia de Usuario:} US08} \\

      \addlinespace
      \textbf{Descripción:} Mostrar un botón para aceptar del registro del usuario. \\

      \addlinespace
      \textbf{Tipo de Tarea:} Desarrollo
      \makebox[6cm][r]{\textbf{Estimación [dias]:} 0.5} \\

      \addlinespace
      \textbf{Programador Responsable:} Edmundo Figueroa \\

      \bottomrule
    \end{tabularx}
    \caption{Tarea de Ingeniería - T033}
    \label{tab:T033}
  \end{center}
\end{table}

\begin{table}[H]
  \begin{center}
    \begin{tabularx}{0.75\textwidth}{ X }
      \toprule
      \textbf{Número de Tarea:} T034
      \makebox[1cm][r]{}
      \makebox[6cm][r]{\textbf{Historia de Usuario:} US08} \\

      \addlinespace
      \textbf{Descripción:} Mostrar un botón para Rechazar del registro del usuario. \\

      \addlinespace
      \textbf{Tipo de Tarea:} Desarrollo
      \makebox[6cm][r]{\textbf{Estimación [dias]:} 0.5} \\

      \addlinespace
      \textbf{Programador Responsable:} Edmundo Figueroa \\

      \bottomrule
    \end{tabularx}
    \caption{Tarea de Ingeniería - T034}
    \label{tab:T034}
  \end{center}
\end{table}

\begin{table}[H]
  \begin{center}
    \begin{tabularx}{0.75\textwidth}{ X }
      \toprule
      \textbf{Número de Tarea:} T035
      \makebox[1cm][r]{}
      \makebox[6cm][r]{\textbf{Historia de Usuario:} US08} \\

      \addlinespace
      \textbf{Descripción:} Mostrar una lista con Usuarios Registrados. \\

      \addlinespace
      \textbf{Tipo de Tarea:} Desarrollo
      \makebox[6cm][r]{\textbf{Estimación [dias]:} 0.5} \\

      \addlinespace
      \textbf{Programador Responsable:} Edmundo Figueroa \\

      \bottomrule
    \end{tabularx}
    \caption{Tarea de Ingeniería - T035}
    \label{tab:T035}
  \end{center}
\end{table}

\begin{table}[H]
  \begin{center}
    \begin{tabularx}{0.75\textwidth}{ X }
      \toprule
      \textbf{Número de Tarea:} T036
      \makebox[1cm][r]{}
      \makebox[6cm][r]{\textbf{Historia de Usuario:} US08} \\

      \addlinespace
      \textbf{Descripción:} Implementar un botón para eliminar Usuarios. \\

      \addlinespace
      \textbf{Tipo de Tarea:} Desarrollo
      \makebox[6cm][r]{\textbf{Estimación [dias]:} 0.5} \\

      \addlinespace
      \textbf{Programador Responsable:} Edmundo Figueroa \\

      \bottomrule
    \end{tabularx}
    \caption{Tarea de Ingeniería - T036}
    \label{tab:T036}
  \end{center}
\end{table}


  % \subsubsection{Tareas del US09}
  % \label{sub:us09_tasks}

  A continuación se analizará la \emph{historia de usuario} US09, ver el cuadro \ref{tab:US09}.

    
\begin{table}[H]
\begin{center}
\begin{tabularx}{0.75\textwidth}{ X }
\toprule
\textbf{Historia de Usuario:} US09
\makebox[6cm][r]{\textbf{Prioridad:} Alta} \\
\makebox[4cm][r]{}
\makebox[6cm][r]{\textbf{Riesgo:} Alto} \\

\addlinespace
\textbf{Nombre:} Implementar el reporte de los lugares más visitados.\\

\addlinespace
\textbf{Descripción:} \\
\tab Yo como usuario administrador.\\
\tab Deseo ver los lugares más visitados. \\
\tab Para obtener información y estadísticas de los lugares dentro del campus Universitario. \\

\addlinespace
\textbf{Criterios de Aceptación:} \\
\tab Quiero ver la cantidad de veces que los usuarios buscan un lugar. \\
\tab Quiero poder guardar el reporte. \\

\bottomrule
\end{tabularx}
\caption{Historia de Usuario - US09}
\label{tab:US09}
\end{center}
\end{table}


    \begin{table}[H]
  \begin{center}
    \begin{tabularx}{0.75\textwidth}{ X }
      \toprule
      \textbf{Número de Tarea:} T037
      \makebox[1cm][r]{}
      \makebox[6cm][r]{\textbf{Historia de Usuario:} US09} \\

      \addlinespace
      \textbf{Descripción:} Implementar un módulo de Reportes. \\

      \addlinespace
      \textbf{Tipo de Tarea:} Desarrollo
      \makebox[6cm][r]{\textbf{Estimación [dias]:} 1} \\

      \addlinespace
      \textbf{Programador Responsable:} Edmundo Figueroa \\

      \bottomrule
    \end{tabularx}
    \caption{Tarea de Ingeniería - T037}
    \label{tab:T037}
  \end{center}
\end{table}


\begin{table}[H]
  \begin{center}
    \begin{tabularx}{0.75\textwidth}{ X }
      \toprule
      \textbf{Número de Tarea:} T038
      \makebox[1cm][r]{}
      \makebox[6cm][r]{\textbf{Historia de Usuario:} US09} \\

      \addlinespace
      \textbf{Descripción:} Mostrar el menú de Reportes. \\

      \addlinespace
      \textbf{Tipo de Tarea:} Desarrollo
      % \makebox[1cm][r]{}
      \makebox[6cm][r]{\textbf{Estimación [dias]:} 0.5} \\

      \addlinespace
      \textbf{Programador Responsable:} Edmundo Figueroa \\

      \bottomrule
    \end{tabularx}
    \caption{Tarea de Ingeniería - T038}
    \label{tab:T038}
  \end{center}
\end{table}

\begin{table}[H]
  \begin{center}
    \begin{tabularx}{0.75\textwidth}{ X }
      \toprule
      \textbf{Número de Tarea:} T039
      \makebox[1cm][r]{}
      \makebox[6cm][r]{\textbf{Historia de Usuario:} US09} \\

      \addlinespace
      \textbf{Descripción:} Mostrar el Reporte de Frecuencia. \\

      \addlinespace
      \textbf{Tipo de Tarea:} Desarrollo
      \makebox[6cm][r]{\textbf{Estimación [dias]:} 0.5} \\

      \addlinespace
      \textbf{Programador Responsable:} Edmundo Figueroa \\

      \bottomrule
    \end{tabularx}
    \caption{Tarea de Ingeniería - T039}
    \label{tab:T039}
  \end{center}
\end{table}

\begin{table}[H]
  \begin{center}
    \begin{tabularx}{0.75\textwidth}{ X }
      \toprule
      \textbf{Número de Tarea:} T040
      \makebox[1cm][r]{}
      \makebox[6cm][r]{\textbf{Historia de Usuario:} US09} \\

      \addlinespace
      \textbf{Descripción:} Mostrar un botón para Guardar el Reporte generado. \\

      \addlinespace
      \textbf{Tipo de Tarea:} Desarrollo
      \makebox[6cm][r]{\textbf{Estimación [dias]:} 0.5} \\

      \addlinespace
      \textbf{Programador Responsable:} Edmundo Figueroa \\

      \bottomrule
    \end{tabularx}
    \caption{Tarea de Ingeniería - T040}
    \label{tab:T040}
  \end{center}
\end{table}



    
\subsection{Diseño}


\begin{itemize}

  \item \textbf{Diagrama Entidad - Relación:}

En la figura \ref{fig:er_ubikate}, se observa el diagrama Entidad - Relación de la aplicación.


\begin{figure}[H]
  \begin{center}
    \includegraphics[width=0.8\textwidth]{diagramas/er_ubikate}
  \end{center}
  \caption{Diagrama ER: Ubikate UMSS}
  \label{fig:er_ubikate}
  \caption*{Fuente: Elaboración propia}
\end{figure}


\item \textbf{Diagrama de Secuencia:}

En la figura \ref{fig:sequence_registrar_usuario}, se observa el diagrama de secuencia correspondiente al registro de un usuario.

\begin{figure}[H]
  \begin{center}
    \includegraphics[width=0.9\textwidth]{diagramas/sequence_registrar_usuario}
  \end{center}
  \caption{Diagrama de Secuencia: Registrar Usuarios}
  \label{fig:sequence_registrar_usuario}
  \caption*{Fuente: Elaboración propia}
\end{figure}

En la figura \ref{fig:sequence_reporte}, se observa el diagrama de secuencia correspondiente a la obtención del reporte.

\begin{figure}[H]
  \begin{center}
    \includegraphics[width=0.9\textwidth]{diagramas/sequence_reporte}
  \end{center}
  \caption{Diagrama de Secuencia: Reporte}
  \label{fig:sequence_reporte}
  \caption*{Fuente: Elaboración propia}
\end{figure}


\item \textbf{Diagrama de Clases:}

En la figura \ref{fig:clases_usuarios}, se observa el diagrama de clases correspondiente al registro de usuarios.

\begin{figure}[H]
\begin{center}
  \includegraphics[width=0.9\textwidth]{diagramas/clases_usuarios}
\end{center}
\caption{Diagrama de Clases: Registro de Usuario}
\label{fig:clases_usuarios}
\caption*{Fuente: Elaboración propia}
\end{figure}


\end{itemize}


    \subsection{Implementación}

    \subsubsection{Registro de Usuarios}

El registro de Usuarios sera mediante un formulario con el nombre, el email y la razon del registrante, y solamente el usuario administrador sera capaz de aceptar o rechazar la solicitud de registro, para lo cual se implemento en la aplicacion el \emph{formulario de registro} y el menu de \emph{menu de registros}. \\



\begin{description}
  \item[Formulario de Registro:] Para completar esta tarea se implemento en la vista de la aplicacion el formulario, ver la figura \ref{fig:registro_form} usando \emph{ember-paper}, que como ya se menciono es la libreria usada para implementar la vista de la aplicacion dandole el ``look and feel'' de una aplicacion movil. \\

  \begin{figure}[H]
        \begin{center}
          \includegraphics[width=0.5\textwidth]{registro_form}

          \caption{Formulario para Registro de Usuario}
          \label{fig:registro_form}
          \caption*{Fuente: Elaboración propia.}
        \end{center}
  \end{figure}


  Posteriormente se implemento en el \emph{backend} de la aplicacion el modulo para guardar la solicitud de registro, para lo cual se  anadio el \emph{endpoint} en el API que escuche la peticion POST generada al aceptar el \emph{formulario de registro}, este \emph{endpoint} inserta los datos colectados por el navegador en la base de datos para que esten disponibles posteriormente en el \emph{Menu de Registros}.\\


  \item[Menu de Registros] El \emph{menu de registro} sera donde un usuario administrador puede ver todas las solicitudes de registro, y de acuerdo de la \emph{razon} de registro se puede determinar si la solicitud es de parte de un usuario responsable, en caso de necesitar mas informacion del registrante se puede usar el \emph{email} enviado, una ves que el administrador considera que el registrante no va a jugar en el sistema para asegurar la integridad de la misma, puede \emph{aceptar} el registro, de esta forma el usuario recibira un email donde se confirma el registro al sistema.\\





\end{description}


\subsubsection{Generacion del Reporte}

Para la implementacion del modulo que genere un reporte, se tuvo que modificar la tabla de los lugares, para saber cuantas veces un ``lugar'' es buscado, con esta informacion se puede proporcionar un reporte de frecuencia, para tal efecto se uso \emph{ember-charts}, que como se puede observar en el codigo \ref{chart_template}, solo se necesita una linea para crear el reporte de frecuencia de la figura \ref{fig:reporte} \\

\begin{center}
  \begin{lstlisting}[label=chart_template,caption=Componente de \emph{Ember-charts}.]

{{horizontal-bar-chart data=model selectedSeedColor="green" sortAscending=false}}

  \end{lstlisting}
\end{center}

\begin{figure}[H]
      \begin{center}
        \includegraphics[width=0.5\textwidth]{reporte}

        \caption{Reporte de Frecuencia}
        \label{fig:reporte}
        \caption*{Fuente: Elaboración propia.}
      \end{center}
\end{figure}



Este reporte es obtiene al conocer la cantidad de veces que los usuarios han buscado un ``lugar'' mediante el SQL query, que se puede ver en el codigo \ref , el cual discrimina los primeros 10 lugares mas buscados dentro del campus universitario, ya que al existir tantas aulas y oficinas el reporte se haria muy extenso. \\

\begin{center}
  \begin{lstlisting}[label=request_visited,caption=Insertar un ``lugar'' en la base de datos.]

    router.get('places/visited/count',  (req, res) => {
      Place.forge()
        .where('visit_count', '>', '1')
        .orderBy('visit_count', 'DESC')
        .query((qb) => qb.limit(10))
        .fetchAll({columns: ["name as label", "visit_count as value"]})
        .then((visited) => {
            res.json(visited.toJSON());
        })
        .catch((err) => {
            res.status(500);
        });
    });

  \end{lstlisting}
\end{center}


  \subsection{Pruebas de Aceptación}


    \begin{table}[H]
  \begin{center}
    \begin{tabularx}{0.75\textwidth}{ X }
      \toprule
      \textbf{Codigo:} CP008
      \makebox[3cm][r]{}
      \makebox[6cm][r]{\textbf{Historia de Usuario:} US005} \\

      \addlinespace
      \textbf{Nombre:} Verificar el Formulario de Registro \\

      \addlinespace
      \textbf{Descripción:} Validar que un usuario puede ver el formulario de registro. \\

      \addlinespace
      \textbf{Condiciones de Ejecución:} El usuario no está registrado. \\

      \addlinespace
      \textbf{Entradas / Pasos de Ejecución:}  \\
      \tab \textbf{1.} Seleccionar el link \emph{Registrarse}. \\
      \tab \textbf{2.} Ingresar el Nombre del usuario.\\
      \tab \textbf{3.} Ingresar el Email del usuario.\\
      \tab \textbf{4.} Ingresar una Razón de registro.\\
      \tab \textbf{5.} Aceptar el envío del formulario.\\


      \addlinespace
      \textbf{Resultado Esperado:} El Usuario recibe un email de confirmación de Email.  \\

      \addlinespace
      \textbf{Evaluación de la Prueba:} Prueba exitosa. \\

      \bottomrule
    \end{tabularx}
    \caption{Caso de Prueba - CP008}
    \label{tab:CP008}
  \end{center}
\end{table}


\begin{table}[H]
  \begin{center}
    \begin{tabularx}{0.75\textwidth}{ X }
      \toprule
      \textbf{Codigo:} CP009
      \makebox[3cm][r]{}
      \makebox[6cm][r]{\textbf{Historia de Usuario:} US008} \\

      \addlinespace
      \textbf{Nombre:} Verificar el Registro del Usuario \\

      \addlinespace
      \textbf{Descripción:} La prueba verifica que un usuario se puede registrar en el sistema.\\

      \addlinespace
      \textbf{Condiciones de Ejecución:} El usuario ha ingresado datos válidos a la solicitud de registro.  \\

      \addlinespace
      \textbf{Entradas / Pasos de Ejecución:}  \\
      \tab \textbf{1.} El Administrador ingresa al \emph{Menú de Registros}. \\
      \tab \textbf{2.} Confirma la validez de los datos de registro del usuario.\\
      \tab \textbf{3.} Acepta el registro del usuario.\\

      \addlinespace
      \textbf{Resultado Esperado:} El usuario recibe un email de confirmación de registro. \\

      \addlinespace
      \textbf{Evaluación de la Prueba:} Prueba exitosa. \\

      \bottomrule
    \end{tabularx}
    \caption{Caso de Prueba - CP009}
    \label{tab:CP009}
  \end{center}
\end{table}

\begin{table}[H]
  \begin{center}
    \begin{tabularx}{0.75\textwidth}{ X }
      \toprule
      \textbf{Codigo:} CP010
      \makebox[3cm][r]{}
      \makebox[6cm][r]{\textbf{Historia de Usuario:} US009} \\

      \addlinespace
      \textbf{Nombre:} Verificar el Reporte de frecuencia. \\

      \addlinespace
      \textbf{Descripción:} La prueba verifica que el usuario puede ver el reporte de frecuencia.\\

      \addlinespace
      \textbf{Condiciones de Ejecución:} El usuario debe tener permisos de Administrador.  \\

      \addlinespace
      \textbf{Entradas / Pasos de Ejecución:}  \\
      \tab \textbf{1.} El usuario ingresa al sistema con permisos de Administrador. \\
      \tab \textbf{2.} Selecciona al menú de \emph{Reportes}. \\

      \addlinespace
      \textbf{Resultado Esperado:} El usuario puede ver la gráfica de barras con los lugares más visitados. \\

      \addlinespace
      \textbf{Evaluación de la Prueba:} Prueba exitosa. \\

      \bottomrule
    \end{tabularx}
    \caption{Caso de Prueba - CP010}
    \label{tab:CP010}
  \end{center}
\end{table}


    % \newpage
\subsubsection{Resultado de las pruebas de la Iteración 4}

Al finalizar la Iteración 4 se ejecutaron todas las pruebas escritas durante la presente y las anteriores iteraciones, en el cuadro \ref{tab:regresion_4} se puede ver el detalle.


\LTXtable{0.8\textwidth}{iteration4/regresion_table}



\begin{itemize}
  \item Se ejecutaron 11 pruebas de funcionalidad positiva, todas pasaron exitosamente.
  \item Se ejecutaron 5 pruebas de funcionalidad negativa, la prueba CP014 falló pero está documentada como un \emph{problema conocido}.
  \item Se ejecutaron 3 prueba de usabilidad, todas pasaron exitosamente.
\end{itemize}




% end desarrollo
