\section{Iteración 3}
\label{sec:iteracion_3}

Al igual que al principio de la segunda iteración, se esperan los resultados de las pruebas realizadas para poder empezar con la planificación de la tercera iteración.

\subsection{Iteration Planning Meeting}
\label{sub:iteration2_planning_meeting}


Los resultados de las pruebas realizadas se analizan para determinar si los criterios de aceptación, de las historias de usuario trabajadas en la segunda iteración, se cumplen para poder continuar con las historias que continúan sin ser desarrolladas, en caso que las pruebas fallen, es necesario continuar con la implementación de las historias inconclusas.\\

En el caso del presente proyecto, las pruebas pasaron exitosamente y se aceptaron los criterios de aceptación de las historias de usuario trabajadas, por lo tanto se procede con la primera fase del “Iteration Planning”. \\


\subsection{Exploración y Planeación}
\label{sub:iteration2_exploracion_planeacion}

Esta fase generalmente se realiza en 2 pasos pero será realizada al mismo tiempo, ya que en la exploración se definen las tareas a realizar y en la planeación se asigna estas tareas al equipo de desarrollo, el cual tiene que estimar las tareas, pero como el equipo de desarrollo está compuesto por mi persona, puedo definir las tareas y asignarles una estimación en el mismo paso.\\

Para la Iteración 3, se trabajarán las historias de usuario 5 y 6, de las cuales serán definidas sus tareas de desarrollo en las siguientes tablas.


\subsection{Tareas del US05}
\label{sub:us05_tasks}

  \begin{table}[H]
  \begin{center}
    \begin{tabularx}{\textwidth}{ c  X  C{2.3cm} }
      \toprule
        \textbf{Código} &
        \multicolumn{1}{c}{\textbf{Tarea}} &
        \textbf{Estimación [dias]}\\

      \midrule
        RF020
        &
        El usuario podrá ver un link hacia el formulario para añadir más lugares desde la lista de lugares existentes.
        &
        0.5 \\

      \addlinespace
        RF021
        &
        El usuario podrá ver un formulario para añadir un lugar con información básica. por ejemplo, el nombre del lugar, descripción, teléfono.
        &
        1 \\

      \addlinespace
        RF022
        &
        El usuario deberá estar cerca del lugar que desea añadir para poder georeferenciarlo.
        &
        1 \\

      \addlinespace
        RF023
        &
        Un usuario no-administrador no debería poder ver el formulario para añadir lugares.
        &
        1 \\


      \addlinespace
        TS004
        &
        Crear pruebas de funcionalidad del US05.
        &
        0.5 \\

      \addlinespace
      \midrule
        & \multicolumn{1}{R{7cm}}{\textbf{Total: }}
        & 4 \\

      \bottomrule
    \end{tabularx}
    \caption{Tareas del US05}
    \label{tab:us05_tasks}
  \end{center}
\end{table}


\subsection{Tareas del US06}
\label{sub:us06_tasks}

  \begin{table}[H]
  \begin{center}
    \begin{tabularx}{\textwidth}{ c  X  C{2.3cm} }
      \toprule
        \textbf{Código} &
        \multicolumn{1}{c}{\textbf{Tarea}} &
        \textbf{Estimación [dias]}\\

      \midrule
        T020
        &
        El usuario podrá ver un link hacia el formulario para añadir más lugares desde la lista de lugares existentes.
        &
        0.5 \\

      \addlinespace
        T021
        &
        El usuario podrá ver un formulario para añadir un lugar con información básica. por ejemplo, el nombre del lugar, descripción, teléfono.
        &
        1 \\

      \addlinespace
        T022
        &
        El usuario deberá estar cerca del lugar que desea añadir para poder georeferenciarlo.
        &
        1 \\

      \addlinespace
        T023
        &
        Un usuario no-administrador no debería poder ver el formulario para añadir lugares.
        &
        1 \\


      \addlinespace
        P004
        &
        Crear pruebas de funcionalidad del US05.
        &
        0.5 \\

      \addlinespace
      \midrule
        & \multicolumn{1}{R{7cm}}{\textbf{Total: }}
        & 4 \\

      \bottomrule
    \end{tabularx}
    \caption{Tareas del US05}
    \label{tab:us05_tasks}
  \end{center}
\end{table}

